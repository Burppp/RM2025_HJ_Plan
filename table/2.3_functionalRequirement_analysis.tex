%%% 本文档作为功能演示

%%% 表格演示-BEGIN
\begin{longtable}{ X | X | X |}

    \hline

    %%% 此处标识以上为表格脚,即在每页的表格底部重复显示的内容(一条横线)
    \endfoot
    
    %%% 此处标识该行颜色为指定定义的颜色,tabhdcolor 在 ./introduction.tex 中定义,颜色为hsb(0, 0, 0.82353),即82度灰
    \rowcolor{tabhdcolor}

        %%% 第1行的表格信息
        功能 &
        需求分析 &
        设计思路
        %%% 第1行的第3~4列单元格被合并,且左侧有分割线,而右侧没有
        %\multicolumn{2}{| c }{header 1 3-4} \\

    %%% 此处完整分割一行
    \hline

    %%% 此处标识以上为表格头,即在每页的表格头部重复显示的内容(一堆header)
    \endhead

        %%% 第2行的表格信息
        content 2 1 &
        content 2 2 &
        content 2 3 \\
        
    \hline

        %%% 第3行的表格信息
        content 3 1 &
        %%% 第3行的第2~3列单元格被合并,左对齐,且左右均有分割线
        %\multicolumn{2}{| l |}{content 3 2-3} &
        content 3 4 \\
        
    \hline
    
        %%% 第4行的表格信息
        %%% 第4行开始,连续2行的第1列单元格被合并,自动列宽
        \multirow{2}{*}{content 4-5 1} &
        content 4 2 &
        content 4 3 \\
        %content 4 4 \\

    %%% 此处分割一行的第2至3个单元格
    \cline{2-3}
    
        %%% 第5行的表格信息
        %%% 第5行由于第1列在上一行已被合并,所以建议不要写入信息,虽然写入信息也会显示出
        &
        content 5 2 &
        content 5 3 \\
        %content 5 4 \\
        
    \hline
    
        %%% 第6行的表格信息
        %content 6 1 &
        %%% 第6行开始,连续2行的第2~4列单元格被合并,居中对齐,且左侧有分割线,自动列宽
        %\multicolumn{3}{| c }{\multirow{2}{*}{content 6-7 2-4}} \\
        &
        &
        \\

    %%% 此处分割一行的第1至1个单元格,即只有第一个单元格下面有横线
    \cline{1-1}
    
        %%% 第7行的表格信息
        content 7 1 &
        %%% 第7行由于第2~4列在上一行已被合并,所以建议不要写入信息,虽然写入信息也会显示出,而且建议像下面一样合并表格防止竖线分割
        %\multicolumn{3}{| c }{} \\
        &
        \\
        
    \hline
    
        %%% 第8行的表格信息
        这个单元格被长文本挤满啦这个单元格被长文本挤满啦这个单元格被长文本挤满啦这个单元格被长文本挤满啦这个单元格被长文本挤满啦这个单元格被长文本挤满啦这个单元格被长文本挤满啦这个单元格被长文本挤满啦这个单元格被长文本挤满啦这个单元格被长文本挤满啦这个单元格被长文本挤满啦这个单元格被长文本挤满啦这个单元格被长文本挤满啦这个单元格被长文本挤满啦。 &
        &
        \\
        
    \hline
    
        %%% 第9行的表格信息
        &
        &
        \\
        
    \hline
    
        %%% 第10行的表格信息
        &
        &
        \\
        
    \hline
    
        %%% 第11行的表格信息
        &
        &
        \\
        
    \hline
    
\end{longtable}
%%% 表格演示-END
%%% 其中,开头的 { X | X | X | X |} 中,竖线标识分割线,字母X为自动对齐,其余可替换的lrc分别为左对齐、右对齐、居中对齐
%%% 该表格的一大特性是跨页后仍保留表头内容,即 \endhead 前的内容