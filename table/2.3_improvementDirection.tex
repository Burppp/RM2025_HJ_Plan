%%% 本文档作为功能演示

%%% 表格演示-BEGIN
\begin{longtable}{ X | X | X | X |}

    \hline

    %%% 此处标识以上为表格脚,即在每页的表格底部重复显示的内容(一条横线)
    \endfoot
    
    %%% 此处标识该行颜色为指定定义的颜色,tabhdcolor 在 ./introduction.tex 中定义,颜色为hsb(0, 0, 0.82353),即82度灰
    \rowcolor{tabhdcolor}

        %%% 第1行的表格信息
        组别 &
        功能 &
        设计需求分析 &
        设计思路\\
        %%% 第1行的第3~4列单元格被合并,且左侧有分割线,而右侧没有
        %\multicolumn{2}{| c }{header 1 3-4} \\

    %%% 此处完整分割一行
    \hline

    %%% 此处标识以上为表格头,即在每页的表格头部重复显示的内容(一堆header)
    \endhead

        %%% 第2行的表格信息
        \multirow{3}{*}{机械组} &
        content 2 2 &
        content 2 3 &
        content 2 4\\
        
    %\hline
        &
        %%% 第3行的表格信息
        content 3 1 &
        %%% 第3行的第2~3列单元格被合并,左对齐,且左右均有分割线
        %\multicolumn{2}{| l |}{content 3 2-3} &
        &
        content 3 4 \\

        &
        &
        &
        \\
        
    \hline
    
        %%% 第4行的表格信息
        %%% 第4行开始,连续2行的第1列单元格被合并,自动列宽
        \multirow{3}{*}{电控组} &
        content 4 2 &
        content 4 3 &
        content 4 4 \\

    %%% 此处分割一行的第2至3个单元格
    %\cline{2-3}
    
        %%% 第5行的表格信息
        %%% 第5行由于第1列在上一行已被合并,所以建议不要写入信息,虽然写入信息也会显示出
        &
        content 5 2 &
        content 5 3 &
        content 5 4 \\

        &
        &
        &
        \\
        
    \hline
    
        %%% 第6行的表格信息
        \multirow{3}{*}{视觉组} &
        %%% 第6行开始,连续2行的第2~4列单元格被合并,居中对齐,且左侧有分割线,自动列宽
        %\multicolumn{3}{| c }{\multirow{2}{*}{content 6-7 2-4}} \\
        &
        &
        \\

    %%% 此处分割一行的第1至1个单元格,即只有第一个单元格下面有横线
    %\cline{1-1}
    
        %%% 第7行的表格信息
        %content 7 1 &
        %%% 第7行由于第2~4列在上一行已被合并,所以建议不要写入信息,虽然写入信息也会显示出,而且建议像下面一样合并表格防止竖线分割
        %\multicolumn{3}{| c }{} \\

        &
        &
        &
        \\

        &
        &
        &
        \\
        
    \hline

        \multirow{3}{*}{硬件组} &
        &
        &
        \\

        &
        &
        &
        \\

        &
        &
        &
        \\
    
\end{longtable}
%%% 表格演示-END
%%% 其中,开头的 { X | X | X | X |} 中,竖线标识分割线,字母X为自动对齐,其余可替换的lrc分别为左对齐、右对齐、居中对齐
%%% 该表格的一大特性是跨页后仍保留表头内容,即 \endhead 前的内容