%%% 本文档作为功能演示

%%% 表格演示-BEGIN
\begin{longtable}{ p{2cm} | p{7.8cm} | p{6cm} |}

    \hline

    %%% 此处标识以上为表格脚,即在每页的表格底部重复显示的内容(一条横线)
    \endfoot
    
    %%% 此处标识该行颜色为指定定义的颜色,tabhdcolor 在 ./introduction.tex 中定义,颜色为hsb(0, 0, 0.82353),即82度灰
    \rowcolor{tabhdcolor}

        %%% 第1行的表格信息
        \begin{center}
            时间
        \end{center} &
        \begin{center}
            周边
        \end{center} &
        \begin{center}
            备注
        \end{center} \\

    %%% 此处完整分割一行
    \hline

    %%% 此处标识以上为表格头,即在每页的表格头部重复显示的内容(一堆header)
    \endhead

        %%% 第2行的表格信息
        \begin{center}
            2024年10月
        \end{center} &
        \begin{center}
            nfc卡
        \end{center} &
        \begin{center}
            
        \end{center} \\
        
    \hline

        \begin{center}
           2024年11月
        \end{center} &
        \begin{center}
            金矿
        \end{center} &
        \begin{center}
            
        \end{center} \\
        
    \hline
    
        \begin{center}
            2024年11月 
        \end{center} \cellcolor{gndcolor} &
        \begin{center}
            能量机关周边
        \end{center} \cellcolor{gndcolor} &
        \begin{center}
            
        \end{center} \cellcolor{gndcolor} \\

    %%% 此处分割一行的第2至3个单元格
    \hline
    %\cline{2-3}
    
        \begin{center}
            2024年12月
        \end{center} &
        \begin{center}
            装甲板周边
        \end{center} &
        \begin{center}
            
        \end{center} \\
        
    \hline
    
        \begin{center}
            2025年2月
        \end{center} &
        \begin{center}
            相框
        \end{center} &
        \begin{center}
            
        \end{center} \\

    %%% 此处分割一行的第1至1个单元格,即只有第一个单元格下面有横线
    %\cline{1-1}
    \hline
    
        \begin{center}
            2025年3月
        \end{center} &
        \begin{center}
            悍标挂件
        \end{center} &
        \begin{center}
            
        \end{center} \\

    \hline
    
\end{longtable}
%%% 表格演示-END
%%% 其中,开头的 { X | X | X | X |} 中,竖线标识分割线,字母X为自动对齐,其余可替换的lrc分别为左对齐、右对齐、居中对齐
%%% 该表格的一大特性是跨页后仍保留表头内容,即 \endhead 前的内容