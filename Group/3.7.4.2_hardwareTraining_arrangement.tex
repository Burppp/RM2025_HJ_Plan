%%% 本文档作为功能演示

%%% 表格演示-BEGIN
\begin{longtable}{ p{1.5cm} | p{3cm} | p{6cm} | p{4.3cm} |}

    \hline

    %%% 此处标识以上为表格脚,即在每页的表格底部重复显示的内容(一条横线)
    \endfoot
    
    %%% 此处标识该行颜色为指定定义的颜色,tabhdcolor 在 ./introduction.tex 中定义,颜色为hsb(0, 0, 0.82353),即82度灰
    \rowcolor{tabhdcolor}

        \begin{center}
            培训内容
        \end{center} &
        \begin{center}
            实践
        \end{center} &
        \begin{center}
            验收任务
        \end{center} &
        \begin{center}
            负责人
        \end{center}\\

    %%% 此处完整分割一行
    \hline

    %%% 此处标识以上为表格头,即在每页的表格头部重复显示的内容(一堆header)
    \endhead

        %%% 第2行的表格信息
        \begin{center}
            %\multirow{3}{*}{机械组}
            无源元器件认知
        \end{center} &
        \begin{center}
            10月8日-10月14日(考核阶段)
        \end{center} &
        \begin{center}
            完成基础知识考核
        \end{center} &
        \begin{center}
            冯志轩
        \end{center}\\
        
    \hline
        \begin{center}
            学习 PCB 设计软件,设计遥控器
        \end{center}&
        \begin{center}
            10月24日-11月17日(校内赛)
        \end{center}&
        \begin{center}
            自主设计焊接校内赛小车遥控器。实现杆量读取与Lora通讯
        \end{center}&
        \begin{center}
            冯志轩
        \end{center}\\
        
    \hline

        \begin{center}
            有源元器件及DCDC电路拓扑认知
        \end{center}&
        \begin{center}
            11月18日-12月15日
        \end{center}&
        \begin{center}
           进行DCDC拓扑仿真。调试双向Buck-Boost开发板
        \end{center}&
        \begin{center}
            冯志轩
        \end{center}\\
        
    \hline
    
        %%% 第4行的表格信息
        %%% 第4行开始,连续2行的第1列单元格被合并,自动列宽
        %\multirow{3}{*}{电控组} &
        \begin{center}
            参与日常研发
        \end{center} &
        \begin{center}
            12月16日-12月30日
        \end{center} &
        \begin{center}
            总结学习成果,参与日常调试维护
        \end{center} &
        \begin{center}
            冯志轩
        \end{center} \\

    \hline
    
\end{longtable}
%%% 表格演示-END
%%% 其中,开头的 { X | X | X | X |} 中,竖线标识分割线,字母X为自动对齐,其余可替换的lrc分别为左对齐、右对齐、居中对齐
%%% 该表格的一大特性是跨页后仍保留表头内容,即 \endhead 前的内容