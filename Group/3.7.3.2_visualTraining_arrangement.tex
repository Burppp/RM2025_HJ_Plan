%%% 本文档作为功能演示

%%% 表格演示-BEGIN
\begin{longtable}{ p{6cm} | p{3cm} | p{1.5cm} | p{4.3cm} |}

    \hline

    %%% 此处标识以上为表格脚,即在每页的表格底部重复显示的内容(一条横线)
    \endfoot
    
    %%% 此处标识该行颜色为指定定义的颜色,tabhdcolor 在 ./introduction.tex 中定义,颜色为hsb(0, 0, 0.82353),即82度灰
    \rowcolor{tabhdcolor}

        \begin{center}
            培训内容
        \end{center} &
        \begin{center}
            实践
        \end{center} &
        \begin{center}
            培训形式
        \end{center} &
        \begin{center}
            负责人
        \end{center}\\

    %%% 此处完整分割一行
    \hline

    %%% 此处标识以上为表格头,即在每页的表格头部重复显示的内容(一堆header)
    \endhead

        %%% 第2行的表格信息
        \begin{center}
            %\multirow{3}{*}{机械组}
            Linux 系统的基本命令
            \newline Linux 文件目录结构
        \end{center} &
        \begin{center}
            2024.11.30-2024.12.1
        \end{center} &
        \begin{center}
            授课
        \end{center} &
        \begin{center}
            徐天翼
        \end{center}\\
        
    \hline
        \begin{center}
            完成 OpenCV 的基础学习并且完成期间布置 OpenCV 作业
        \end{center}&
        \begin{center}
            2024.12.1-2024.12.15
        \end{center}&
        \begin{center}
            自主学习
        \end{center}&
        \begin{center}
            徐天翼
        \end{center}\\
        
    \hline

        \begin{center}
            OpenCV 基础知识讲解(图片二值化、通道分离、轮廓查找)
        \end{center}&
        \begin{center}
            2024.12.15
        \end{center}&
        \begin{center}
            授课
        \end{center}&
        \begin{center}
            徐天翼
        \end{center}\\
        
    \hline
    
        %%% 第4行的表格信息
        %%% 第4行开始,连续2行的第1列单元格被合并,自动列宽
        %\multirow{3}{*}{电控组} &
        \begin{center}
            使用C++和 OpenCV 实现银行卡识别项目
        \end{center} &
        \begin{center}
            2024.12.15-2024.12.20
        \end{center} &
        \begin{center}
            考核
        \end{center} &
        \begin{center}
            徐天翼
        \end{center} \\

    \hline
    %%% 此处分割一行的第2至3个单元格
    %\cline{2-3}
    
        %%% 第5行的表格信息
        %%% 第5行由于第1列在上一行已被合并,所以建议不要写入信息,虽然写入信息也会显示出
        \begin{center}
            装甲板识别的基本思路、方法
        \end{center} &
        \begin{center}
            2024.12.20
        \end{center} &
        \begin{center}
            授课
        \end{center} &
        \begin{center}
            徐天翼
        \end{center} \\

    \hline

       \begin{center}
           相机的软硬件基础知识学习
       \end{center} &
       \begin{center}
           2024.12.26-2025.1.1
       \end{center} &
       \begin{center}
           自主学习
       \end{center} &
       \begin{center}
           徐天翼
       \end{center} \\
        
    \hline
    
        %%% 第6行的表格信息
        %\multirow{3}{*}{视觉组} &
        \begin{center}
            使用机器学习、OpenCV、C++的知识,实现装甲板的识别
        \end{center}&
        \begin{center}
            2025.1-2025.2
        \end{center}&
        \begin{center}
            实际项目
        \end{center}&
        \begin{center}
            徐天翼
        \end{center}\\

    \hline

        \begin{center}
            讲解装甲板识别的注意事项和主要的思路方法
        \end{center}&
        \begin{center}
            2025.2.15
        \end{center}&
        \begin{center}
            授课
        \end{center}&
        \begin{center}
            陈鹏天、佘耀翔
        \end{center}\\

    \hline

        \begin{center}
            自主研读车载视觉代码
        \end{center}&
        \begin{center}
            2025.2.15-2025.2.25
        \end{center}&
        \begin{center}
            自主学习
        \end{center}&
        \begin{center}
            徐天翼
        \end{center}\\

    \hline

        \begin{center}
            世界坐标系、相机坐标系、图像物理坐标系、图像像素坐标系之间的关系以及转换思路
        \end{center}&
        \begin{center}
            2025.2.25
        \end{center}&
        \begin{center}
            授课
        \end{center}&
        \begin{center}
            徐天翼
        \end{center}\\

    \hline

        \begin{center}
            ROS入门学习
        \end{center}&
        \begin{center}
            2025.3.1
        \end{center}&
        \begin{center}
            授课
        \end{center}&
        \begin{center}
            徐天翼
        \end{center}\\

    \hline

        \begin{center}
            雷达的基本思路和实现方法
        \end{center}&
        \begin{center}
            2025.3.10
        \end{center}&
        \begin{center}
            授课
        \end{center}&
        \begin{center}
            徐天翼
        \end{center}\\

    \hline
    
\end{longtable}
%%% 表格演示-END
%%% 其中,开头的 { X | X | X | X |} 中,竖线标识分割线,字母X为自动对齐,其余可替换的lrc分别为左对齐、右对齐、居中对齐
%%% 该表格的一大特性是跨页后仍保留表头内容,即 \endhead 前的内容