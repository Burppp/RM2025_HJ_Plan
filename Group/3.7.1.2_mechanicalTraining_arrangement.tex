%%% 本文档作为功能演示

%%% 表格演示-BEGIN
\begin{longtable}{ p{1.5cm} | p{3cm} | p{6cm} | p{4.3cm} |}

    \hline

    %%% 此处标识以上为表格脚,即在每页的表格底部重复显示的内容(一条横线)
    \endfoot
    
    %%% 此处标识该行颜色为指定定义的颜色,tabhdcolor 在 ./introduction.tex 中定义,颜色为hsb(0, 0, 0.82353),即82度灰
    \rowcolor{tabhdcolor}

        \begin{center}
            培训内容
        \end{center} &
        \begin{center}
            实践
        \end{center} &
        \begin{center}
            验收任务
        \end{center} &
        \begin{center}
            负责人
        \end{center}\\

    %%% 此处完整分割一行
    \hline

    %%% 此处标识以上为表格头,即在每页的表格头部重复显示的内容(一堆header)
    \endhead

        %%% 第2行的表格信息
        \begin{center}
            %\multirow{3}{*}{机械组}
            机械原理知识讲解
        \end{center} &
        \begin{center}
            2024年10月4日——10月9日
        \end{center} &
        \begin{center}
            完成调研报告并提交至负责人
        \end{center} &
        \begin{center}
            龙俊洁
        \end{center}\\
        
    \hline
        \begin{center}
            sw使用方法讲解及简单机械原理和材料力学的学习
        \end{center}&
        \begin{center}
            2024年10月9日——10月13日
        \end{center}&
        \begin{center}
            完成sw建模及规则、机械原理、材料力学考核(机试\&笔试)
        \end{center}&
        \begin{center}
            龙俊洁
        \end{center}\\
        
    \hline

        \begin{center}
            工程制图学习
        \end{center}&
        \begin{center}
            2024年10月22日——10月25日
        \end{center}&
        \begin{center}
            基本工程图的尺寸标注及识图
        \end{center}&
        \begin{center}
            龙俊洁
        \end{center}\\
        
    \hline
    
        %%% 第4行的表格信息
        %%% 第4行开始,连续2行的第1列单元格被合并,自动列宽
        %\multirow{3}{*}{电控组} &
        \begin{center}
            校内赛
        \end{center} &
        \begin{center}
            2024年10月27日——12月15日
        \end{center} &
        \begin{center}
            自主设计机器人,完成比赛内容
        \end{center} &
        \begin{center}
            洪俊钦
        \end{center} \\

    \hline
    %%% 此处分割一行的第2至3个单元格
    %\cline{2-3}
    
        %%% 第5行的表格信息
        %%% 第5行由于第1列在上一行已被合并,所以建议不要写入信息,虽然写入信息也会显示出
        \begin{center}
            3D打印学习
        \end{center} &
        \begin{center}
            2024年10月27日——11月1日
        \end{center} &
        \begin{center}
            自主设计指尖陀螺并打印
        \end{center} &
        \begin{center}
            黄德曦、黄智衡、严浩
        \end{center} \\

    \hline

       \begin{center}
           校内赛场地搭建
       \end{center} &
       \begin{center}
           2024年11月6日——11月9日
       \end{center} &
       \begin{center}
           搭建完整的校内赛场地
       \end{center} &
       \begin{center}
           龙俊洁
       \end{center} \\
        
    \hline
    
        %%% 第6行的表格信息
        %\multirow{3}{*}{视觉组} &
        \begin{center}
            校内赛第一阶段任务
        \end{center}&
        \begin{center}
            2024年11月11日——11月17日
        \end{center}&
        \begin{center}
            机械结构出图并撰写技术文档
        \end{center}&
        \begin{center}
            洪俊钦、龙俊洁、严浩、黄德曦、黄智衡、曾粤炜、殷晟琦
        \end{center}\\

    %%% 此处分割一行的第1至1个单元格,即只有第一个单元格下面有横线
    %\cline{1-1}
    \hline
    
        %%% 第7行的表格信息
        %content 7 1 &
        %%% 第7行由于第2~4列在上一行已被合并,所以建议不要写入信息,虽然写入信息也会显示出,而且建议像下面一样合并表格防止竖线分割
        %\multicolumn{3}{| c }{} \\

       \begin{center}
           校内赛第二阶段任务
       \end{center} &
       \begin{center}
           2024年11月19日——11月26日
       \end{center} &
       \begin{center}
           第一版小车结构验证并进行竞速赛
       \end{center} &
       \begin{center}
           洪俊钦、龙俊洁、严浩
       \end{center} \\

    \hline

    \begin{center}
           校内赛第三阶段任务
       \end{center} &
       \begin{center}
           2024年11月27日——12月3日
       \end{center} &
       \begin{center}
           继续优化机械结构
       \end{center} &
       \begin{center}
           洪俊钦、龙俊洁
       \end{center} \\

    \hline

    \begin{center}
           校内赛第四阶段任务
       \end{center} &
       \begin{center}
           2024年12月4日——12月15日
       \end{center} &
       \begin{center}
           特色功能展示
       \end{center} &
       \begin{center}
           洪俊钦、龙俊洁
       \end{center} \\

    \hline
    
\end{longtable}
%%% 表格演示-END
%%% 其中,开头的 { X | X | X | X |} 中,竖线标识分割线,字母X为自动对齐,其余可替换的lrc分别为左对齐、右对齐、居中对齐
%%% 该表格的一大特性是跨页后仍保留表头内容,即 \endhead 前的内容