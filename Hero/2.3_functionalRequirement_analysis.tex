%%% 本文档作为功能演示

%%% 表格演示-BEGIN
\begin{longtable}{ p{2cm} | p{7.8cm} | p{6cm} |}

    \hline

    %%% 此处标识以上为表格脚,即在每页的表格底部重复显示的内容(一条横线)
    \endfoot
    
    %%% 此处标识该行颜色为指定定义的颜色,tabhdcolor 在 ./introduction.tex 中定义,颜色为hsb(0, 0, 0.82353),即82度灰
    \rowcolor{tabhdcolor}

        %%% 第1行的表格信息
        \begin{center}
            功能
        \end{center} &
        \begin{center}
            需求分析
        \end{center} &
        \begin{center}
            设计思路
        \end{center} \\

    %%% 此处完整分割一行
    \hline

    %%% 此处标识以上为表格头,即在每页的表格头部重复显示的内容(一堆header)
    \endhead

        %%% 第2行的表格信息
        \begin{center}
            稳定的发射(弹速)
        \end{center} &
        \begin{center}
            出弹时需要保证弹丸处于枪管中心轴,需要较高精度的定心。最主要的是保持电机一致的转速,当弹丸通过加速时,多个电机转速变化一致
        \end{center} &
        \begin{center}
            使用双级发射,利用两对摩擦轮实现双级加速,延长加速道路,使弹速尽量稳定。并且在上下部分增加了两个摩擦轮进行单发限位,并且进行上下定心
        \end{center} \\
        
    \hline

        \begin{center}
            爬坡
        \end{center} &
        \begin{center}
            需要较高的功率利用效果,保证爬坡时功率充足,底盘和地面之间需要一定的高度保证通过角可以通过43度坡
        \end{center} &
        \begin{center}
            采用功率利用率较高的舵轮底盘,保证爬坡时功率充足。舵轮底盘的轮组较高,可以完美的实现增大通过角,提高爬坡能力
        \end{center} \\
        
    \hline
    
        \begin{center}
            飞坡 
        \end{center} \cellcolor{gndcolor} &
        \begin{center}
            需要稳定的出坡姿态,保持飞坡的仪态正常,尽可能的让轮子先着地,并且不损坏机械主体结构
        \end{center} \cellcolor{gndcolor} &
        \begin{center}
            调配前后重心,使整体重心尽量趋于中心,并且加强底盘的悬挂功能,增加落地时的缓冲,防止飞坡落地对机械结构造成损坏
        \end{center} \cellcolor{gndcolor} \\

    %%% 此处分割一行的第2至3个单元格
    \hline
    %\cline{2-3}
    
        \begin{center}
            流畅供弹
        \end{center} &
        \begin{center}
            需要保证下链路与上链路丝滑不卡弹,弹仓的部分不滞弹不空弹,进入仓管的部分不卡弹
        \end{center} &
        \begin{center}
            采用中心供弹和侧供弹,并且在整条弹链上加上小轴承润滑,保证弹丸在弹链中的顺滑运行。并且在弹仓部分使用角度较高的垫底放在弹仓四周,引导弹丸向下运动,减少滞弹与空弹现象
        \end{center} \\
        
    \hline
    
        \begin{center}
            精准打击
        \end{center} &
        \begin{center}
            尽可能小的散布,出弹稳定的弹速,并且需要稳定的弹道
        \end{center} &
        \begin{center}
            采用电机摩擦轮左右定心,上下摩擦轮单发限位定心,尽可能的保证弹丸处于枪管中心轴,再利用到相对稳定的电机速度来保证出弹速度稳定,以此来实现稳定的弹道
            \newline 并且可以通过发射弹丸时锁死底盘轮子的方法,来减小后坐力带来的底盘运动,尽可能减小弹丸的上下散布
        \end{center} \\

    %%% 此处分割一行的第1至1个单元格,即只有第一个单元格下面有横线
    %\cline{1-1}
    \hline
    
        \begin{center}
            精确的瞄准
        \end{center} &
        \begin{center}
            需要在击打点的距离与高低落差不确定的情况下缩短瞄准与校射流程
        \end{center} &
        \begin{center}
            通过传感器融合、弹道预测、快速调节与实时反馈相结合,实现发弹机器人在复杂条件下的高效瞄准和精准校射
        \end{center} \\
        
    \hline
    
        \begin{center}
            校正弹道
        \end{center} &
        \begin{center}
            弹道会因为各种原因偏移原来的预定轨迹,而人工调整太慢且不精确。所以需要新的算法来来识别弹道的轨迹和落点来自动修正枪口的射击方向
        \end{center} &
        \begin{center}
            首先是识别出弹丸轨迹,确定出落点后,根据落点与目标的偏差来调整枪管移动方向和抬高角度
        \end{center} \\
        
    \hline
    
        \begin{center}
            吊射
        \end{center} &
        \begin{center}
            配合雷达或者激光测距,获取敌方位置信息后实现远距离击打
        \end{center} &
        \begin{center}
            获得位置信息后,设计算法来调整枪管角度来实现吊射
        \end{center} \\
        
    \hline
    
        \begin{center}
            优化操作体验
        \end{center} &
        \begin{center}
            为了最大化英雄机器人性能,确保操作手通过键鼠控制时机器人运动流畅敏捷,操作手界面(UI)将实时显示功能状态并辅助瞄准
            \newline 倍镜功能在吊射作业中提供精准瞄准支持
            \newline 为提升响应速度,我们将优化控制算法,超越单一PID反馈控制,确保机器人在复杂任务中的高效性和稳定性
        \end{center} &
        \begin{center}
            选取合适的滤波器对鼠标返回值进行滤波,确保操作流畅
            \newline 界面UI将实时显示电容当前电量百分比、摩擦轮和陀螺仪的开关状态,以及准星、通过框、辅助瞄准线和自瞄有效范围,帮助操作手精准掌控机器人状态
            \newline 在图传系统前加装三倍镜,便于在吊射敌方基地时进行远距离精确瞄准
            \newline 同时,采用前馈控制技术提升云台响应速度和精度,确保机器人操作的高效性和精准度
        \end{center} \\
        
    \hline
    
\end{longtable}
%%% 表格演示-END
%%% 其中,开头的 { X | X | X | X |} 中,竖线标识分割线,字母X为自动对齐,其余可替换的lrc分别为左对齐、右对齐、居中对齐
%%% 该表格的一大特性是跨页后仍保留表头内容,即 \endhead 前的内容