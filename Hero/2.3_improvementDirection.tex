%%% 本文档作为功能演示

%%% 表格演示-BEGIN
\begin{longtable}{ p{1.5cm} | p{3cm} | p{6cm} | p{4.3cm} |}

    \hline

    %%% 此处标识以上为表格脚,即在每页的表格底部重复显示的内容(一条横线)
    \endfoot
    
    %%% 此处标识该行颜色为指定定义的颜色,tabhdcolor 在 ./introduction.tex 中定义,颜色为hsb(0, 0, 0.82353),即82度灰
    \rowcolor{tabhdcolor}

        \begin{center}
            组别
        \end{center} &
        \begin{center}
            改进对象
        \end{center} &
        \begin{center}
            改进内容
        \end{center} &
        \begin{center}
            验收指标
        \end{center}\\

    %%% 此处完整分割一行
    \hline

    %%% 此处标识以上为表格头,即在每页的表格头部重复显示的内容(一堆header)
    \endhead

        %%% 第2行的表格信息
        \begin{center}
            %\multirow{3}{*}{机械组}
            机械组
        \end{center} &
        \begin{center}
            轮组
        \end{center} &
        \begin{center}
            轮组由麦轮轮组改成舵轮轮组,舵轮轮组相较于麦轮轮组具有更低的功耗,以及更加优秀的爬坡能力,以适应新赛季的多种地形
        \end{center} &
        \begin{center}
            缓冲能够正常使用,转弯能够实现全向移动,运动时不会干涉,可以通过新赛季43度坡
        \end{center}\\
        
    \hline
        \begin{center}
            机械组
        \end{center}&
        \begin{center}
            供弹方式
        \end{center}&
        \begin{center}
            今年将供弹方式从鹅颈供弹换成侧供弹结构,侧供弹相对于鹅颈可以缩小体积,并且使弹丸进入摩擦轮更加稳定,来实现更好的定心和稳定弹速的作用
        \end{center}&
        \begin{center}
            弹链做到不卡弹,且较为丝滑
        \end{center}\\
        
    \hline

        \begin{center}
            机械组
        \end{center}&
        \begin{center}
            发射机构
        \end{center}&
        \begin{center}
            今年将发射机构的三摩擦轮结构换成双级摩擦轮结构,需要改进发射定心的位置与摩擦轮的大小硬度等
        \end{center}&
        \begin{center}
            命中率相较于三摩擦轮有所提高
        \end{center}\\
        
    \hline
    
        %%% 第4行的表格信息
        %%% 第4行开始,连续2行的第1列单元格被合并,自动列宽
        %\multirow{3}{*}{电控组} &
        \begin{center}
            机械组
        \end{center} &
        \begin{center}
            pitch
        \end{center} &
        \begin{center}
            将原本的丝杆中心放置位置向外移动,形成丝杆偏置,改进了旧丝杆位置会与c板放置位置发生干涉的不利,并且增大了俯仰角度
        \end{center} &
        \begin{center}
            能够正常的上下摆动角度,并且实现在不产生干扰的情况下有45度角以上的机械仰角
        \end{center} \\

    \hline
    %%% 此处分割一行的第2至3个单元格
    %\cline{2-3}
    
        %%% 第5行的表格信息
        %%% 第5行由于第1列在上一行已被合并,所以建议不要写入信息,虽然写入信息也会显示出
        \begin{center}
            机械组
        \end{center} &
        \begin{center}
            图传
        \end{center} &
        \begin{center}
            重新架构了图传的运动结构和放大倍镜的放置位置及其机械限位。改用了简单的连杆传动,并且在图传左右方都加了连杆支撑,增加了图传运动时的稳定性以及俯视角度,尽量满足吊射时的观察需要
        \end{center} &
        \begin{center}
            能够满足増大俯角的情况下有稳定且清晰的图像,能满足在最大仰角的情况下能够观察到前方目标
        \end{center} \\

    \hline

       \begin{center}
           电控组
       \end{center} &
       \begin{center}
           双级摩擦轮
       \end{center} &
       \begin{center}
           今年将发射机构的三摩擦轮结构换成双级摩擦轮结构,需要改进发射逻辑,对于不同摩擦轮控制算法也要有所不同,来实现定心和稳定弹速的作用
       \end{center} &
       \begin{center}
           双级摩擦轮结构在发射过程中能够实现精确定心,确保弹速稳定,摩擦轮在不同工作环境下运行稳定,弹道偏差不超过2°,弹速波动控制在±2\%以内
       \end{center} \\
        
    \hline
    
        %%% 第6行的表格信息
        %\multirow{3}{*}{视觉组} &
        \begin{center}
            电控组
        \end{center}&
        \begin{center}
            代码新框架
        \end{center}&
        \begin{center}
            在25赛季初,我们对24赛季的英雄机器人代码框架进行了迭代升级,完成了功能封装与改进,优化了内存管理,提高了代码的可读性
        \end{center}&
        \begin{center}
            功能模块封装完善,内存管理优化无泄漏,代码可读性提升,关键功能性能提高,兼容性测试无问题,运行稳定性通过实际比赛验证
        \end{center}\\

    %%% 此处分割一行的第1至1个单元格,即只有第一个单元格下面有横线
    %\cline{1-1}
    \hline
    
        %%% 第7行的表格信息
        %content 7 1 &
        %%% 第7行由于第2~4列在上一行已被合并,所以建议不要写入信息,虽然写入信息也会显示出,而且建议像下面一样合并表格防止竖线分割
        %\multicolumn{3}{| c }{} \\

       \begin{center}
           电控组
       \end{center} &
       \begin{center}
           PID+前馈控制
       \end{center} &
       \begin{center}
           上赛季英雄机器人控制未引入前馈控制。前馈控制能够在系统动态响应中提前调整控制信号,有效减少响应误差,提升响应速度和系统稳定性
       \end{center} &
       \begin{center}
           前馈控制成功引入,系统响应时间缩短≥10\%,响应误差降低≥15\%,多场景测试中运行稳定,无明显振荡或延迟
       \end{center} \\

    \hline

       \begin{center}
           电控组
       \end{center} &
       \begin{center}
           布线
       \end{center} &
       \begin{center}
           自己制作不同线长的线材;所有线材需使用缠绕管保护;滑环接线板需牢固固定;滑环接线连接处需重点保护,避免出现虚焊问题
       \end{center} &
       \begin{center}
           暴力测试下整车信号完好;布线清晰,排查方便
       \end{center} \\
        
    \hline

        %\multirow{3}{*}{硬件组} &
       \begin{center}
           视觉组
       \end{center} &
       \begin{center}
           弹道解算
       \end{center} &
       \begin{center}
           用更高阶的弹道解算模型,优化弹道解算,以使枪管抬高的更精确,并使自瞄能适应各种距离
       \end{center} &
       \begin{center}
           小车瞄准车后枪管的抬高角度更准
       \end{center} \\

    \hline

       \begin{center}
           视觉组
       \end{center} &
       \begin{center}
           反陀螺
       \end{center} &
       \begin{center}
           优化算法,填补缺陷,提升命中率
       \end{center} &
       \begin{center}
           击打旋转装甲板的命中率比之前更高
       \end{center} \\

    \hline

       \begin{center}
           视觉组
       \end{center} &
       \begin{center}
           弹道校正
       \end{center} &
       \begin{center}
           新的模块,用来校正弹道,根据不同的距离自动添加一个枪管补偿来解决距离带来的误差
       \end{center} &
       \begin{center}
           距离对命中率的影响变小,能适应各种距离
       \end{center} \\
    
\end{longtable}
%%% 表格演示-END
%%% 其中,开头的 { X | X | X | X |} 中,竖线标识分割线,字母X为自动对齐,其余可替换的lrc分别为左对齐、右对齐、居中对齐
%%% 该表格的一大特性是跨页后仍保留表头内容,即 \endhead 前的内容