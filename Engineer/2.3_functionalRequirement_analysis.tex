%%% 本文档作为功能演示

%%% 表格演示-BEGIN
\begin{longtable}{ p{2cm} | p{7.8cm} | p{6cm} |}

    \hline

    %%% 此处标识以上为表格脚,即在每页的表格底部重复显示的内容(一条横线)
    \endfoot
    
    %%% 此处标识该行颜色为指定定义的颜色,tabhdcolor 在 ./introduction.tex 中定义,颜色为hsb(0, 0, 0.82353),即82度灰
    \rowcolor{tabhdcolor}

        %%% 第1行的表格信息
        \begin{center}
            功能
        \end{center} &
        \begin{center}
            需求分析
        \end{center} &
        \begin{center}
            设计思路
        \end{center} \\

    %%% 此处完整分割一行
    \hline

    %%% 此处标识以上为表格头,即在每页的表格头部重复显示的内容(一堆header)
    \endhead

        %%% 第2行的表格信息
        \begin{center}
            重心分配合理的底盘
        \end{center} &
        \begin{center}
            重心分配合理的底盘	由于我们采取的月球车存在显著的缺点就是地盘高,需要在保持车上重量的同时,在地盘下进行配重来降低重心避免翻车的可能	底盘采用田字形框架,采用2020铝方管作为整体框架,尽可能的将电池、气泵等配件往下放以降低重心
        \end{center} &
        \begin{center}
            底盘采用田字形框架,采用2020铝方管作为整体框架,尽可能的将电池、气泵等配件往下放以降低重心
        \end{center} \\
        
    \hline

        \begin{center}
            高灵活性,能跨台阶的月球车底盘
        \end{center} &
        \begin{center}
            减少到达大资源岛的时间,加  强争夺大资源岛金矿石的能力
        \end{center} &
        \begin{center}
            轮组采用麦克纳姆轮,框架仿照月球车进行设计
        \end{center} \\
        
    \hline
    
        \begin{center}
            剪式升降台 
        \end{center} \cellcolor{gndcolor} &
        \begin{center}
            抬升取银矿机械臂
        \end{center} \cellcolor{gndcolor} &
        \begin{center}
            在原来老工程基础上升高高度
        \end{center} \cellcolor{gndcolor} \\

    %%% 此处分割一行的第2至3个单元格
    \hline
    %\cline{2-3}
    
        \begin{center}
            一键三矿
        \end{center} &
        \begin{center}
            争取做到一次性取三个银矿
        \end{center} &
        \begin{center}
            通过两个电机固定旋转轴,同时在伸缩主板上预留取银矿的机械臂,控制距离,取银矿则采用吸盘完全吸住上升
        \end{center} \\
        
    \hline
    
        \begin{center}
            自定义控制器
        \end{center} &
        \begin{center}
            争取将大部分功能集合在自定义控制器中且做出舒适且精确的自定义控制器
        \end{center} &
        \begin{center}
            机械制作大臂缩小模型的小机械臂,带有六个角度传感器,奖角度直接传入大臂。
        \end{center} \\
        
    \hline
    
\end{longtable}
%%% 表格演示-END
%%% 其中,开头的 { X | X | X | X |} 中,竖线标识分割线,字母X为自动对齐,其余可替换的lrc分别为左对齐、右对齐、居中对齐
%%% 该表格的一大特性是跨页后仍保留表头内容,即 \endhead 前的内容