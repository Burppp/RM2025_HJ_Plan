%%% 本文档作为功能演示

%%% 表格演示-BEGIN
\begin{longtable}{ p{1.5cm} | p{3cm} | p{6cm} | p{4.3cm} |}

    \hline

    %%% 此处标识以上为表格脚,即在每页的表格底部重复显示的内容(一条横线)
    \endfoot
    
    %%% 此处标识该行颜色为指定定义的颜色,tabhdcolor 在 ./introduction.tex 中定义,颜色为hsb(0, 0, 0.82353),即82度灰
    \rowcolor{tabhdcolor}

        \begin{center}
            组别
        \end{center} &
        \begin{center}
            改进对象
        \end{center} &
        \begin{center}
            改进内容
        \end{center} &
        \begin{center}
            验收指标
        \end{center}\\

    %%% 此处完整分割一行
    \hline

    %%% 此处标识以上为表格头,即在每页的表格头部重复显示的内容(一堆header)
    \endhead

        %%% 第2行的表格信息
        \begin{center}
            %\multirow{3}{*}{机械组}
            机械组
        \end{center} &
        \begin{center}
            轮组
        \end{center} &
        \begin{center}
            改舵轮为麦克纳姆轮
        \end{center} &
        \begin{center}
            提高底盘稳定性,完成底盘的稳定转向
        \end{center}\\
        
    \hline
        \begin{center}
            机械组
        \end{center}&
        \begin{center}
            重心问题
        \end{center}&
        \begin{center}
            减少机械臂和底盘高带来的翻车可能
        \end{center}&
        \begin{center}
            合理选择避震器,以及对车的配件进行合理放置,降低重心
        \end{center}\\
        
    \hline

        \begin{center}
            机械组
        \end{center}&
        \begin{center}
            抬升
        \end{center}&
        \begin{center}
            比原来抬升再高一点使其可以取出银矿
        \end{center}&
        \begin{center}
            高度可以使得取完银矿后翻转180度也不会超过1200*1200
        \end{center}\\
        
    \hline
    
        %%% 第4行的表格信息
        %%% 第4行开始,连续2行的第1列单元格被合并,自动列宽
        %\multirow{3}{*}{电控组} &
        \begin{center}
            机械组
        \end{center} &
        \begin{center}
            抬升架上取矿个数
        \end{center} &
        \begin{center}
            一键三矿
        \end{center} &
        \begin{center}
            使其机械臂可以完成一次性取完三个银矿,且将该机构作为存矿机构
        \end{center} \\

    \hline
    %%% 此处分割一行的第2至3个单元格
    %\cline{2-3}
    
        %%% 第5行的表格信息
        %%% 第5行由于第1列在上一行已被合并,所以建议不要写入信息,虽然写入信息也会显示出
        \begin{center}
            电控组
        \end{center} &
        \begin{center}
            底层通讯
        \end{center} &
        \begin{center}
            主板数量多,板间通讯复杂,减少主板优化通讯
        \end{center} &
        \begin{center}
            保守使用电机与优化CAN线上的通讯
        \end{center} \\

    \hline

       \begin{center}
           电控组
       \end{center} &
       \begin{center}
           上位机
       \end{center} &
       \begin{center}
           ros能力较有限,基于TCP/IP的通信,不适合硬实时应用
       \end{center} &
       \begin{center}
           采用ros2进行开发。使用了DDS作为其通信中间件,提供了更好的数据传输性能和确定性
       \end{center} \\
        
    \hline
    
        %%% 第6行的表格信息
        %\multirow{3}{*}{视觉组} &
        \begin{center}
            电控组
        \end{center}&
        \begin{center}
            控制器
        \end{center}&
        \begin{center}
            解决旧控制器操作复杂,学习成本高,操作不易且较慢的问题
        \end{center}&
        \begin{center}
            确定使用大臂模型缩小机械臂进行控制
        \end{center}\\

    %%% 此处分割一行的第1至1个单元格,即只有第一个单元格下面有横线
    %\cline{1-1}
    \hline
    
        %%% 第7行的表格信息
        %content 7 1 &
        %%% 第7行由于第2~4列在上一行已被合并,所以建议不要写入信息,虽然写入信息也会显示出,而且建议像下面一样合并表格防止竖线分割
        %\multicolumn{3}{| c }{} \\

       \begin{center}
           视觉组
       \end{center} &
       \begin{center}
           自动取矿
       \end{center} &
       \begin{center}
           实现视觉自动兑矿
       \end{center} &
       \begin{center}
           视觉框取兑矿框的四个角,获取兑矿框中点并传入上位机实现路径规划
       \end{center} \\

    \hline
    
\end{longtable}
%%% 表格演示-END
%%% 其中,开头的 { X | X | X | X |} 中,竖线标识分割线,字母X为自动对齐,其余可替换的lrc分别为左对齐、右对齐、居中对齐
%%% 该表格的一大特性是跨页后仍保留表头内容,即 \endhead 前的内容