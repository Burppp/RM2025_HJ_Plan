%%% 本文档作为功能演示

%%% 表格演示-BEGIN
\begin{longtable}{ p{2cm} | p{3cm} | p{3cm} | p{4.8cm} | p{2cm} |}

    \hline

    %%% 此处标识以上为表格脚,即在每页的表格底部重复显示的内容(一条横线)
    \endfoot
    
    %%% 此处标识该行颜色为指定定义的颜色,tabhdcolor 在 ./introduction.tex 中定义,颜色为hsb(0, 0, 0.82353),即82度灰
    \rowcolor{tabhdcolor}

        \begin{center}
            项目
        \end{center}  &
        \begin{center}
            任务
        \end{center}  &
        \begin{center}
           人力评估
        \end{center} &
        \begin{center}
            人员技能要求
        \end{center}  &
        \begin{center}
            耗时评估
        \end{center}  \\ 
        
    \hline

    %%% 此处标识以上为表格头,即在每页的表格头部重复显示的内容(一堆header)
    \endhead

        %%% 第2行的表格信息
        %\multirow{3}{*}{机械组} &
        \begin{center}
            机械结构
        \end{center} &
        \begin{center}
            完成工程的机械结构,包括底盘、机械臂、存取银矿机构、自定义控制器
        \end{center} &
        \begin{center}
            3名机械组成员
        \end{center} &
        \begin{center}
            solidwork使用,机械零件选型
        \end{center} &
        \begin{center}
            4周
        \end{center}\\
        
    \hline
        \begin{center}
            自定义控制器
        \end{center} &
        %%% 第3行的表格信息
        \begin{center}
            完成一个完善且可以通过UART/CAN向主板发送6个角度的自定义控制器,在此基础上根据不同模式添加不同内容
        \end{center} &
        %%% 第3行的第2~3列单元格被合并,左对齐,且左右均有分割线
        %\multicolumn{2}{| l |}{content 3 2-3} &
        \begin{center}
            1名电控组成员
        \end{center} &
        \begin{center}
            can通讯,stm32基础,机械臂解算
        \end{center} &
        \begin{center}
            3周
        \end{center}\\

    \hline
    
        \begin{center}
            下位机通讯与框架
        \end{center} &
        \begin{center}
            完成自定义控制器数据接受、图传链路、上下位机通讯等,对底盘进行解算与对机械臂电机进行限位、控制等
        \end{center} &
        \begin{center}
            1名电控组成员
        \end{center} &
        \begin{center}
            FreeRTOS系统熟练度,掌握不同通信协议,pid算法控制,底盘解算、电机控制与参数调整等。尝试在下位机实现机械臂正逆解算
        \end{center} &
        \begin{center}
            4周
        \end{center} \\
        
    \hline
    
        %%% 第4行的表格信息
        %%% 第4行开始,连续2行的第1列单元格被合并,自动列宽
        %\multirow{3}{*}{电控组} &
        \begin{center}
            上位机
        \end{center} &
        \begin{center}
            采用ros2进行开发,将机械臂urdf与模型转化并进行摄像头位置、控制仿真等,对机械臂机械路径规划
        \end{center} &
        \begin{center}
            1名电控组成员
        \end{center} &
        \begin{center}
            熟悉ros控制和moveit框架,对urdf文件有一定的了解
        \end{center} &
        \begin{center}
            4周
        \end{center}\\

    \hline
    %%% 此处分割一行的第2至3个单元格
    %\cline{2-3}
    
        %%% 第5行的表格信息
        %%% 第5行由于第1列在上一行已被合并,所以建议不要写入信息,虽然写入信息也会显示出
        \begin{center}
            整车调试
        \end{center} &
        \begin{center}
            对第一版车进行调试,然后总结问题并做好调试日志与问题解决方案等,针对问题进行修改
        \end{center} &
        \begin{center}
            3名机械组成员,3名电控组成员
        \end{center} &
        \begin{center}
            调试方法与日志管理
        \end{center} &
        \begin{center}
            长期进行
        \end{center}\\

    \hline
    
\end{longtable}
%%% 表格演示-END
%%% 其中,开头的 { X | X | X | X |} 中,竖线标识分割线,字母X为自动对齐,其余可替换的lrc分别为左对齐、右对齐、居中对齐
%%% 该表格的一大特性是跨页后仍保留表头内容,即 \endhead 前的内容