%%% 本文档作为功能演示

%%% 表格演示-BEGIN
\begin{longtable}{ p{1.5cm} | p{3cm} | p{6cm} | p{4.3cm} |}

    \hline

    %%% 此处标识以上为表格脚,即在每页的表格底部重复显示的内容(一条横线)
    \endfoot
    
    %%% 此处标识该行颜色为指定定义的颜色,tabhdcolor 在 ./introduction.tex 中定义,颜色为hsb(0, 0, 0.82353),即82度灰
    \rowcolor{tabhdcolor}

        \begin{center}
            组别
        \end{center} &
        \begin{center}
            改进对象
        \end{center} &
        \begin{center}
            改进内容
        \end{center} &
        \begin{center}
            验收指标
        \end{center}\\

    %%% 此处完整分割一行
    \hline

    %%% 此处标识以上为表格头,即在每页的表格头部重复显示的内容(一堆header)
    \endhead

        %%% 第2行的表格信息
        \begin{center}
            %\multirow{3}{*}{机械组}
            视觉组
        \end{center} &
        \begin{center}
            神经网络模型
        \end{center} &
        \begin{center}
            设置多层识别模型
        \end{center} &
        \begin{center}
            
        \end{center}\\
        
    \hline
        \begin{center}
            视觉组
        \end{center}&
        \begin{center}
            点云图搭建
        \end{center}&
        \begin{center}
            结合激光雷达与点云匹配算法实现自建点云图快速对齐
        \end{center}&
        \begin{center}
            
        \end{center}\\
        
    \hline
    
\end{longtable}
%%% 表格演示-END
%%% 其中,开头的 { X | X | X | X |} 中,竖线标识分割线,字母X为自动对齐,其余可替换的lrc分别为左对齐、右对齐、居中对齐
%%% 该表格的一大特性是跨页后仍保留表头内容,即 \endhead 前的内容