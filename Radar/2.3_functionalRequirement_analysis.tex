%%% 本文档作为功能演示

%%% 表格演示-BEGIN
\begin{longtable}{ p{2cm} | p{7.8cm} | p{6cm} |}

    \hline

    %%% 此处标识以上为表格脚,即在每页的表格底部重复显示的内容(一条横线)
    \endfoot
    
    %%% 此处标识该行颜色为指定定义的颜色,tabhdcolor 在 ./introduction.tex 中定义,颜色为hsb(0, 0, 0.82353),即82度灰
    \rowcolor{tabhdcolor}

        %%% 第1行的表格信息
        \begin{center}
            功能
        \end{center} &
        \begin{center}
            需求分析
        \end{center} &
        \begin{center}
            设计思路
        \end{center} \\

    %%% 此处完整分割一行
    \hline

    %%% 此处标识以上为表格头,即在每页的表格头部重复显示的内容(一堆header)
    \endhead

        %%% 第2行的表格信息
        \begin{center}
            精准定位以及赛场快速部署
        \end{center} &
        \begin{center}
            尽可能精确的识别地方位置且保证赛场快速部署
        \end{center} &
        \begin{center}
            自建点云图保证精度,结合激光雷达以及点云匹配算法实现快速部署
        \end{center} \\
        
    \hline

        \begin{center}
            多层识别
        \end{center} &
        \begin{center}
            在敌方机器人部分遮挡的情况下任然能识别出其位置
        \end{center} &
        \begin{center}
            搭建多层视觉神经网络来实现机器人不同暴露度下的识别
        \end{center} \\
        
    \hline
    
        \begin{center}
            协助哨兵进行重定位
        \end{center} \cellcolor{gndcolor} &
        \begin{center}
            规则允许雷达接收哨兵的信息,可利用哨兵的重定位功能精确定位
        \end{center} \cellcolor{gndcolor} &
        \begin{center}
            建立专门面对哨兵的串口,通过裁判系统接收哨兵信息
        \end{center} \cellcolor{gndcolor} \\

    %%% 此处分割一行的第2至3个单元格
    \hline
    %\cline{2-3}
    
        \begin{center}
           区域预警
        \end{center} &
        \begin{center}
            依据识别在特殊位置的敌方机器人并发送预警信息
        \end{center} &
        \begin{center}
            通过点云图分区以及定位获取位置,然后通过裁判系统向己方预警
        \end{center} \\
        
    \hline
    
        \begin{center}
            协助英雄进行吊射
        \end{center} &
        \begin{center}
            获取己方英雄的位置并计算出与地方基地的距离然后将数据发回英雄
        \end{center} &
        \begin{center}
            识别我方英雄的位置,通过位置计算与敌方基地的距离,通过裁判系统发回信息
        \end{center} \\
        
    \hline
    
\end{longtable}
%%% 表格演示-END
%%% 其中,开头的 { X | X | X | X |} 中,竖线标识分割线,字母X为自动对齐,其余可替换的lrc分别为左对齐、右对齐、居中对齐
%%% 该表格的一大特性是跨页后仍保留表头内容,即 \endhead 前的内容