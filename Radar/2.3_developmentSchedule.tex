%%% 本文档作为功能演示

%%% 表格演示-BEGIN
\begin{longtable}{ p{2cm} | p{3cm} | p{3cm} | p{4.8cm} | p{2cm} |}

    \hline

    %%% 此处标识以上为表格脚,即在每页的表格底部重复显示的内容(一条横线)
    \endfoot
    
    %%% 此处标识该行颜色为指定定义的颜色,tabhdcolor 在 ./introduction.tex 中定义,颜色为hsb(0, 0, 0.82353),即82度灰
    \rowcolor{tabhdcolor}

        \begin{center}
            项目
        \end{center}  &
        \begin{center}
            任务
        \end{center}  &
        \begin{center}
           人力评估
        \end{center} &
        \begin{center}
            人员技能要求
        \end{center}  &
        \begin{center}
            耗时评估
        \end{center}  \\ 
        
    \hline

    %%% 此处标识以上为表格头,即在每页的表格头部重复显示的内容(一堆header)
    \endhead

        %%% 第2行的表格信息
        %\multirow{3}{*}{机械组} &
        \begin{center}
            点云匹配
        \end{center} &
        \begin{center}
            实现点云匹配算法以及熟悉激光雷达的部署使用
        \end{center} &
        \begin{center}
            1人
        \end{center} &
        \begin{center}
            实现点云匹配算法
        \end{center} &
        \begin{center}
            1个月
        \end{center}\\
        
    \hline
        \begin{center}
            新地图建模
        \end{center} &
        %%% 第3行的表格信息
        \begin{center}
            对新赛季地图进行建模
        \end{center} &
        %%% 第3行的第2~3列单元格被合并,左对齐,且左右均有分割线
        %\multicolumn{2}{| l |}{content 3 2-3} &
        \begin{center}
            1人
        \end{center} &
        \begin{center}
            利用blender建模
        \end{center} &
        \begin{center}
            20天
        \end{center}\\

    \hline
    
        \begin{center}
            神经网络识别改进
        \end{center} &
        \begin{center}
            搭建多层识别网络并对模型进行训练
        \end{center} &
        \begin{center}
            1~2人
        \end{center} &
        \begin{center}
            搭建神经网络,收集数据集,模型训练
        \end{center} &
        \begin{center}
            25天
        \end{center} \\
        
    \hline
    
        %%% 第4行的表格信息
        %%% 第4行开始,连续2行的第1列单元格被合并,自动列宽
        %\multirow{3}{*}{电控组} &
        \begin{center}
            搭建对哨兵的串口
        \end{center} &
        \begin{center}
            搭建专门接收哨兵信息的串口通讯接口
        \end{center} &
        \begin{center}
            1人
        \end{center} &
        \begin{center}
            串口通讯
        \end{center} &
        \begin{center}
            一周
        \end{center}\\

    \hline
    %%% 此处分割一行的第2至3个单元格
    %\cline{2-3}
    
        %%% 第5行的表格信息
        %%% 第5行由于第1列在上一行已被合并,所以建议不要写入信息,虽然写入信息也会显示出
        \begin{center}
            
        \end{center} &
        \begin{center}
            
        \end{center} &
        \begin{center}
            
        \end{center} &
        \begin{center}
            
        \end{center} &
        \begin{center}
            
        \end{center}\\

    \hline
    
        \begin{center}
            
        \end{center} &
        \begin{center}
            
        \end{center} &
        \begin{center}
            
        \end{center} &
        \begin{center}
            
        \end{center} &
        \begin{center}
            
        \end{center} \\
        
    \hline
    
        %%% 第6行的表格信息
        %\multirow{3}{*}{视觉组} &
        &
        &
        %%% 第6行开始,连续2行的第2~4列单元格被合并,居中对齐,且左侧有分割线,自动列宽
        %\multicolumn{3}{| c }{\multirow{2}{*}{content 6-7 2-4}} \\
        &
        &
        \\

    \hline
    %%% 此处分割一行的第1至1个单元格,即只有第一个单元格下面有横线
    %\cline{1-1}
    
        %%% 第7行的表格信息
        %content 7 1 &
        %%% 第7行由于第2~4列在上一行已被合并,所以建议不要写入信息,虽然写入信息也会显示出,而且建议像下面一样合并表格防止竖线分割
        %\multicolumn{3}{| c }{} \\

        &
        &
        &
        &
        \\

    \hline

        &
        &
        &
        &
        \\
        
    \hline

        %\multirow{3}{*}{硬件组} &
        &
        &
        &
        &
        \\

    \hline

        &
        &
        &
        &
        \\

    \hline

        &
        &
        &
        &
        \\
    
\end{longtable}
%%% 表格演示-END
%%% 其中,开头的 { X | X | X | X |} 中,竖线标识分割线,字母X为自动对齐,其余可替换的lrc分别为左对齐、右对齐、居中对齐
%%% 该表格的一大特性是跨页后仍保留表头内容,即 \endhead 前的内容