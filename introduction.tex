% 文档全局设定与引入必要宏包

\documentclass[a4paper, twoside, zihao=-4, sub3section]{article}
\usepackage[hmargin=0.79in, top=0.95in, bottom=1.13in]{geometry}
\geometry{headsep=0.14in, footskip=0.47in}

\usepackage{pdfpages}
\usepackage{amsmath}
\numberwithin{figure}{section}
\usepackage{fancyhdr}

\usepackage{graphicx}
\usepackage{float}

\usepackage{ltxtable}
\usepackage{colortbl}
\usepackage{multirow}
\usepackage{graphicx}

\usepackage{caption}

\captionsetup[figure]
{
    labelformat=empty
}

% 字体配置

\usepackage[UTF8, heading=true]{ctex}
\setmainfont{Times New Roman}
\setsansfont{Arial}
\setCJKmainfont[AutoFakeBold, AutoFakeSlant]{SimSun.ttc}
\setCJKsansfont[AutoFakeBold, AutoFakeSlant]{SimHei.ttf}

% 段落格式配置

\linespread{1.5}
\setlength{\parskip}{5pt}

% 目录配置

\usepackage[titles]{tocloft}
\usepackage{hyperref}
% 去除引用红框,改变颜色
\hypersetup{colorlinks=true,linkcolor=black}
% 重新定义目录的标题样式
\renewcommand*\contentsname{\hfill \color{black}\zihao{-2}\textbf{目录} \hfill}
% 设置点间距
\renewcommand\cftdotsep{0.1}
\renewcommand\cftsecdotsep{\cftdotsep}
\CTEXsetup[name={,.}]{section}
% 调节缩进
\setlength{\cftsubsecindent}{1em}
\setlength{\cftsubsubsecindent}{3em}
% 调节编号宽度
\setlength{\cftsecnumwidth}{14pt}
\setlength{\cftsubsecnumwidth}{20pt}
\setlength{\cftsubsubsecnumwidth}{30pt}
% 调节行距
\setlength{\cftbeforesecskip}{0pt}

% 标题配置

\usepackage{xcolor}
\newcommand{\titlecolor}{\color[RGB]{51, 51, 153}}
\usepackage{titlesec}
\titleformat{\section}
  {\titlecolor\zihao{2}\bfseries}
  {\thesection.}{0.5em}{}
\titleformat{\subsection}
  {\titlecolor\zihao{-2}\bfseries}
  {\thesubsection}{0.5em}{}
\titleformat{\subsubsection}
  {\titlecolor\zihao{3}\bfseries}
  {\thesubsubsection}{0.5em}{}
% 四级标题
\setcounter{secnumdepth}{4}
\titleformat{\paragraph}
  {\titlecolor\zihao{-3}\bfseries}
  {\theparagraph}{0.5em}{}

% 表格样式配置

\definecolor{tabhdcolor}{hsb}{0, 0, 0.72353}
\definecolor{gndcolor}{hsb}{0, 0, 0.96}
\AtBeginEnvironment{longtable}{\sffamily}
\renewcommand{\arraystretch}{1.1}