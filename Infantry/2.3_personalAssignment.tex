%%% 本文档作为功能演示

%%% 表格演示-BEGIN
\begin{longtable}{ p{2cm} | p{3.5cm} | p{9.3cm} |}

    \hline

    %%% 此处标识以上为表格脚,即在每页的表格底部重复显示的内容(一条横线)
    \endfoot
    
    %%% 此处标识该行颜色为指定定义的颜色,tabhdcolor 在 ./introduction.tex 中定义,颜色为hsb(0, 0, 0.82353),即82度灰
    \rowcolor{tabhdcolor}

        %%% 第1行的表格信息
        \begin{center}
            组别
        \end{center} &
        \begin{center}
            姓名
        \end{center} &
        \begin{center}
            主要工作
        \end{center} \\
        %%% 第1行的第3~4列单元格被合并,且左侧有分割线,而右侧没有
        %\multicolumn{2}{| c }{header 1 3-4} \\

    %%% 此处完整分割一行
    \hline

    %%% 此处标识以上为表格头,即在每页的表格头部重复显示的内容(一堆header)
    \endhead

        %%% 第2行的表格信息
        %\multirow{3}{*}{机械组} &
        \begin{center}
            机械
        \end{center} &
        \begin{center}
            严浩
        \end{center} &
        \begin{center}
            优化轮腿步兵的结构,研发狗腿步兵
        \end{center}  \\
        
    \hline
        %%% 第3行的表格信息
        \begin{center}
            机械
        \end{center}&
        \begin{center}
            汪明暄
        \end{center}&
        \begin{center}
            研发舵轮步兵底盘
        \end{center}\\

    \hline

        \begin{center}
            机械
        \end{center}&
        \begin{center}
            杨胜锋
        \end{center}&
        \begin{center}
            研发舵轮步兵云台
        \end{center}\\
        
    \hline
    
        \begin{center}
            机械
        \end{center}&
        \begin{center}
            黄德曦
        \end{center}&
        \begin{center}
            对麦轮步兵进行维护并迭代
        \end{center}\\

    \hline
    
        \begin{center}
            电控
        \end{center}&
        \begin{center}
            陈开烁
        \end{center}&
        \begin{center}
            调试和研发轮腿步兵的功能
        \end{center}\\
        
    \hline
    
        \begin{center}
            电控
        \end{center}&
        \begin{center}
            梁红涛
        \end{center}&
        \begin{center}
            调试和研发麦轮以及全向轮步兵的功能
        \end{center}\\

    \hline
    
        \begin{center}
            电控
        \end{center}&
        \begin{center}
            唐源
        \end{center}&
        \begin{center}
            协作调试和研发轮腿步兵的功能
        \end{center}\\

    \hline

        \begin{center}
            电控
        \end{center}&
        \begin{center}
            石金杨
        \end{center}&
        \begin{center}
            研发和调试舵轮步兵的功能
        \end{center}\\
        
    \hline

        \begin{center}
            电控
        \end{center}&
        \begin{center}
            黄盛任
        \end{center}&
        \begin{center}
            协助步兵的调试和测试
        \end{center}\\

    \hline

        \begin{center}
            视觉
        \end{center}&
        \begin{center}
            卢敏妍
        \end{center}&
        \begin{center}
            调试优化自瞄
        \end{center}\\

    \hline

        \begin{center}
            硬件
        \end{center}&
        \begin{center}
            冯志轩
        \end{center}&
        \begin{center}
            负责步兵滑环板、中心板等硬件设计
        \end{center}\\
    
\end{longtable}
%%% 表格演示-END
%%% 其中,开头的 { X | X | X | X |} 中,竖线标识分割线,字母X为自动对齐,其余可替换的lrc分别为左对齐、右对齐、居中对齐
%%% 该表格的一大特性是跨页后仍保留表头内容,即 \endhead 前的内容