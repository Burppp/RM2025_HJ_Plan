%%% 本文档作为功能演示

%%% 表格演示-BEGIN
\begin{longtable}{ p{2cm} | p{3cm} | p{3cm} | p{4.8cm} | p{2cm} |}

    \hline

    %%% 此处标识以上为表格脚,即在每页的表格底部重复显示的内容(一条横线)
    \endfoot
    
    %%% 此处标识该行颜色为指定定义的颜色,tabhdcolor 在 ./introduction.tex 中定义,颜色为hsb(0, 0, 0.82353),即82度灰
    \rowcolor{tabhdcolor}

        \begin{center}
            项目
        \end{center}  &
        \begin{center}
            任务
        \end{center}  &
        \begin{center}
           人力评估
        \end{center} &
        \begin{center}
            人员技能要求
        \end{center}  &
        \begin{center}
            耗时评估
        \end{center}  \\ 
        
    \hline

    %%% 此处标识以上为表格头,即在每页的表格头部重复显示的内容(一堆header)
    \endhead

        %%% 第2行的表格信息
        %\multirow{3}{*}{机械组} &
        \begin{center}
            熟悉LQR与VMC
        \end{center} &
        \begin{center}
            阅读知乎关于VMC的文章以及b站对LQR的讲解
        \end{center} &
        \begin{center}
            1名电控成员
        \end{center} &
        \begin{center}
            有线性代数和现代控制理论的基础
        \end{center} &
        \begin{center}
            2周
        \end{center}\\
        
    \hline
        \begin{center}
            Pid
        \end{center} &
        %%% 第3行的表格信息
        \begin{center}
            优化pid算法
        \end{center} &
        %%% 第3行的第2~3列单元格被合并,左对齐,且左右均有分割线
        %\multicolumn{2}{| l |}{content 3 2-3} &
        \begin{center}
            1名电控组成员
        \end{center} &
        \begin{center}
            熟悉微分先行等pid优化方式,熟悉pid计算算法
        \end{center} &
        \begin{center}
            2周
        \end{center}\\

    \hline
    
        \begin{center}
            主动关闭电容
        \end{center} &
        \begin{center}
            添加主动关闭电容功能
        \end{center} &
        \begin{center}
            1名电控组成员
        \end{center} &
        \begin{center}
            熟悉电容代码
        \end{center} &
        \begin{center}
            1天
        \end{center} \\
        
    \hline
    
        %%% 第4行的表格信息
        %%% 第4行开始,连续2行的第1列单元格被合并,自动列宽
        %\multirow{3}{*}{电控组} &
        \begin{center}
            测试能量消耗
        \end{center} &
        \begin{center}
            测试各种情况下能量消耗
        \end{center} &
        \begin{center}
            1名电控组成员,1名硬件组成员
        \end{center} &
        \begin{center}
            制作功率计、熟悉monitor、了解基本物理知识
        \end{center} &
        \begin{center}
            2周
        \end{center}\\

    \hline
    %%% 此处分割一行的第2至3个单元格
    %\cline{2-3}
    
        %%% 第5行的表格信息
        %%% 第5行由于第1列在上一行已被合并,所以建议不要写入信息,虽然写入信息也会显示出
        \begin{center}
            舵轮底盘
        \end{center} &
        \begin{center}
            完成舵轮控制代码的编写,以及各项参数的调试
        \end{center} &
        \begin{center}
            1名电控组成员
        \end{center} &
        \begin{center}
            熟悉舵轮运动解算和控制框架
        \end{center} &
        \begin{center}
            1周
        \end{center}\\

    \hline
    
        \begin{center}
            研发弹道计算代码
        \end{center} &
        \begin{center}
            开发适应不同距离的自适应弹道算法
        \end{center} &
        \begin{center}
            2名视觉组成员
        \end{center} &
        \begin{center}
            学习具体抛物线和各种初始参数
        \end{center} &
        \begin{center}
            3周
        \end{center} \\
        
    \hline
    
        \begin{center}
            狗腿构型demo
        \end{center} &
        \begin{center}
            做出狗腿构型demo测试方案可行性
        \end{center} &
        \begin{center}
            1名机械组成员,1名电控组成员
        \end{center} &
        \begin{center}
            熟练使用solidworks、了解布线细节、熟悉控制框架
        \end{center} &
        \begin{center}
            2周
        \end{center} \\

    \hline
    
\end{longtable}
%%% 表格演示-END
%%% 其中,开头的 { X | X | X | X |} 中,竖线标识分割线,字母X为自动对齐,其余可替换的lrc分别为左对齐、右对齐、居中对齐
%%% 该表格的一大特性是跨页后仍保留表头内容,即 \endhead 前的内容