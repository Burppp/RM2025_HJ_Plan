%%% 本文档作为功能演示

%%% 表格演示-BEGIN
\begin{longtable}{ p{2cm} | p{7.8cm} | p{6cm} |}

    \hline

    %%% 此处标识以上为表格脚,即在每页的表格底部重复显示的内容(一条横线)
    \endfoot
    
    %%% 此处标识该行颜色为指定定义的颜色,tabhdcolor 在 ./introduction.tex 中定义,颜色为hsb(0, 0, 0.82353),即82度灰
    \rowcolor{tabhdcolor}

        %%% 第1行的表格信息
        \begin{center}
            功能
        \end{center} &
        \begin{center}
            需求分析
        \end{center} &
        \begin{center}
            设计思路
        \end{center} \\

    %%% 此处完整分割一行
    \hline

    %%% 此处标识以上为表格头,即在每页的表格头部重复显示的内容(一堆header)
    \endhead

        %%% 第2行的表格信息
        \begin{center}
            稳定的倒地自救
        \end{center} &
        \begin{center}
            在赛场中被对方机器人撞击或在飞坡时踩到对方机器人,可能会发生翻倒。
        \end{center} &
        \begin{center}
            车检测到机体俯仰角不平衡时,若此时关节位置异常,则关闭轮毂和关节输出,用电机本身的位置pid使关节恢复正常角度,再用轮毂平衡,平衡后再开启关节输出。
        \end{center} \\
        
    \hline

        \begin{center}
            蹭台阶。
        \end{center} &
        \begin{center}
            相比于通过跳跃获取公路区地形跨越增益,不少队伍选用高腿长蹭台阶的策略,稳定性高。
        \end{center} &
        \begin{center}
            预备时选用最高档腿长,加速向台阶,当检测到机体俯仰角倾倒时收腿。
        \end{center} \\
        
    \hline
    
        \begin{center}
            穿越隧道 
        \end{center} \cellcolor{gndcolor} &
        \begin{center}
            新赛季地形需要步兵具有更加小的体积去适应各种地形,能够穿越狗洞将能更快到达高地区减少能量消耗
        \end{center} \cellcolor{gndcolor} &
        \begin{center}
            将原有的云台进行改进,降低云台高度,使步兵可以顺利的通过隧洞
        \end{center} \cellcolor{gndcolor} \\

    %%% 此处分割一行的第2至3个单元格
    \hline
    %\cline{2-3}
    
        \begin{center}
            稳定飞坡
        \end{center} &
        \begin{center}
            新赛季英雄吊射需要进入部署模式,提高飞坡稳定性,快速打击敌方车辆十分重要
        \end{center} &
        \begin{center}
            电控方面,优化功率控制,完善电容充放电操作
        \end{center} \\
        
    \hline
    
        \begin{center}
            稳定跨越复杂地形
        \end{center} &
        \begin{center}
            能不依靠跳跃在规定时间内登上高地或跨越其他地形
        \end{center} &
        \begin{center}
            设计四轮足式底盘(麦轮+狗腿构型)
        \end{center} \\

    %%% 此处分割一行的第1至1个单元格,即只有第一个单元格下面有横线
    %\cline{1-1}
    \hline
    
        \begin{center}
            Pid优化
        \end{center} &
        \begin{center}
            步兵普遍存在yaw轴角度环pout周期性振荡产生莫名响声的情况,需要优化pid。或是为了视觉击打需求优化轮腿步兵的pid有转向、防劈叉、腿长和Roll。选择合适的参数可以使机器人在赛场中的表现更优越
        \end{center} &
        \begin{center}
            对于yaw异响问题,增加微分先行,ols拟合的同时,优化pid计算函数。对于满足视觉需求加快响应速度等要求,添加前馈,梯形积分等
            \newline 防劈叉:把它调得贼硬,避免发生劈叉动作(能做到稳定可控有用的劈叉动作另说)
            \newline 腿长:PD控制,做到收敛快,不超调且软(按下去像弹簧一样),p和d不能太大,根据经验,p < d
            \newline Roll : 同样是PD,做到收敛快,不超调且软,不能加入积分项,避免车侧翻时误差累积
        \end{center} \\
        
    \hline
    
        \begin{center}
            稳定发射机构
        \end{center} &
        \begin{center}
            在上赛季的基础上,对发射机构以及相关算法进行优化完善,避免卡弹,单发变双发
        \end{center} &
        \begin{center}
            电控方面,尝试新的堵转判断方式,减少堵转。机械方面,通过更改u型轴承间距,避免限位过紧
        \end{center} \\
        
    \hline
    
        \begin{center}
            键鼠控制逻辑优化
        \end{center} &
        \begin{center}
            新赛季换新操作手,不同操作习惯不同,做到不同车不同操作逻辑
        \end{center} &
        \begin{center}
            与操作手沟通修改键鼠控制逻辑,自己上手操控机器人进行对战,发现当前键鼠控制逻辑定制化进行修改
        \end{center} \\
        
    \hline
    
        \begin{center}
            比赛期间20000J能量使用策略制定
        \end{center} &
        \begin{center}
            新赛季20000J能量上限需要对能量分配进行制定,需要了解不同功率上限下进行各类行动所需能量,对能量消耗制定策略
        \end{center} &
        \begin{center}
            测算进行飞坡,上各个角度坡,走过设定进攻路线所消耗的能量。编写操作手主动关闭电容操作
        \end{center} \\
        
    \hline
    
        \begin{center}
            舵轮底盘
        \end{center} &
        \begin{center}
            实现基本的全向移动和陀螺
        \end{center} &
        \begin{center}
            使用队内现有可行的舵轮控制方案,通过舵轮运动学解算,实现舵轮全向移动和陀螺
        \end{center} \\
        
    \hline
    
\end{longtable}
%%% 表格演示-END
%%% 其中,开头的 { X | X | X | X |} 中,竖线标识分割线,字母X为自动对齐,其余可替换的lrc分别为左对齐、右对齐、居中对齐
%%% 该表格的一大特性是跨页后仍保留表头内容,即 \endhead 前的内容