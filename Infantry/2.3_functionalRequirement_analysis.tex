%%% 本文档作为功能演示

%%% 表格演示-BEGIN
\begin{longtable}{ p{2cm} | p{7.8cm} | p{6cm} |}

    \hline

    %%% 此处标识以上为表格脚,即在每页的表格底部重复显示的内容(一条横线)
    \endfoot
    
    %%% 此处标识该行颜色为指定定义的颜色,tabhdcolor 在 ./introduction.tex 中定义,颜色为hsb(0, 0, 0.82353),即82度灰
    \rowcolor{tabhdcolor}

        %%% 第1行的表格信息
        \begin{center}
            功能
        \end{center} &
        \begin{center}
            需求分析
        \end{center} &
        \begin{center}
            设计思路
        \end{center} \\

    %%% 此处完整分割一行
    \hline

    %%% 此处标识以上为表格头,即在每页的表格头部重复显示的内容(一堆header)
    \endhead

        %%% 第2行的表格信息
        \begin{center}
            稳定的倒地自救
        \end{center} &
        \begin{center}
            在赛场中被对方机器人撞击或在飞坡时踩到对方机器人,可能会发生翻倒。
        \end{center} &
        \begin{center}
            车检测到机体俯仰角不平衡时,若此时关节位置异常,则关闭轮毂和关节输出,用电机本身的位置pid使关节恢复正常角度,再用轮毂平衡,平衡后再开启关节输出。
        \end{center} \\
        
    \hline

        \begin{center}
            蹭台阶。
        \end{center} &
        \begin{center}
            相比于通过跳跃获取公路区地形跨越增益,不少队伍选用高腿长蹭台阶的策略,稳定性高。
        \end{center} &
        \begin{center}
            预备时选用最高档腿长,加速向台阶,当检测到机体俯仰角倾倒时收腿。
        \end{center} \\
        
    \hline
    
        \begin{center}
            外壳快拆设计 
        \end{center} \cellcolor{gndcolor} &
        \begin{center}
            因为原先的保护壳并没有使用快拆的设计,导致在平常测试与在赛场维修时出现了较多的不便,带来了不必要的时间浪费
        \end{center} \cellcolor{gndcolor} &
        \begin{center}
            设计上队内先进行沟通,先是电控和视觉来提出需要较常拆装和维修的位置,再由机械来设计出快拆功能;快拆主要使用了合页与弹簧搭扣的配合设计
        \end{center} \cellcolor{gndcolor} \\

    %%% 此处分割一行的第2至3个单元格
    \hline
    %\cline{2-3}
    
        \begin{center}
            
        \end{center} &
        \begin{center}
            
        \end{center} &
        \begin{center}
            
        \end{center} \\
        
    \hline
    
        \begin{center}
            
        \end{center} &
        \begin{center}
            
        \end{center} &
        \begin{center}
            
        \end{center} \\

    %%% 此处分割一行的第1至1个单元格,即只有第一个单元格下面有横线
    %\cline{1-1}
    \hline
    
        \begin{center}
            
        \end{center} &
        \begin{center}
            
        \end{center} &
        \begin{center}
            
        \end{center} \\
        
    \hline
    
        \begin{center}
            
        \end{center} &
        \begin{center}
            
        \end{center} &
        \begin{center}
            
        \end{center} \\
        
    \hline
    
        \begin{center}
            
        \end{center} &
        \begin{center}
            
        \end{center} &
        \begin{center}
            
        \end{center} \\
        
    \hline
    
        \begin{center}
            
        \end{center} &
        \begin{center}
            
        \end{center} &
        \begin{center}
            
        \end{center} \\
        
    \hline
    
        \begin{center}
            
        \end{center} &
        \begin{center}
            
        \end{center} &
        \begin{center}
            
        \end{center} \\
        
    \hline
    
\end{longtable}
%%% 表格演示-END
%%% 其中,开头的 { X | X | X | X |} 中,竖线标识分割线,字母X为自动对齐,其余可替换的lrc分别为左对齐、右对齐、居中对齐
%%% 该表格的一大特性是跨页后仍保留表头内容,即 \endhead 前的内容