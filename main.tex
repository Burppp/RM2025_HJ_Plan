% 标单百分号的注释,如本句所示。是原本模板注释
%%% 标三个百分号的注释,如本句所示。是辅助使用者理解的指导性演示,以“-BEGIN”与“-END”进行标记代码的起始位置和结束位置

%%% 与此同时,本文档作为包含子文件与分页符演示,其余功能演示内容详见 ./section/2.3.1_步兵机器人.tex 文档

% 引入导言区配置文件

%%% 包含子文件演示-BEGIN
% 文档全局设定与引入必要宏包

\documentclass[a4paper, twoside, zihao=-4, sub3section]{article}
\usepackage[hmargin=0.79in, top=0.95in, bottom=1.13in]{geometry}
\geometry{headsep=0.14in, footskip=0.47in}

\usepackage{pdfpages}
\usepackage{amsmath}
\numberwithin{figure}{section}
\usepackage{fancyhdr}

\usepackage{graphicx}
\usepackage{float}

\usepackage{ltxtable}
\usepackage{colortbl}
\usepackage{multirow}
\usepackage{graphicx}

\usepackage{caption}

\captionsetup[figure]
{
    labelformat=empty
}

% 字体配置

\usepackage[UTF8, heading=true]{ctex}
\setmainfont{Times New Roman}
\setsansfont{Arial}
\setCJKmainfont[AutoFakeBold, AutoFakeSlant]{SimSun.ttc}
\setCJKsansfont[AutoFakeBold, AutoFakeSlant]{SimHei.ttf}

% 段落格式配置

\linespread{1.5}
\setlength{\parskip}{5pt}

% 目录配置

\usepackage[titles]{tocloft}
\usepackage{hyperref}
% 去除引用红框,改变颜色
\hypersetup{colorlinks=true,linkcolor=black}
% 重新定义目录的标题样式
\renewcommand*\contentsname{\hfill \color{black}\zihao{-2}\textbf{目录} \hfill}
% 设置点间距
\renewcommand\cftdotsep{0.1}
\renewcommand\cftsecdotsep{\cftdotsep}
\CTEXsetup[name={,.}]{section}
% 调节缩进
\setlength{\cftsubsecindent}{1em}
\setlength{\cftsubsubsecindent}{3em}
% 调节编号宽度
\setlength{\cftsecnumwidth}{14pt}
\setlength{\cftsubsecnumwidth}{20pt}
\setlength{\cftsubsubsecnumwidth}{30pt}
% 调节行距
\setlength{\cftbeforesecskip}{0pt}

% 标题配置

\usepackage{xcolor}
\newcommand{\titlecolor}{\color[RGB]{51, 51, 153}}
\usepackage{titlesec}
\titleformat{\section}
  {\titlecolor\zihao{2}\bfseries}
  {\thesection.}{0.5em}{}
\titleformat{\subsection}
  {\titlecolor\zihao{-2}\bfseries}
  {\thesubsection}{0.5em}{}
\titleformat{\subsubsection}
  {\titlecolor\zihao{3}\bfseries}
  {\thesubsubsection}{0.5em}{}
% 四级标题
\setcounter{secnumdepth}{5}
\titleformat{\paragraph}
  {\titlecolor\zihao{-3}\bfseries}
  {\theparagraph}{0.5em}{}

% 表格样式配置

\definecolor{tabhdcolor}{hsb}{0, 0, 0.72353}
\definecolor{gndcolor}{hsb}{0, 0, 0.96}
\AtBeginEnvironment{longtable}{\sffamily}
\renewcommand{\arraystretch}{1.1}
%%% 包含子文件演示-END
%%% 包含子文件,相当于把被包含的文件全部内容复制粘贴到此处
\usepackage{enumitem}
\usepackage{amssymb}
\usepackage{pifont}
\usepackage{graphicx}
\usepackage{array}
\usepackage{longtable}
\usepackage{tabularx}
% 文档内容-BEGIN
\begin{document}

    % 引入封面
    % 封面

\includepdf{section/cover1.pdf}

% 编码页数从1开始
\setcounter{page}{1}

% 页眉页脚
\pagestyle{fancy}
\fancyhf{}
\fancyhead[EL, OR]{
  \includegraphics[height=0.15in]{figure/header_img.png}
}
\fancyfoot[EL]{
  \zihao{-4}\thepage
  \hspace{0.4cm}
  \raisebox{.15ex}{
    \color{gray}{
      \zihao{-5}© 2023 \ 大疆创新 \ \ 版权所有
    }
  }
}
\fancyfoot[OR]{
  \raisebox{.15ex}{
    \color{gray}{
      \zihao{-5}© 2023 \ 大疆创新 \ \ 版权所有
    }
  }
  \hspace{0.4cm}
  \zihao{-4}\thepage
}

% 目录
\begin{center}
  \zihao{5}\tableofcontents
\end{center}

\newpage
    
    \section*{前言}

\addcontentsline{toc}{section}{前言}
    
    \noindent
    本报告由深圳技术大学悍匠战队编制,适用于RoboMaster 2025机甲大师超级对抗赛。主要撰写人员包括:
    
    \noindent
    \LTXtable{\textwidth}{table/0_writersList.tex}
    
    %%% 分页符演示-BEGIN
    \newpage
    %%% 分页符演示-END
    %%% 分页符,相当于从此处开辟新的一页,无论上一页内容是否凑够一页,都会从新的一页开始
    
    \section{团队目标}


    \subsection{团队目标概述}

    \subsubsection{团队实际情况}

    \subsubsection{本赛季要完成的基础内容及进阶优化内容}


    \subsection{个方向上的目标}

    \subsubsection{目标成绩}

    \subsubsection{任务进度管理制度}

    \subsubsection{队员管理制度}

    \subsubsection{物资管理制度}

    \subsubsection{新人培养体系}

    \subsubsection{管理流程改变}

    \subsubsection{技术传承建设}

    \subsection{技术突破}

    \setlist[itemize]{label=\raisebox{-1.2ex}{\scalebox{3}{$\textbullet$}}}

    \begin{itemize}
        \item 在机械方面,实现 17mm 弹丸下供弹结构设计,实现机械臂整体设计制造,并与工程机器人适配,实现无人机机架轻量化设计,研发三摩擦轮与双极摩擦轮的发射机构,研发拉簧蓄能飞镖发射机构,研发月球车爬台阶底盘机构,预研狗腿结构。
        \item 在电控层面,实现轮腿平衡步兵机器人的稳定控制,在底盘、云台、发射等模块引入前馈控制进一步提高机器人性能,在IMU等传感器模块引入高阶和优化滤波算法处理反馈的原始数据使数据平滑,实现机械臂运动学解算及路径规划控制,在下位机上实现机械臂正运动学解算,实现步兵、英雄、哨兵机器人与雷达之间的通讯接口,使用机器学习和强化学习优化哨兵机器人行为树的自主决策,着手研发自生成行为树。
        \item 在视觉方面,实现自主修正弹道或者实现快速手动校准,实现高精度反前哨站、飞镖镖架视觉制导;提升传统识别和神经网络识别装甲板速度,哨兵实现更加精准的定位与导航、实现全向感知、实现辅助雷达标定;雷达实现更精确的标定和与哨兵的通信。
    \end{itemize}
    
    \newpage
    
    \section{复盘分析}

    \subsection{整体分析}

    上一赛季的对抗赛我们的队伍未出线,成绩不理想,但相较于更早些时候的失败,机器人的性能有显著上升,稳定性仍然欠佳。\par
    整体而言上个赛季备赛过程中出现的错误以及进度落后归根结底是赛季初的决策失误导致的,在赛季初不应当因为经济富余、时间充分而采取极端激进的备赛策略,应当量力而行,建立在上一赛季的基础之上维护、测试、优化,循环往复,在进度稳定推进之余尝试创新或接纳新技术。\par
    在新赛季,我们将着力于测试开发、对机器人进一步迭代和优化,继承原有基础,深入学习钻研应用广泛的通用技术,同时积极推动新技术的研发、丰富技术储备。\par

    \input{section/2.2_problemChallenge}

    \subsection{上赛季的成功方面}

    \setlist[itemize]{label=\raisebox{-1.2ex}{\scalebox{3}{$\textbullet$}}}

    \begin{itemize}
        \item 2024赛季中更新了除飞镖以外的所有兵种、新增空中机器人参加对抗赛。
        \item 上赛季无论是成本上、人力上对主力兵种的投入相当大,备赛过程中维护充分。
        \item 24赛季中大大提高了各兵种的性能上限,25赛季中应当专注于如何将高性能稳定、充分地在赛场中发挥。
        \item 上赛季置办了较为完整的备赛工具,减轻了25赛季的成本负担。
        \item 上赛季积极推动了技术传承,交接时保留了各兵种的技术文档。
        \item 24赛季的主体管理架构较为完善,延续至25赛季并逐步优化。
        \item 在24赛季中深刻认识到测试的重要性,25赛季中会有所侧重。
    \end{itemize}

    \subsection{根因分析}

    \subsubsection{问题分析}

        \setlist[itemize]{label=\raisebox{-1.2ex}{\scalebox{3}{$\textbullet$}}}

        \begin{itemize}
            \item 为什么赛场上四号步兵底盘异常?
            \item 四号步兵右前轮电机三相线磨破短路。
            \item 为什么三相线会磨破?
            \item 检录时干涉、磨线。
            \item 为什么测试时会干涉、磨线,为什么缺乏测试?
            \item 结构设计时没有预留足够的走线空间。
            \item 为什么在赛场上才暴露出这个问题?
            \item 备赛过程中没有预留足够的测试时间。
            \item 为什么没有足够的测试时间?
            \item 早期进度规划过于理想化、队员不重视测试。\\
            
            \item 为什么赛场上无人机摇摇晃晃、飘忽不定?
            \item 光流不稳定且妙算电脑不能自启动。
            \item 为什么光流不稳定且没有更换电脑?
            \item 前期进度安排不重视光流,且电脑的选型考虑片面,仅仅考虑重量问题。
            \item 为什么不重视光流定高?
            \item 因为没有进行长期规划,仅仅追求阶段性成果,初始目标较低(不炸就行)。
            \item 为什么没有做好长期规划?
            \item 因为无人机作为此前没有尝试过的兵种,从零开始会遇到很多困难,更多的预期和注意力放在了传统的地面作战单元上。\\
            
            \item 为什么无人机在比赛中出现pitch轴卡住的情况?
            \item 因为pitch的限位逻辑是不适用于无人机的工况的,整机会出现较大的前倾或后仰的。
            \item 为什么限位逻辑不适应具体工况也没有及时纠偏?
            \item 事实上队内已经有合适的解决方案,因为没有充分及时的沟通,这个问题到赛前也没有解决。
            \item 为什么队内会存在这样的信息差?
            \item 队员总是专注于眼前的难题、并不关注通用技术泛化到其他兵种。\\
            
            \item 为什么赛场上无人机会突然出现掉高的情况?
            \item 因为分电方案是两并两串,其他需要24v电压的设备从串联中间取电,导致在开启摩擦轮或者拨盘堵转的时候耗电量突增,整体电压下降而掉高。
            \item 为什么没有设计出合适的无人机分电方案?
            \item 硬件组人手短缺、技术传承艰难。\\
            
            \item 为什么进度管理如此艰难?
            \item 进度管理流程繁杂,负责人无法自觉上传车组进度,需要队长、项管挨个问。
            \item 为什么没有优化进度管理流程?
            \item 队员总是专注于攻克技术难题,无法估计整体的进度规划。\\

            \item 为什么经费管理没有预计的规范化?
            \item 因为造新车大量的物资需求和迟滞的进度,有大量的物资采购、物流信息来往。
            \item 为什么大量的物资无法妥善分配管理?
            \item 采购物资太散太细,填写采购时并不考虑节约物资,财务人手短缺。\\

            \item 为什么大一新人参与度普遍低?
            \item 因为培训后,备赛过程中疏于培养新人,没有给他们更多参与的机会。
            \item 为什么疏于培养新人?
            \item 进度滞后时主力开发在推动进度,空闲时懒得教基础尚浅的新人,新人难以一同进步,两方隔阂越来越大,最终导致梯队人员流失。\\

            \item 为什么备赛时有人空闲有人忙碌?
            \item 赛季规划时没有充分考虑每个主力开发的能力和时间精力,只考虑了每一个功能需求有人去完善。
            \item 为什么规划时没有充分考虑人员需求?
            \item 没有意识到队员的成长总是非线性的,很大程度上取决于对比赛和机器人的热爱,而不是已有的技术基础。\\

            \item 为什么工具、螺丝、标准件线材整理混乱?
            \item 因为各组进度不一,队员取用工具、标准件等习惯不好,整体不注重实验室工具整齐。
            \item 为什么各组进度差异较大、取用工具不规范?
            \item 缺少一套详细的备赛流程和进度规划,和落实进度规划的决心和方案。\\

            \item 为什么场地利用率低?
            \item 整体缺乏测试,而且没有意识到测试的重要性。\\

            \item 为什么设计缺陷无法及时更改?
            \item 因为进度缓慢,大多数时候在被赛程推着走,技术上难以更进一步;由于缺乏测试,许多设计缺陷无法及时发现。\\

            \item 为什么到赛前还总是忙前忙后?
            \item 缺少一套详细的赛前维护、没有实现弹道自动校正、需要手动调整偏置。\\
        \end{itemize}

    \subsection{问题解决方案}

    \setlist[itemize]{label=\raisebox{-1.2ex}{\scalebox{3}{$\textbullet$}}}

    \begin{itemize}
        \item 增加测试频率和测试强度,总结出适应各个兵种的测试方案和测试流程。
        \item 在赛季中后期组织队内3V3,尽可能模拟赛场实况,培训操作手。
        \item 倾注更多人力物力投入远程击打兵种。
        \item 活跃实验室气氛,充分沟通交流,打通队内信息差。
        \item 硬件组独立招新、培训。
        \item 将主力开发的工位安排尽可能集中,便于了解各兵种进度。
        \item 优化物资采购,改为由组长审核、分批次购买,通过其他赛事吸引运营人才。
        \item 举办校内赛,鼓励新人参与实验室事项。
        \item 制定一套详细的赛前维护。
        \item 实现弹道自动校正。
    \end{itemize}

    \subsection{上赛季经验总结}

    \newpage
    
    \section{项目分析}


    \subsection{新赛季规则解读}

    \subsubsection{整体规则分析}

        整体规则上进一步平衡了地面作战单元的战力水平,删除了哨兵的无敌机制、删除平衡步兵概念及其增益、新增堡垒增益、新增地形跨越增益、对地面作战单元的20000J的底盘能量限制等,勾勒出的赛季轮廓是削弱地面作战单元的单一兵种独特优势,转向鼓励队伍拿到堡垒增益、地形跨越增益这种对于多兵种一视同仁的增益机制。这个赛季单论地面作战单元作战将会互相拉扯,更加胶着。\par
        地面作战单元难以取得巨大战术成果的情况下,远程击打能力将变得更加重要。无论是新增的英雄部署模式、飞镖三种靶位设置、无人机增强空中支援,都极大提升了远程击打兵种的上限。\par

    \subsubsection{比赛机制调整}

        \setlist[itemize]{label=\raisebox{-1.2ex}{\scalebox{3}{$\textbullet$}}}

        \begin{itemize}
            \item 初始金币与金币增益机制:初始获得400金币,每过一分钟自动获取50金币,在倒计时最后一分钟时获得150金币。这一机制与上赛季相同。
            \item 兑换规则变化:空中支援的兑换机制发生了变化,取消了次数上限。这一调整大大增强了空中支援的战略灵活性,可以根据需要更频繁地调用空中支援。
            \item 矿石兑换机制调整:兑换矿石的难度从五个级别精简为四个级别,降低了整体兑换的复杂性。而且在难度为一级和二级时,兑换银矿石和金矿石的收益显著提升。这使得低难度兑换更加有利,根据自身经济状况做出更加灵活的选择。难度为三级和四级的矿石兑换收益与上赛季的四级和五级相同,保持了高难度兑换的挑战性。
            \item 累计金币与兑换矿石的关联:新赛季中,累计金币数对兑换矿石的难度限制有所调整。相比上赛季,新赛季要求积累更多的金币,才能解锁最低难度的矿石兑换。这一变化提高了工程机器人的游戏下限,使得低难度兑换成为一种有效的盈利策略,减少了对高难度矿石兑换的依赖。
            \item 经验体系调整:新赛季中,击毁机器人所获得的经验值计算方式有所变化,获得的经验值大幅提升。值得注意的是,助攻行为将不再获得经验,所有经验都集中于击毁敌方机器人上。击打大能量机关将直接为击打的步兵英雄提供500经验。相比上赛季,经验不再平均分配给参与打击能量机关的步兵英雄,而是直接分配给所有场上存活的步兵英雄。这一改动提高了战场上的经验分配效率。
        \end{itemize}

    \subsubsection{场地调整}

        \setlist[itemize]{label=\raisebox{-1.2ex}{\scalebox{3}{$\textbullet$}}}

        \begin{itemize}
            \item 在2024年和2025年场地设计之间,经过全面的重新架构,场地结构得到了显著的优化和改进。
            \item 基地区改进:基地区新增了堡垒,提升了防御能力。而机器人启动区则取消了哨兵启动区,为场地提供了更多灵活性。此外,梯形高地的高度有所增加,坡度也被提升至43度,为战术部署提供了更多层次。
            \item 场地布局优化:基地区的左侧是梯形高地,右侧为公路区。由于前哨站位置的调整以及中央高地与基地区之间的衔接方式变化,英雄在开场时的最佳路径为向左行进,这样能够最大化作战效率和资源收获。
            \item 公路区设计:公路区包括台阶增益点、直角弯道、飞坡区以及与中央高地相接的区域。目前,公路区是本队英雄的唯一有效进攻路径,因此其战略意义不言而喻。
            \item 台阶增益点的设计:台阶增益点的设计旨在提升步兵的在各个地形的作战效能,尤其是对平衡步兵最为有利。然而,对于其他机器人而言,跨越此台阶存在一定挑战。其尺寸和重量是首先需要解决的问题。
            \item 敌方战略考虑:对于敌方的狗洞和公路区出口,其设计更多是为了战略需求。在进攻或撤退过程中,需要特别警惕敌方步兵机器人的出现,这将对战场态势产生重要影响。这些改进和设计考虑将进一步提升场地的战术深度和灵活性,为各方战力的发挥提供更加丰富的战术选择。
        \end{itemize}

    \subsubsection{各兵种规则分析}

        \paragraph{步兵规则分析}

            \setlist[itemize]{label=\raisebox{-1.2ex}{\scalebox{3}{$\textbullet$}}}
    
            \begin{itemize}
                \item 25赛季可在准备3分钟内预装弹,且不再限制初始预装弹数量,根据以往比赛经验,要求步兵机器人至少拥有400发的预装弹量。本赛季现有设计中,步兵机器人预装弹量已满足要求,但机械组仍需尝试在不牺牲其它性能的前提下继续扩大预装弹量。
                \item 25赛季步兵通往关键交战区中央高地共有四条可选路径:翻越小资源岛台阶、公路区、公路隧道、高地隧道。由于普通步兵不具备翻越台阶的能力,故小资源岛台阶不予考虑;公路区与公路隧道需要经过颠簸路段,且公路区路程较长,考虑25赛季新增使用能量总额限制(20000焦),为减少功耗,普通步兵在非必要情况下应避免选择这两条路径;高地隧道路程短且无需经过颠簸路段,是普通步兵出入中央高地的最佳路径,因此本赛季普通步兵机器人必须具备穿越隧道的能力,这就要求机械组必须设计出尺寸合适,能够穿越隧道的机器人。此外,步兵可通过飞坡直达敌方公路区,从而阻击敌方英雄吊射以及快速接近敌方基地区,这要求本队步兵机器人机械组必须研发出重心适中,具备稳定飞坡能力的步兵机器人。
                \item 相较于24赛季,平衡步兵的概念以及经验加成等规则红利被取消,归类到普通步兵,从前后两块大装甲板变为了四块小装甲板,更容易被对方机器人打击,看似轮腿步兵相对于普通步兵毫无优势,但因为其功率计算的特殊性,使得它能够在一个较小的功率内仍有较快的移动速度,进攻、回防速度快,同时可以限制对方英雄吊射;并且五连杆机构给予了轮腿步兵很强的灵活性,能够轻松通过起伏路段,实现蹭台阶和飞坡等地形跨越动作来获得地形跨越增益,因此轮腿步兵在比赛中应该充当速攻、抢点、回防的角色。
                \item 25赛季堡垒增益点仅在己方前哨站被击毁,且基地血量落后时生效,可被一台哨兵或步兵占领,获取无敌、枪口热量、发弹量等增益,在防守基地区时拥有巨大优势。当基地区遭到入侵且哨兵不能优先占领堡垒点时,步兵应积极占领堡垒点。
                \item 25赛季能量机关可由任意机器人激活,但出于经济考虑,激活能量机关的最佳选项仍是机动性最高的步兵,此外,激活能量机关后会为基地和前哨站带来防御增益,为机器人带来攻击增益,使得队伍更容易展开进攻或进行防守,因此能量机关的激活将是本赛季的重中之重。
                \item 25赛季步兵机器人仅靠发射弹丸、或在 3 分钟内对前哨站造成 500 点伤害、或成功击打对方机器人所带来的经验值是不足以升级的,这要求步兵机器人在本赛季要做到快速精准的打击,并与队友制定相关战术策略以更快提升等级。进攻时,步兵机器人应积极占领中央高地区域,为己方英雄和步兵带来增益,对对方机器人造成更有力的打击,防守时积极占领堡垒、前哨站和基地区,获取防御增益进行反击。
                \item 对于轮腿步兵,除上述占领增益点外,还能在飞坡点和公路区地形跨越增益点进行交互获得增益,因此要求电控的算法要具备很强的稳定性。
                \item 操作方面,25赛季与24赛季一致,可选择手动控制或半自动控制。半自动控制相比24赛季,增加了25\%的防御增益。半自动步兵在数值上将会比手动控制步兵有更大的优势。	
                \item 25赛季新增底盘能量总额20000焦限制,一旦超出限制,则机器人的功率永久削减,因此,本赛季应制定能量使用策略,减少非必要的能量消耗。
            \end{itemize}
        
        \paragraph{步兵策略战术}
    
            \setlist[itemize]{label=\raisebox{-1.2ex}{\scalebox{3}{$\textbullet$}}}
    
            \begin{itemize}
                \item 步兵在具有高机动性和灵活性的同时,对前哨站和基地的伤害收益小,且血量较少,因此步兵在比赛中的主要职能为占领增益点、掩护和支援我方其它机器人单位、阻击敌方英雄、攻击敌方地面机器人以及协同激活能量机关。
                \item 在开局针对中央高地的争夺战中,轮腿步兵因为功率计算和五连杆机构的特殊性,具有较强的机动性,可以通过蹭台阶迅速占领中央高地增益点并对对方机器人进行打击,同时,我方步兵应当尽量避免进入敌方哨兵的索敌范围。
                \item 当己方英雄吊射命中率较高时,步兵优先选择掩护英雄,防御敌方飞坡步兵;当英雄吊射命中率偏低时,选择从香蕉道到对方区域干扰敌方英雄吊射(具体能否实现需要看最终的尺寸),以及攻击敌方地面机器人。
            \end{itemize}
    
            \begin{figure}[H]
                \centering
                \includegraphics[height=0.35\textwidth]{figure/infantry_tactics.png}
                \hspace{0.5em}
                \label{fig:infantry_tactics}
            \end{figure}

        \paragraph{英雄规则分析}

            \setlist[itemize]{label=\raisebox{-1.2ex}{\scalebox{3}{$\textbullet$}}}
    
            \begin{itemize}
                \item 今年的英雄机器人引入了“部署模式”,在己方半场内,英雄机器人可通过2秒的确认时间进入“部署模式”,底盘断电后获得25\%的防御增益,并获得狙击攻击增益与经济加成。这一机制的引入,英雄在己方半场的任意位置,对敌方基地造成可观的伤害输出。“部署模式”需要其他兵种的协同支持,为团队协作和战术规划提供了更多可能性。同时规则对适用于部署模式的“比赛节奏”做出了一些限制,开局有2次命中能享受部署模式加成,此后每20s增加一次机会。
                
                \item 今年新增的42°坡提升了地形复杂度,对英雄机器人的底盘设计和功率分配提出了全新要求。传统的麦克纳姆轮加上英雄机器人的重量难以胜任这一高难度场景,地形上鼓励其他底盘结构(例如舵轮底盘)。42°坡本身是一个极具战略价值的狙击点,占领这里可以对战局产生重要影响。同时这一位置容易受到对方无人机的威胁,对时机的精准把控至关重要。解决这些技术难点,我们就能在比赛中抢占先机,充分释放英雄机器人的潜能。
                
                \item 能量体系:今年的规则更新中,英雄、步兵、哨兵机器人在底盘能量消耗达到20000J后将进入“虚弱”状态,对于比赛节奏和战术规划有新的要求。这一规则也对电容的性能和整车的能量控制系统提出了更高要求。如何提升电容效率、优化整车能量管理,将成为提高胜率的关键技术突破点。
                
                \item 赛制时间分析:在新赛季中,空中支援机制进行了调整,英雄在前30秒需要占领梯形高地、准备击打前哨站,30秒后将与无人机和对方英雄机器人形成制约。地形变化使得梯形高地的防御增益对英雄机器人的吊射和部署模式至关重要。若某方的梯形高地增益点仅被该方机器人占领,则在比赛的不同时间段(2-3分钟、3-5分钟、5-7分钟),占领的机器人可获得2、3、5倍的枪口热量冷却增益和25\%的防御增益。
                
                \item 雷达机制分析:雷达可以向所有己方机器人发送数据,接收哨兵机器人的数据。新规则下英雄吊射点位更加灵活,但在部署模式下底盘处于“失能状态”,此时很容易陷入被动。对于吊射点位的选择,可以通过感知敌方位置,为英雄操作手实时更新可以进行吊射的位置;或直接检测预设的吊射点位是否有敌方单位,方便操作手进行决策,提前预警,减少被抓风险,提高灵活性。还能在敌方进入基地范围内时,向英雄操作手发送”防守”的决策信号,调动英雄回防。
    
                \item 英雄狙击模式下,命中敌方基地会获得50金币。英雄狙击伤害可获得100经验值,跟上赛季一样,但是区别在于这赛季英雄新增的部署模式,可以随时狙击。也就是说英雄如果能利用好这个狙击能大大增长升级速度,同时前期也可以使用狙击击打较好命中的目标,加速升级。

                \item 而中央高地的20°坡、狗洞和弯曲的香蕉道狭道,对于目前战队的英雄机器人而言是无法通过的,战略意义不大。如果后续场地不加以大的改动,可以尝试制造一台超小英雄过此类狭窄通道。
    
                \begin{figure}[H]
                    \centering
                    \includegraphics[height=0.35\textwidth]{figure/trapezoidalElevation.png}
                    \hspace{0.5em}
                    \caption{\textbf{\zihao{-4}\textbf{梯形高地场地示意图}}}
                    \label{fig:trapezoidalElevation}
                \end{figure}

                \newpage

                \begin{figure}[H]
                    \centering
                    \includegraphics[height=0.35\textwidth]{figure/RM2024_map.png}
                    \hspace{0.5em}
                    % 添加居中、黑体、加大的字体
                    \caption{\textbf{\zihao{-4}\textbf{RM2024场地示意图}}}
                    \label{fig:RM2024_map}
                \end{figure}
                
                \begin{figure}[H]
                    \centering
                    \includegraphics[height=0.35\textwidth]{figure/RM2025_map.png}
                    \hspace{0.5em}
                    \caption{\textbf{\zihao{-4}\textbf{RM2025场地示意图}}}
                    \label{fig:RM2025_map}
                \end{figure}
    
                \item 场地中央的环形高地更换为中央高地,进入中央高地的路径也完全更改——24赛季可以由基地区直冲环形高地,但25赛季将公路区和中央高地直接连接,阻断了基地区直冲中央高地的行为,影响了英雄机器人冲锋的效率。并且中央高地和梯形高地改动,导致飞坡后英雄的路线选择减少,加大了飞坡的风险,由此可能要重新考虑飞坡的进攻价值性。

                \newpage
    
                \item 梯形高地最受瞩目的就是43°高地,600cm的高度让能够上到这个高地的英雄机器人安全性得到了保障,目前看来所有兵种之中对在高地上有较大威胁的是平衡步兵和无人机。此高地还是全场中最适合吊射基地的位置,且距离前哨站只有7~8米的距离,还没有任何的障碍物遮挡视野。可以说,它就是提供给英雄的专属发射场地,重要性不言而喻,所以己方梯形高地是英雄前期必须尝试占领的战略要地。在前哨站失守的中期焦灼阶段和后期进攻阶段都可以尝试飞坡占领敌方梯形高地,为英雄自己创造一个较为安全的输出基地环境。但实现上坡对于英雄机器人的性能有一定要求,不管是轮毂影响,还是整个底盘通过角和重心位置设计,对于每一个队伍都是一个大的挑战。
                
                \begin{figure}[H]
                    \centering
                    \includegraphics[height=0.35\textwidth]{figure/highwayArea.png}
                    \hspace{0.5em}
                    \caption{\textbf{\zihao{-4}\textbf{公路区场地示意图}}}
                    \label{fig:highwayArea}
                \end{figure}
    
                \item 直角弯对英雄机器人的转弯性能提出较高的水平要求,虽然在机械结构上麦轮底盘和舵轮底盘是可以通过,但对于操作手来说需要有熟练的操作才能减少过弯时间,避免磕碰,提高进攻效率。飞坡区域直接衔接到敌方梯形高地,并且只有梯形高地一条路可以走,一旦被敌方察觉并派遣机器人堵在梯形高地的出口,那么飞坡过去的英雄只能被瓮中之鳖,除非明确知道敌方所有机器人分布位置。在战略规划上,若非紧急情况不推荐英雄飞坡,但在功能规划上,英雄要具备飞坡功能。在通过与中央高地衔接部分和香蕉道的上坡部分,英雄机器人需要注意的是在进攻时段是否有敌方的步兵机器人存在,进攻或撤退时需要保证自身安全,此外没有任何其他过多要求。
    
    
                \begin{figure}[H]
                    \centering
                    \includegraphics[height=0.35\textwidth]{figure/centralTableland.png}
                    \hspace{0.5em}
                    \caption{\textbf{\zihao{-4}\textbf{中央高地场地示意图}}}
                    \label{fig:centralTableland}
                \end{figure}
    
                \item 关于中央高地英雄机器人需要注意的就是颠簸路段、敌方“香蕉道”入口、狗洞以及敌方公路区出口。对机器人在性能有较高要求的就颠簸路段,在机器人制作中需尽量减少颠簸的影响性,但在多年比赛的经验来看,过颠簸路段时,无法发射,无法防御,只能快速通过,不然只能当活靶子。当英雄机器人在中央高地的时候,一定要时刻注意背后的香蕉道入口,中央高地四通八达,前后都有可能面临来敌。
    
            \end{itemize}

        \paragraph{英雄策略战术}

            \setlist[itemize]{label=\raisebox{-1.2ex}{\scalebox{3}{$\textbullet$}}}
    
            \begin{itemize}
                \item 出门左拐就是梯形高地了,英雄在比赛开始后可以尝试站住,攻击敌方前哨站,并为己方提供对于中央高地的态势感知。在设想中,梯形高地的落差同时也可为英雄提供对低处攻击的掩蔽,因此这里应该是英雄经常来的地方。
                \item 部署机制以底盘断电为代价换取一系列加成,虽然此时英雄有25\%防御增益,但在潜在的陆空威胁下仍十分脆弱。在地形复杂且四通八达的环境下,要求己方为英雄构建安全的“堡垒地带”显然难度较高,为此英雄操作手应该留意敌方无人机与地面机器人的出动与补给节奏(后半场可能的充电高峰期),见缝插针地抵近输出敌方前哨站与基地,并在危险来临时及时转入防守。
                \item 虽然组委会提过会增加飞坡之后的可选路径,但目前跨过沟壑之后,撤退的路径会经过敌方堡垒区,极有可能一去不复还,从敌方的角度而言飞坡的风险增大了不少。但同时加高的梯形高地也为到达敌方半场的机器人提供了更好的掩体。如果是快结束了想放手一搏可以考虑考虑。
                \item 无论是爬坡上梯形高地还是绕道己方公路区前往中央高地都势必会消耗不少能量,因此英雄应精打细算,增大补给间隔,并在比赛进行到下半场时择机在补给区“充电”。
                \item 对于底盘能量20000J的分配,前期应集中在占领梯形高地,中期寻找不同的部署点攻击对方基地,后期则需退回基地区拦截敌方步兵和英雄.
            \end{itemize}

        \paragraph{哨兵规则分析}

            \setlist[itemize]{label=\raisebox{-1.2ex}{\scalebox{3}{$\textbullet$}}}
    
            \begin{itemize}
                \item 哨兵在这个赛季的机制改动相当大,主要的改动在于哨兵取消了与前哨站挂钩的无敌机制、哨兵与比赛胜负的判定机制以及哨兵巡逻区的取消。此外,哨兵的初始发弹量也从400发削减到了300发,而且虽然底盘功率上限未做改动,但是哨兵和步兵、英雄一样适用于全局20000J底盘总能量的约束。这些改动无疑削弱了哨兵的单兵作战能力,也进一步提高了哨兵对完善的导航避障、自身定位、全局规划、灵活决策、资源调度和与其他机器人打配合的需求。
                \item 哨兵在这个赛季的战场上也依旧存在着不小的优势:首先,这个赛季并未对哨兵的枪口热量上限、每秒冷却做改动,这保证了哨兵在可发弹量充足的情况下输出能力依旧恐怖,而且,每一分钟哨兵都可以通过导航到补给区补血补弹或者远程兑弹的方式补充发弹量,这些机制平衡了哨兵初始发弹量削减的影响,在导航完善和工程获取经济能力强的情况下哨兵的发弹量甚至可能只取决于预装的弹丸有多少;其次,哨兵虽然取消了与前哨站挂钩的无敌状态,但是初始数据400血量100功率上限(步兵满级数据)保证了哨兵前期的性能和生存能力在线,配合恐怖的火力,在前期哨兵在面对对面的地面输出单位是具有绝对的压制力的,而且哨兵可以远程兑换血量以及它的复活机制并未改变,这意味着原本弹尽粮绝、性命垂危的哨兵可能突然以一个较好的状态重返赛场,对敌方形成巨大的威胁;况且,这个赛季新增了哨兵可以占领的堡垒增益点,当堡垒增益点开启,哨兵占领之后依旧可以获得无敌、额外发弹量和额外的枪口冷却增益,这一改动大大增强了哨兵的防守能力。
                \item 这个赛季的机制变动使哨兵更加全能,不论是在进攻端还是在防守端都有着独特的优势。显然,本赛季的改动无疑是令哨兵机器人的技术栈更加接近于全自动机器人的本质。没有一个完善的决策体系的话,在全局资源的规划和调动上会出现不可忽视的问题;巡逻区的取消无疑是对哨兵如何及时的规划进攻和回防路线将会是导航和定位上的挑战。
            \end{itemize}

        \paragraph{哨兵策略战术}

            \setlist[itemize]{label=\raisebox{-1.2ex}{\scalebox{3}{$\textbullet$}}}
        
                \begin{itemize}
                    \item 哨兵的前期作战能力是所有地面机器人中最强的,在前期利用高性能迅速占领中央高地上的增益点的战术价值很高,不论是前压前哨站还是阻击对面攻势都会令对手很头疼。
                    \item 哨兵可以在能量机关激活的时候击打能量机关获得额外增益。
                    \item 哨兵能远程补弹,可以在自身增益足够高或敌方状态差时乘胜追击,或是在被动防御时 尝试突围、扭转局势。
                    \item 哨兵可以通过雷达通讯接口得到鸟瞰全局的数据,通过自主决策可以选择进攻、防守、配合其他机器人或者规避围剿。
                    \item 哨兵可以占据中央高地,对靠近高地的机器人进行阻击和对取矿的工程机器人进行拦截。
                    \item 哨兵能接收云台手发送指令,在必要的时候可以解决一些意料之外、情况紧急的情况。
                \end{itemize}

        \paragraph{工程规则分析}

            \setlist[itemize]{label=\raisebox{-1.2ex}{\scalebox{3}{$\textbullet$}}}
    
            \begin{itemize}
                \item 小资源岛变更:小资源岛依旧紧贴中央高地护栏外侧,但银矿由原来的半嵌入式改变为完全嵌入,从原来可以通过夹爪直接取出改变成要求工程机器人从上面将其拔出,对工程机器人的机构要求更加灵活。
                \item 前往资源岛的路径变更:原来前往大资源岛的路径可以通过前哨站然后直接登上大资源岛或者直接钻隧道,而25赛季登上大资源岛的路径更加灵活,在基地正方向新增一个高300mm的二级台阶,第一级是高为200mm的小资源岛,所以工程机器人可以设计一个灵活的机构来完成登岛直接抵达大资源岛,而且现在是公路区连接着中央高地,所以工程机器人也可以选择绕远路先通过公路区再上岛,而上公路区又有选择,可以通过上200mm高的台阶进入公路区,也可以通过爬10°坡上公路区进而前往大资源岛,所以工程机器人的构型可以有更多选择特别是在底盘上。
                \item 经济体制改变:删除了五级难度的兑矿等级,等比例增加了剩余难度等q级的金币数量,且比例提高了随着兑矿次数增多时最低可选择兑矿难度的金币倍率,且新增选择四级难度时,工程机器人需要在15秒的时间限制内,连续兑换2块矿石,若在时限内兑换成功,2块矿石的所得金币将依次连续结算,否则将不会获得任何金币的机制,这需要工程机器人需要有非常灵活的兑矿机构,且操作手需要对操作更加熟练。 
            \end{itemize}

        \paragraph{工程策略战术}

            \setlist[itemize]{label=\raisebox{-1.2ex}{\scalebox{3}{$\textbullet$}}}
    
            \begin{itemize}
                \item 工程机器人定位:根据 25赛季规则的改动,我们认为工程机器人在25赛季的定位更加 偏向于通过取矿-兑矿换取金币达到为团队提供经济的一个角色,为其他车组提供更好的 配置条件,为多样化的战术选择提供经济支持。
                \item 取矿机构:六轴机械臂+吸盘。考虑到维修成本和迭代的选择上,使用板材替代上届的机加工方案。同时考虑到6r型机械臂Link1和Link2电机的承重,决定在这两个关节中引入弹簧的重力补偿,去避免电机的发热,延长使用寿命。
                \item 底盘选择:月球车底盘。由于场地的变更,工程机器人可以选择跨台阶的方式直接登上大资源岛,为了更快的争夺金矿来提高团队经济,我们选择可以适应复杂地形的月球车底盘来实现跨台阶的目标,在轮组的选择上我们将采用麦克纳姆轮来实现底盘的转向。 抬升兼存矿机构:剪式抬升架+一键2/3银矿.
                \item 抬升架的选择:剪式抬升。由于六轴机械臂的操控,我们采用剪式抬升架来抬升,并且在上面安装取银矿机构,使得可以取2/3个矿,由规则手册中场地部分可知,每两个银矿的中心距离为270mm,一键3矿则需要540mm,是可以尝试并且完成的(就怕后面又改规则),同时银矿一开始距离地面200mm,完全抽出来则需要400mm,所以会做一个180度的翻转使其更加稳定并且为后期存矿做足准备。
                \item 自定义控制器:带有六个角传感器的大臂缩小机械臂,采用大臂构型缩小的自定义控制器,有利用操作手更加直观的操作机械臂的运动,并同时减少解算需求,并考虑在自定义控制器中融入移动、吸取等快捷功能,以提升工程在存取矿的能力。
                \item 上位机实现:上位机采用ros2和movelt,将实时获取大臂的六个角度并跟踪机械臂的构型,在需要存取矿时,上位机将实现一次性取矿的路径规划。
                \item 下位机:优化各部分通讯,尝试在下位机实现机械臂正逆解算。
            \end{itemize}

        \paragraph{飞镖系统规则分析}

            \setlist[itemize]{label=\raisebox{-1.2ex}{\scalebox{3}{$\textbullet$}}}
    
            \begin{itemize}
                \item 飞镖命中收益改变:25赛季飞镖命中的基地固定目标和随机固定目标的收益下降。具体表现为命中基地固定目标、随机固定目标为步兵和英雄机器人下降到200和600、随机固定目标命中扣除地方机器人血量由25\%改为10\%、命中固定目标由1000下调至625,命中随机固定目标由1200下调至2500。25赛季新增的随机移动目标,命中难度更大,但收益比24赛季的随机固定目标更高。随机移动目标命中提供2500点巨额经验,同时对方操作收界面遮挡15秒,且对方全部存活的地面机器人立即受到相当于各自当前上限血量25\%的攻击伤害、对方基地护甲立即展开。随机固定目标和随机移动目标的收益下降、随机移动目标收益巨大,使我们这赛季击打的目标确定在随机固定目标甚至随机移动目标上。
                \item 场地变化:对比24赛季,飞镖发射站的结构有所变化,三分钟准备阶段内,飞镖发射站闸门将保持闭合状态,飞镖系统需从飞镖发射站的后方放入。这意味着在准备阶段无法进行人工标定及预瞄,飞镖系统需要更有效的利用开启闸门后的时间,对飞镖系统纯视觉瞄准的速度以及精度都有一定的要求。25赛季飞镖发射站闸门的开启次数和时间节点从原本的开始比赛30秒后两次开启机会,变为了开始比赛30秒后一次和4分钟后一次且未使用的机会可以累加。这一改动使得前期无法集中使用飞镖,对于需要在前期使用飞镖获得更大战果的情况,飞镖的准度和命中的稳定性显得更为重要。对比24赛季,从场地尺寸上来看前哨站与飞镖发射站的相对位置变化较为明显,夹角从24赛季的6.6°变为了2.1°。除此之外,飞镖发射站和基地沿与战场长边平行方向的距离减少了640mm。
            \end{itemize}

        \paragraph{飞镖系统策略战术}

            \setlist[itemize]{label=\raisebox{-1.2ex}{\scalebox{3}{$\textbullet$}}}
    
            \begin{itemize}
                \item 飞镖定位:根据25赛季规则的改动,飞镖在25赛季的定位偏向于战略进攻单位,优势时与地面配合进攻,在局势焦灼时通过飞镖致盲效果实现局势逆转。
                \item 准备阶段:发射台是从后方开放放入飞镖系统,在赛前准备时间3分钟内不能人工标定前哨站与基地,因此决定在裁判系统自检的五秒内通过飞镖架的摄像头识别引导灯,标定前哨站并记录敌方基地方位。提前锁定目标方向。缩短飞镖发射时间。
                \item 目标选择:在基地随机移动靶和随机固定靶带来的巨大收益,本赛季飞镖击打的目标需往随机移动靶和随机固定靶靠拢,由此需要研发制导镖架和制导镖体的预研发。
                \item 发射结构:由于摩擦轮发射的飞镖发射架具有较多的不稳定因素,本赛季决定采用拉簧弹射的方式,解决了摩擦轮旋转转速不一致可能导致飞镖不在预估轨道上的问题,同时,拉簧可以为镖体提供稳定的发射能量。
                \item 战术选择:飞镖系统“致盲”的时间改动,增加了首次命中的的“致盲”效果的时间,这提高飞镖的命中的重要性,充分利用飞镖的致盲效果,配合地面单位对敌方发起进攻,对敌方进行火力压制,消灭敌方地面单位,为己方建立优势。
            \end{itemize}

        \paragraph{雷达系统规则分析}

            \setlist[itemize]{label=\raisebox{-1.2ex}{\scalebox{3}{$\textbullet$}}}
        
            \begin{itemize}
                \item 雷达可以为队伍提供敌方机器人的位置信息,提供视野给到队伍。
                \item 雷达可以通过裁判系统将位置信息发给己方单位,并且只能接收哨兵的信息
                \item 己方的雷达可识别对方地面机器人的位置,并将该机器人的坐标发送至裁判系统,若精度符合条件则能持续积攒标记进度,当标记进度大于等于100时被标记的对象回获得一个-15\%的防御增益。
                \item 当雷达每累计使对方机器人易伤 1 分钟(同时有多台机器人易伤时,时间不累加),将会获得 1 次触发“双倍易伤”的机会,雷达可以通过裁判系统主动发送命令消耗机会,并使当前所有正处于易伤状态的负防御增益数值由-15\%变为-30\%,持续 30 秒。每局比赛中,雷达至多可以触发 2 次“双倍易伤”。
            \end{itemize}

        \paragraph{雷达系统策略战术}

            \setlist[itemize]{label=\raisebox{-1.2ex}{\scalebox{3}{$\textbullet$}}}
        
            \begin{itemize}
                \item 精准识别出敌方机器人的位置信息,累计标记进度并发送信息给予队伍,为队伍增加优势。
                \item 当有双倍易伤的机会时,配合队伍需求,触发该效果。
                \item 识别出敌方机器人在某些特殊位置(如我方基地前、飞坡等)时,可以通过裁判系统向己方单位发送预警信息。
                \item 通过裁判系统接收哨兵传来的重定位的信息,更加精确的锁定部分目标。
                \item 协助英雄对地方基地进行吊射。
            \end{itemize}

        \paragraph{空中机器人规则分析}

            \setlist[itemize]{label=\raisebox{-1.2ex}{\scalebox{3}{$\textbullet$}}}
        
            \begin{itemize}
                \item 空中机器人相较于上个赛季改动最大的是再比赛开始即可起飞,起飞后可以通过花钱的方式续费,这无疑是对空中机器人的大增强,首先就是对于哨兵和英雄存在极其强大的压制力,并且对于大部分队伍也能开局清除前哨站。
                \item 空中机器人的限重减少,这一改动直接影响了空中机器人的设计,并且取消了中途补弹,如果想发挥空中机器人的最大优势,首先便是需要将空中机器人的整体做轻,并且是弹舱的设计要能容纳的下最少1500发弹,对于很多队伍来说相当于只能重新设计一台无人机。
            \end{itemize}

        \paragraph{空中机器人策略战术}

            \setlist[itemize]{label=\raisebox{-1.2ex}{\scalebox{3}{$\textbullet$}}}
    
            \begin{itemize}
                \item 前期刚开局,压制英雄前期的吊射,英雄的改动使得其可以随处吊射,对于前哨站来说也是非常有威胁,如果不针对英雄便会与上赛季一样,英雄轻易吊射,并且新增的43°高坡使得步兵更难第一时间针对对方英雄,因此空中机器人前期可以采取压制哨兵与英雄的打法,如果对方并没有很强势的英雄与哨兵则可以优先打击前哨站。
                \item 在中期时,空中机器人主要起到帮助己推进或是防守的角色,在对方英雄,步兵处于优势位置时,可以优先针对,并且对于信息的获取也是一大关键,空中机器人也能开能量机关为团队进一步提供帮助。
                \item 后期可能会遇见两方的功率消耗殆尽,或者难以攻入对方,亦或者是基地/前哨/哨兵,血量接近此时无人机如果还能有电量留存则可以起飞去消磨对方的血量,拿下关键点。
            \end{itemize}

    \input{section/3.2_projectPlan}

    \input{section/3.2.1_infantry}

    \subsubsection{英雄机器人}
    
    \paragraph{功能需求分析}

    \LTXtable{\textwidth}{Hero/2.3_functionalRequirement_analysis.tex}
    
    \paragraph{改进方向}

    \LTXtable{\textwidth}{Hero/2.3_improvementDirection.tex}

    \paragraph{研发进度安排}

    \LTXtable{\textwidth}{Hero/2.3_developmentSchedule.tex}

    \newpage

    \paragraph{项目组人员分配}

    \LTXtable{\textwidth}{Hero/2.3_personalAssignment.tex}
    

    \input{section/3.2.3_engineer}

    \subsubsection{哨兵机器人}
    
    \paragraph{功能需求分析}

    \LTXtable{\textwidth}{Sentry/2.3_functionalRequirement_analysis.tex}

    \newpage
    
    \paragraph{改进方向}

    \LTXtable{\textwidth}{Sentry/2.3_improvementDirection.tex}

    \paragraph{研发进度安排}

    \LTXtable{\textwidth}{Sentry/2.3_developmentSchedule.tex}

    \paragraph{项目组人员分配}

    \LTXtable{\textwidth}{Sentry/2.3_personalAssignment.tex}
    

    \input{section/3.2.5_drone}

    \subsubsection{飞镖系统}
    
    \paragraph{功能需求分析}

    \LTXtable{\textwidth}{Dart/2.3_functionalRequirement_analysis.tex}
    
    \paragraph{改进方向}

    \LTXtable{\textwidth}{Dart/2.3_improvementDirection.tex}

    \newpage

    \paragraph{研发进度安排}

    \LTXtable{\textwidth}{Dart/2.3_developmentSchedule.tex}

    \paragraph{项目组人员分配}

    \LTXtable{\textwidth}{Dart/2.3_personalAssignment.tex}
    

    \subsubsection{雷达系统}
    
    \paragraph{功能需求分析}

    \LTXtable{\textwidth}{Radar/2.3_functionalRequirement_analysis.tex}
    
    \paragraph{改进方向}

    \LTXtable{\textwidth}{Radar/2.3_improvementDirection.tex}

    \newpage

    \paragraph{研发进度安排}

    \LTXtable{\textwidth}{Radar/2.3_developmentSchedule.tex}

    \newpage

    \paragraph{项目组人员分配}

    \LTXtable{\textwidth}{Radar/2.3_personalAssignment.tex}
    

    \subsubsection{人机交互}

    \paragraph{自定义UI}

        新赛季针对大地图界面的UI设计问题。我们规划,保证基本的等级、电池电量等 UI 图形和数据信息正常显示外,增加更多的机器人状态信息实时反馈,并计划从图形化的角度设计新UI,如用滑块实时滑动来反映数据的变化、颜色变化反映开关状态、圆弧转动与否反映底盘小陀螺是否开启、圆弧缺口方向反映底盘相对于云台的方向等,为操作手提供重要决策信息的同时,增加UI的可读性。
    
    \paragraph{自定义控制器}

        针对工程机械臂关节数量多,控制难度大,在上赛季尝试使用和调研之后,新赛季我们选择使用大机械臂相同构型的机械臂结构,利用2个2006电机和达妙4个4310电机制作了一板缩小的机械臂,来实时获取机械臂的角度传入大机械臂。另外对机械臂一键取存矿,底盘运动和兑矿等做设计。由于使用了达妙电机,可以在自定义控制器中添加力反馈以及在平时中作为代码验证,在新人培训等也能发挥作用。

\newpage
    

    \input{section/3.3_technologyReserve_plan}

    \subsubsection{通用技术储备}
    
    \paragraph{超级电容}
    
    \paragraph{功率控制算法}

    \subsubsection{特定兵种技术储备}


    \subsubsection{机械组}

    \paragraph{现有技术优化}

        \setlist[itemize]{label=\raisebox{-1.2ex}{\scalebox{3}{$\textbullet$}}}
    
        \begin{itemize}
            \item 技术点:步兵枪管上下v型轴承定心与单发限位。
            \item 优化目标:现步兵枪管基本为上下v型轴承定心且充当单发限位,可以做到不卡弹,但弹道散布一般且有双发问题,调整轴承位置中心及选型以达到不双发且不漏弹。
            \item 技术点:轮系电机内嵌式设计。
            \item 优化目标:步兵与英雄机器人如使用常见的电机直接与麦克纳姆轮连接,整个机器人自身重量的所有冲击都会作用在电机轴上,出现“外八”的情况,而内嵌式轮组将冲击分散给电机套筒,从而解决了这个问题。继续沿用电机内嵌式结构,搭配纵臂悬挂来获得比较好的避震效果。
            \item 技术点:半下供弹云台和下供云台。
            \item 优化目标:为解决弹丸数量云台重心影响大的问题,设计出半下供弹云台和下供云台,将弹舱与pitch分离优化拨盘,降低拨盘卡弹率;优化拨盘以及弹链位置以及尺寸,减小云台转动半径。
            \item 技术点:渐开线拨盘。
            \item 优化目标:为解决卡弹问题,设计出渐开线式的拨盘和拨爪,契合弹丸运动轨迹,有效减少卡弹的几率。
            \item 技术点:中置弹舱设计。
            \item 优化目标:空中机器人的重心最佳位置在机翼平面,而内嵌式弹仓可以打通上下中心板,使其重心在中间位置。继续对中置弹舱做极限设计,极限利用空中机器人中心板之间的位置,而又能保证其强度。
            \item 技术点:管结构机架。
            \item 优化目标:为减少空中机器人整体重量,采用管结构机架替代层结构,在保证整体刚度和正常飞行的同时有效减少较多重量。
            \item 技术点:同步带涨紧技术。
            \item 优化目标:英雄机器人yaw轴采用自行设计松紧可调节的利用法兰轴承进行张紧的结构对同步带进行涨紧;工程机器人机械臂的同步带采用预留多个孔位用于安装轴承进行涨紧的方式,对同步带进行涨紧,消除同步带传动的虚位。
            \item 技术点:双级摩擦轮发射机构。
            \item 优化目标:利用两对摩擦轮实现双级加速,延长加速道路,使弹速尽量稳定。并且在上下部分增加了两个摩擦轮进行单发限位,并且进行上下精确定心,进一步减小弹道扩散,提高吊射命中率。
            \item 技术点:转臂平衡重力补偿。
            \item 优化目标:应用在步兵和英雄机器人上,可以对云台进行配平(使用齿轮与滑块与拉簧制作的紧凑型重力补偿,将小齿轮固连在云台pitch轴上,在pitch轴转动时,带动另一个大齿轮旋转,而大齿轮上连接着一个连杆,连杆会推动滑块,而滑块上有一根拉簧做连接,从而达到在云台旋转时,拉簧做重力补偿。
            \item 技术点:鹅颈供弹。
            \item 优化目标:顺滑不卡弹的鹅颈供弹,配合上超高弹频的拨盘,可以大大程度增加机器人攻击性。可应用在哨兵上,一定程度减少卡弹情况,并且可以优化云台空间利用率,轻量化设计云台。
            \item 技术点:舵轮轮组。
            \item 优化目标:目前以3508电机和6020电机为驱动制作了两版舵轮底盘供英雄机器人和步兵机器人使用,采用垂直悬挂作为减震方式,以90°夹角的两组滑块滑轨限制轮组的自由度。舵轮轮组的装配方式有待进行测试改善,减小维修难度,对轮组部分做有效保护。
            \item 技术点:剪式升降台。
            \item 优化目标:供工程机器人用来抬升取矿机械臂,通过丝杆来控制高度,稳定性高、承载能力强,本赛季将继续优化,提高其升高高度。
            \item 技术点:自定义控制器。
            \item 优化目标:机械制作大臂缩小模型的小机械臂,带有六个角度传感器,将角度直接传入大臂。争取将大部分功能集合在自定义控制器中且做出舒适且精确的自定义控制器。
            \item 技术点:拉簧式弹射型发射架。
            \item 优化目标:将飞镖机器人的摩擦轮式发射架更改为拉簧式弹射型发射架,利用蓄能能量稳定的优势,减小飞镖体落点散布,增加飞镖发射的稳定性。
        \end{itemize}

    \paragraph{技术发展计划}
        

    \input{section/3.3.4_electricalComponent}

    \subsubsection{视觉组}

    \paragraph{技术储备规划}

        \setlist[itemize]{label=\raisebox{-1.2ex}{\scalebox{3}{$\textbullet$}}}

        \begin{itemize}
            \item 优化识别速度:在原来的代码基础框架下对代码进行优化,实现更高的识别速度,提高自瞄帧率
            \item 能量机关击打:本赛季可以在更多的位置上击打能量机关,自瞄算法需要针对多角度的识别以及瞄准做出调整
            \item 弹道运动轨迹计算:弹道解算模型可以更精确,提升自瞄准度
            \item 针对各兵种的适配优化:自瞄代码在步兵上的准度较低,而在固定云台基座的无人机上效果良好;而将来无人机在空中的工况也会和地面的固定位置射击有不同。也就是说代码需要针对个兵种的云台稳定性以及响应速度做出相应的优化
        \end{itemize}
        
    \paragraph{技术发展计划}

        \setlist[itemize]{label=\raisebox{-1.2ex}{\scalebox{3}{$\textbullet$}}}

        \begin{itemize}
            \item 弹道偏移校正:弹道会因为各种原因偏移原来的预定轨迹,而人工调准又十分繁琐耗时,现在需要一套算法来识别弹道的轨迹和落点来自动修正枪口的射击方向
            \item 场上手动弹道校正:由于场下调整的弹道在场上不一定适用,而弹道自动修正难度较大,需要写一个场上手动校正弹道的程序作为保底方案
            \item 全向感知:继续完成哨兵的全向感知的研发,在条件允许的情况下进行上车调试
            \item 避障与重定位:由于没有很好的初始位姿校准机制,没有精确摆放的哨兵导航路线会向一个方向偏移,现在需要重定位机制来修正哨兵的初始位姿;由于上赛季没有避障机制,本赛季需要添加
            \item 点云聚类:通过雷达点云聚类来识别该点云是否是敌方机器人的图像,可以作为哨兵全向感知的补充
            \item 镖架制导:通过识别前哨站和基地绿灯实现飞镖镖架的自动瞄准
            \item 视觉辅助兑矿:通过视觉算法结算兑矿站的位姿,辅助工程机器人进行矿石兑换
        \end{itemize}

    \input{section/3.3.6_hardwareComponent}
    
    \newpage

    \section{资源可行性分析}

    \subsection{本赛季可用资源概述}
    
        \noindent
        \LTXtable{\textwidth}{table/4_availableResources_overview.tex}
    
    \subsection{资金预算分配规划}

        \noindent
        \LTXtable{\textwidth}{table/4_budgetAllocation_plan.tex}

    \newpage
    
    \section{团队管理}

    \subsection{团队架构}

    \subsubsection{队伍架构概述}
    
        \noindent
        \LTXtable{\textwidth}{table/3_managementStructure.tex}

        \noindent
        \LTXtable{\textwidth}{table/3_memberStructure.tex}

    \subsubsection{团队招募计划}

        \paragraph{}{目标人员分析}

        \paragraph{招新渠道及招新方式分析}

            \subparagraph{机械组招新方式}

                招新流程分为多个阶段,确保选拔到具备全面素质和技能的优秀成员。具体流程如下:

                \setlist[itemize]{label=\raisebox{-1.2ex}{\scalebox{3}{$\textbullet$}}}

                \begin{itemize}
                    \item 首先,有意参加的新同学需提交报名表和个人简历,以展示个人基本信息和兴趣方向;
                    \item 其次,通过报名表和简历初步筛选后,我们将要求新同学自学相关方向,并提交学习计划、 学习内容以及链接的课程资源,例如在SW软件领域的学习教程、RoboMaster论坛链接,或其他关键领域的材料力学课程等。这一步主要考核新同学的自学能力和对所报名方向的热情。
                    \item 之后,选中的新同学将对论坛开源的机器人模型挑选感兴趣的模块,并完成一份分析报告,展示对模型的基本理解和认识。这有助于进一步评估新同学对机器人模型的逻辑思考和分析能力。
                    \item 然后,通过前述步骤的筛选,挑选出第一批同学后,进行SW软件的上机考试和有关于比赛规则及基本材料力学的笔试,以深入考核新同学对SW软件的熟练运用情况和对比赛的认识程度。通过提供零件工程图和装配图,限时三小时的上机和笔试考试,以评估新同学的建模能力和自学能力。 
                    \item 之后在第二轮中,筛选第一轮表现优异的同学,进行多对一的面试。这一轮的面试将聚焦于新同学的生理素质、心态品质,以及对比赛的热情和团队协作意识。 
                    \item 最终,根据整个招新流程的综合表现,确定入选机械组梯队的队员名单。这一过程将确保新队员在技术和团队合作方面都具备出色的素质,为团队的协同工作和未来比赛做好充分准备。
                \end{itemize}

            \subparagraph{电控组招新方式}

                电控组招新分多个阶段:

                \setlist[itemize]{label=\raisebox{-1.2ex}{\scalebox{3}{$\textbullet$}}}

                \begin{itemize}
                    \item 首先进行笔试,着重考察C语言基础和编程能力以及一部分的嵌入式知识,筛选第一批基础达标的人通过笔试。
                    \item 2.通过笔试的人会进行面试,面试会提问C语言、嵌入式的基础知识和对比赛的了解程度,但是着重考察对团队对开发对比赛的态度,通过面试的人员就成为悍匠战队的预备队员。
                    \item 3.预备队员会进入培训阶段。培训阶段会培训嵌入式基础和电控算法基础,然后根据培训进度发布阶段性任务;培训结束之后会对阶段性任务的完成度评分,会对评分不合格人进行筛选。
                    \item 通过培训的预备队员将会和其他组别的预备队员参与校内赛,校内赛进行期间将会对预备队员发布周期任务并对周期任务进行评分,筛选评分过低和态度不端正的人员。
                    \item 校内赛过后,留下的预备队员就会(以个人意愿优先)被分配到各个兵种进行后续学习,成为各兵种的预备开发人员,此时会发布新人的终期考核任务用于考察他们的实际开发能力。此后便协助各兵种的主力开发测试及维护,并且我们会筛选一部分出色的人员加入梯队,参与正式比赛。
                \end{itemize}

            \subparagraph{视觉组招新方式}

                \setlist[itemize]{label=\raisebox{-1.2ex}{\scalebox{3}{$\textbullet$}}}

                \begin{itemize}
                    \item 笔试:通过基础规则常识考核、c++考核等阶段性考核的方式逐层选拔,最终挑选一批能力优秀且有潜力的新人。
                    \item 最终会让新人写一个装甲板或者能量机关的识别程序,通过最终的考核即成为梯队队员。
                    \item 面试:面试着重考察临场反应能力和平时积累,通过面试的新人通常具有比较高的水准。
                \end{itemize}

            \subparagraph{硬件组招新方式}

                针对不同的技术需求,本赛季硬件组的招新考核部分与电控组分离。

                \setlist[itemize]{label=\raisebox{-1.2ex}{\scalebox{3}{$\textbullet$}}}

                \begin{itemize}
                    \item 参与者首先与电控组一同进行第一轮的编程题机考。
                    \item 而后经历为期半周的硬件通识考核.
                    \item 最后进行线下面试,三次考核分数经由不同权重组成最终得分.
                    \item 过者由主力队员组织培训后,成为梯队队员参与到日常维护与技术研发。
                \end{itemize}

            \subparagraph{运营组招新方式}

                随着深圳技术大学悍匠战队的不断发展与壮大,为了进一步提升战队运营效率,增强团队创新能力与竞争力,运营组在每学期初始都会开展新一轮招新活动。运营组招新旨在吸引一批具有创新思维、良好的沟通能力、扎实的专业技能及强烈责任感的大一新生加入悍匠战队,共同推动战队稳健前进。
                运营组主要从三个方向招收大一新生:

                \setlist[itemize]{label=\raisebox{-1.2ex}{\scalebox{3}{$\textbullet$}}}

                \begin{itemize}
                    \item 宣传部:负责线上线下活动的策划与微信公众号等平台的内容策划、编辑与发布,提高悍匠战队活跃度。
                    \item 财务部:对战队资金流动等运营数据进行监控、分析,为战队决策提供数据支持。
                    \item 设计部:负责周边产品的设计。
                    \item 在明确招新目标、时间安排等后,运营组利用海报、宣传视频等,通过学校官网、微信公众号、QQ群等社交平台发布招新信息,扩大影响力。同时运营组成员会在食堂、图书馆、教学楼等人流密集区域张贴招新海报,设置咨询摊位,面对面解答疑问。
                    \item 新生投递报名资料后,运营组将对申请者的基本信息、专业技能、过往经历等进行初步筛选,同时布置考核作业与提交期限。在收到申请者考核作业后,运营组将通过微信群通知入选者参加面试,明确面试时间、地点及注意事项。
                \end{itemize}

        \paragraph{招募时间安排}

    \subsubsection{团队培训计划}

        \paragraph{机械组}

            \subparagraph{培训形式}

                机械组对新人的培训主要采取一对一导师制,按新人意愿优先原则分配新人到各个项目组中,在设计时进行讲解,也会不定时分配小任务给新人完成,让新人参与到整个项目中;不定期开展机械组内会议,也是作为机械组新人培训会,激发新人的自主学习能力,师傅领进门,修行在个人,新人们会到论坛上爬取开源,或者主动询问老队员,最后将调研后的知识写成感悟提交给负责人检查,这样做提升了新人寻找资料来完成任务的能力;举行新人校内赛,让各车组新人初次体验如何通过团队合作、相互沟通完成车辆的设计的装配,进一步激发新人对比赛的热情,同时也提高他们的软件熟练度和设计思维及装配能力。
  
            \subparagraph{培训安排}

                \LTXtable{\textwidth}{Group/3.7.1.2_mechanicalTraining_arrangement.tex}

        \paragraph{电控组}

            \subparagraph{培训形式}

                \setlist[itemize]{label=\raisebox{-1.2ex}{\scalebox{3}{$\textbullet$}}}

                \begin{itemize}
                    \item 电控组对新人的培训方式主要采取导师制,授课加任务验收的方式。
                    \item 在这个培训体系中,一个导师将会辅导一组(3到4名)新人,负责引导他们解决在学习过程中遇到的困难和疑惑,导师一般由老队员担任,旨在通过他们丰富的经验和知识帮助新人更快的入门;
                    \item 导师还担任着监督者的任务,监督新人及时参与培训并按时完成任务,并观察他们的态度。
                    \item 此外,定期授课的方式将比赛中所运用的各项技术拆分细化后融合进若干课程中,经过课程讲解基础知识,再布置验收任务,任务难度由易至难呈阶梯式上升。
                    \item 除了定期授课,定期布置任务并进行任务验收是必不可少的一环。导师会对新人完成的任务进行评估,提供具体的反馈,并根 据需要调整培训计划。这种及时的反馈机制有助于新人快速纠正错误,不断改进自己的表现。当所有培训课程结束后,会筛选一部分态度不端正和培训任务完成度过低的新人。
                    \item 之后新人将会参与校内赛体验真实的开发和备赛过程,校内赛期间会周期性布置校内赛的任务并对任务评分以及对新人的态度做评估,筛选不过关的人员。
                    \item 校内赛过后新人(以个人意愿优先)将会被分配到各个兵种进行后续学习,此时将会发布终期考核任务考察他们的实际开发能力。
                    \item 此后便协助各兵种的主力开发测试及维护机器人,逐渐熟悉开发流程和积累调试和开发经验,最终达到可以独自承担机器人控制部分的所有任务的目标。
                \end{itemize}

            \subparagraph{培训安排}

                \LTXtable{\textwidth}{Group/3.7.2.2_electricalTraining_arrangement.tex}

        \paragraph{视觉组}

            \subparagraph{培训形式}

                视觉组技术培训旨在提升学员在 C++、OpenCV 和 Linux 环境下的开发能力,通过多方位的培训形式,使学员能够在实际项目中熟练运用所学知识。将以授课、考核、作业、项目的形式进行培训。\par
                在第一轮选拔中,我们将进行 C++考核,通过对新生的笔试和面试评估,选取表现突出的新生作为预备队员。这些预备队员将进入下一轮的培训阶段,由主力队员亲自进行指导和培训。\par
                在培训过程中,我们将设定多个阶段性考核,以评估预备队员的学习进度和技能掌握情况。及时发现培训中出现的问题、提供针对性的帮助,并确保每位预备队员都能够逐步成长和适应团队的需求。 \par
                经过全面的培训和阶段性考核后,预备队员将参与到最终考核中。通过这一环节的评估,我们将选定表现出色的成员,正式纳入梯队,成为视觉开发团队的一员。\par
                
                \setlist[itemize]{label=\raisebox{-1.2ex}{\scalebox{3}{$\textbullet$}}}

                \begin{itemize}
                    \item 授课:每周定期授课,介绍新知识点和技术。提供示例代码和案例,帮助学员理解和应用所学内容。
                    \item 答疑:新人可以线下来实验室学习并互相交流或者找主力队员答疑。
                    \item 考核:定期小测验,检查学员对于每个知识点的理解程度。中期项目考核,要求学员完成一个小型项目,运用所学技能。
                    \item 作业:每节课后布置相应的作业,巩固学员的学习。作业涵盖编程练习、问题解决和小项目任务。
                    \item 实际项目:通过实际项目锻炼学员的实际开发能力。项目中模拟团队协作和开发流程,培养团队协同能力。
                \end{itemize}

            \subparagraph{培训安排}

                \LTXtable{\textwidth}{Group/3.7.3.2_visualTraining_arrangement.tex}

        \paragraph{硬件组}

            \subparagraph{培训形式}

                硬件组对新人的培训方式主要采取阶段性任务考核,分层渐进式完成指定目标。吸取上赛季培训周期过长的问题,本赛季在招新阶段融入了通识考核,新人业已完成基础知识的扫盲。同时,我们新开展第一届校内赛。校内赛旨在培养新成员的团队合作精神,面对技术挑战时的问题解决能力。在校内赛中,明确团队分工,让每个成员都能发挥作用。老成员负责定期验收本阶段任务,并进行评估,提供具体的反馈。由此新成员能够在短期经历完整的备赛体验,对悍匠团队和Robomaster赛事的认知亦更上一层。

            \subparagraph{培训安排}

                \LTXtable{\textwidth}{Group/3.7.4.2_hardwareTraining_arrangement.tex}

        \paragraph{运营组}

            \subparagraph{培训形式}

                运营组培训形式为各部门负责人牵头,向新成员讲解工作内容与运用到的相关软件,并每周进行相应课时的技能培训,在为期一月的培训结束后对新成员开展考核。在考核开始前,培训负责人应清晰地向新生解释考核的内容、评分标准以及注意事项。培训负责人应保持耐心,给予他们充分的展示机会,并在必要时提供适当的指导和鼓励。培训负责人要及时给予新生反馈,既包括他们表现优异的地方,也包括需要改进的地方。对于不足之处,尽量提出具体的、建设性的建议,帮助他们明确改进的方向和方法。在考核结果之外,培训负责人应更关注整个考核过程中新生的学习和成长,鼓励他们从每次尝试中汲取经验,不断完善自己,而不是仅仅追求最终的考核成绩。

            \subparagraph{培训安排}

                %\LTXtable{\textwidth}{Group/3.7.5.2_operatorTraining_arragement.tex}

                \setlist[itemize]{label=\raisebox{-1.2ex}{\scalebox{3}{$\textbullet$}}}

                \begin{itemize}
                    \item 鉴于大学生普遍缺乏财务实践经验,运营组财务部将展开“一对一”示范,以老队员带动新队员的形式,开展财务基础知识教学,采用案例分析、模拟演练等方式,使新成员能够初步掌握财务管理的基本技能。
                    \item 同时,针对设计岗位的新成员,策划部将开设Adobe系列设计软件(如Photoshop)教学,通过线上教程、实操练习及小项目设计,快速提升其设计技能。
                    \item 除此之外,为了积累宣传部成员的实战经验,将由宣传部新成员组织运营组内部的小型项目,如校园宣传活动、团队建设等,让成员在实战中积累经验,提升项目管理、团队协作及问题解决能力。
                \end{itemize}

    \subsubsection{团队架构梳理}

        \begin{figure}[H]
                    \centering
                    \includegraphics[height=0.9\textwidth]{figure/teamStructure_review.png}
                    \hspace{0.5em}
                    \caption{\textbf{\zihao{-4}\textbf{组织架构示意图}}}
                    \label{fig:teamStructure_review}
        \end{figure}

    \subsection{规章制度建设}

    \subsubsection{规章制度建设重点分析}

        robomaster这个比赛相比其他比赛,投入时间长,压力大,而且虽然对成员的锻炼很大但是投入产出比相比于其他比赛还是稍显逊色,所以成员个人对于比赛的投入程度主要还是依靠热爱和对技术的不懈追求,并且优秀的人才都是自驱的,所以队伍的规章制度较为开放轻松,但是这主要面向自觉的队员,对于不自觉且态度不端正的队员,实行相应的惩罚,如果屡教不改,则在不影响队伍进度的前提下,实行冷处理,不派发核心任务,所以规章制度建设的重点就在于对于自觉者来说可以很相对自如,而对于不自觉者而言可能倍感压力。\par
        且团队毕竟是比赛形式和科技社团性质,按理来说不应当过度干涉队员的个人生活,所以规章制度上最严重的处罚也仅仅是离队,并不能够解决实际进度遇到的问题。所以规章制度对于不自觉的成员更倾向于更换主力开发、回收工位、下放梯队等措施,保留该成员在战队中的知情权和参与开发的权限,但不给予核心任务、降低团队对该成员的期望,制度相较灵活,进退留有余量。\par

    \subsubsection{规章制度落地方案分析}

        \setlist[itemize]{label=\raisebox{-1.2ex}{\scalebox{3}{$\textbullet$}}}

        \begin{itemize}
            \item 实行开放的打卡考勤制度,对于进度可查,且经常出现在实验室者,不做硬性要求,对于进度模糊,而时长不出现在实验室者,考勤便可以作为考察依据;
            \item 对于物资管理,由于低级错误或者恶意毁坏高价值物资者,将根据该物品市场价的百分之二十赔偿,第二次毁坏则增加百分之二十,以此类推,实行至今,此事件出现几率减少近百分之八十,效果显著;
            \item 制定电机出入库记录表,有效避免电机随意放在地上、甚至电机消失不见的现象;
            \item 对于进度,制定阶段性目标,负责人参与主力开发任务,日常监督进度,阶段性考核,动态调整项目进度,效果良好;
            \item 队长、项管长期在实验室,和参赛队员坐在一起,能够方便、有效地视察进度,效果显著;
            \item 制定“警钟长鸣”文档,记录各种物件的损坏,警示所有队员不要犯同样的错误,并记录某些重要部件的使用寿命;
            \item 继续推动进度公示制度,给每一个车组建立进度规划表格,负责人长期维护表格,组员自觉新增项目和更新进度,但组员总是忘记新增项目和更新进度,制度推进效果差;
            \item 优化财务流程,由技术组组长分摊一部分财务压力,负责财务的队员只需要汇总发票、对接指导老师和学院方;
            \item 完善实验室卫生值日表,将实验室卫生责任落实到个人,实践效果可以应付上级的检查,日常卫生仍然需要提醒,值日人通常比较乐意整理实验室;
            \item 完善对外宣传值日表,将成员分组轮流接待各种宣传需要,扩大战队和赛事文化的知名度,实践效果非常有效;
            \item 完善会议记录,战队大会、技术组或运营组组会、车组组会,留下会议记录,写各类文档时能派上用场;
            \item 大体使用飞书平台互相交流、进度管理,将成员飞书活跃度作为衡量其贡献度的维度之一;
            \item 整顿实验室游戏现象,在未完成个人任务时在实验室玩游戏,接到举报后当即惩罚该成员请在场所有人喝奶茶,有效减少游戏现象;
            \item 后期紧张备赛阶段日更新各车组每日进度。
            \item 管理层内部共同维护一个待办清单表格,防止个别成员遗忘进度;
            \item 管理层、负责人和指导老师每两周开一次例会,可以及时向指导老师反馈当前进度和面临的问题。
        \end{itemize}

    \subsubsection{规章制度落地闭环分析}

        \setlist[itemize]{label=\raisebox{-1.2ex}{\scalebox{3}{$\textbullet$}}}

        \begin{itemize}
            \item 某成员长时间缺勤不打卡,队长、项管进行约谈,约谈后出勤情况有明显改善;
            \item 打卡制度后队员到勤率有明显上升;
            \item 物资管理办法施行后,乱丢工具、耗材的现象有明显改善,取用电机、开发板等规范化;
            \item 负责人、队长、项管和其他队员物理距离的拉近能够显著提升进度监察的效率;
            \item “警钟长鸣”文档后可以明晰哪些问题是备赛过程中常常遇见的,数据积累中;
            \item 进度公示制度推行困难,队员难以养成自觉新增项目和更新进度的协管;
            \item 优化财务流程后,财务负责人有更多的时间和精力确保报账准确无误;
            \item 实验室卫生值日表可以督促值日人整理实验室,但常常需要因为发现某些地方脏乱后经人提醒才来收拾;
            \item 对外宣传值日表将宣传责任明确地分配到每一个组,不会出现没有人愿意去或一群人抢着去的情况,效果良好;
            \item 例会通常不会写会议记录,组会、大会会写会议记录,提供给以后作为进度参考依据;
            \item 使用飞书平台管理,功能齐全、后台能够收集很多队员的日常数据,个别队员仍然无法及时关注到飞书消息;
            \item 整顿实验室游戏现象后明显减少了游戏现象;
            \item 备赛日更新进度还未开始进行;
            \item 管理层待办清单通常由队长一人新增任务,提醒其他管理层成员任务截止日期;
            \item 指导老师例会确保了财务安全、进度清晰。
        \end{itemize}

    \subsection{文化建设}

    \subsubsection{战队文化建设}

        \noindent
        \LTXtable{\textwidth}{table/5.3.1_teamCulture_construction.tex}

    \subsubsection{赛事文化渗透}

        
    
    \newpage
    
    \section{宣传及商业计划}


    \input{section/6.1_publicityBusiness_plan}

    \subsection{商业计划}

    \subsubsection{战队招商客户规划}
    
    \subsubsection{战队招商资源优势及亮点}
    
    \subsubsection{战队招商目标规划}

    
    \newpage

    % 引入封底
    % 封底
    
\includepdf{section/cover2.pdf}

\end{document}
%%% 文档内容-END
%%% 文档内容是正文部分,我们只需编辑 ./section/ 目录下的文档即可