%%% 本文档作为功能演示

%%% 表格演示-BEGIN
\begin{longtable}{ p{2cm} | p{7.8cm} | p{6cm} |}

    \hline

    %%% 此处标识以上为表格脚,即在每页的表格底部重复显示的内容(一条横线)
    \endfoot
    
    %%% 此处标识该行颜色为指定定义的颜色,tabhdcolor 在 ./introduction.tex 中定义,颜色为hsb(0, 0, 0.82353),即82度灰
    \rowcolor{tabhdcolor}

        %%% 第1行的表格信息
        \begin{center}
            功能
        \end{center} &
        \begin{center}
            需求分析
        \end{center} &
        \begin{center}
            设计思路
        \end{center} \\

    %%% 此处完整分割一行
    \hline

    %%% 此处标识以上为表格头,即在每页的表格头部重复显示的内容(一堆header)
    \endhead

        %%% 第2行的表格信息
        \begin{center}
            重定位
        \end{center} &
        \begin{center}
            修正哨兵的初始位姿
        \end{center} &
        \begin{center}
            icp点云配准计算观测点云和先验点云的变换矩阵,返回给tf变换改变里程计坐标和地图坐标的位置关系
        \end{center} \\
        
    \hline

        \begin{center}
            避障
        \end{center} &
        \begin{center}
            避免哨兵导航时与其他物体碰撞
        \end{center} &
        \begin{center}
            测试teb导航包是否可用,参考其进行避障程序编写
        \end{center} \\
        
    \hline
    
        \begin{center}
            全向感知 
        \end{center} \cellcolor{gndcolor} &
        \begin{center}
            360°全方位检测敌方单位
        \end{center} \cellcolor{gndcolor} &
        \begin{center}
            沿用装甲板识别程序,通过四个全向摄像机的位姿关系得到敌人的坐标
        \end{center} \cellcolor{gndcolor} \\

    %%% 此处分割一行的第2至3个单元格
    \hline
    %\cline{2-3}
    
        \begin{center}
            行为树
        \end{center} &
        \begin{center}
            哨兵在场上的自主决策
        \end{center} &
        \begin{center}
            使用BehaviorTreeCPP功能包构建哨兵在各种情况下与该情况的需求相对应的功能,并通过通讯的接口下发到下位机执行
        \end{center} \\
        
    \hline
    
        \begin{center}
            基础功能
        \end{center} &
        \begin{center}
            哨兵底盘移动,云台转动、发射等
        \end{center} &
        \begin{center}
            舵轮底盘在速度模式下可以全向移动,舵向电机外翻时行进不转舵,在陀螺模式下全向移动,云台能够迅速的实现回中,巡逻,自瞄,导航等功能,和依据战术实现云台发现敌方,另外 一个云台迅速转到对应角度去击打敌方等功能
        \end{center} \\
        
    \hline
    
\end{longtable}
%%% 表格演示-END
%%% 其中,开头的 { X | X | X | X |} 中,竖线标识分割线,字母X为自动对齐,其余可替换的lrc分别为左对齐、右对齐、居中对齐
%%% 该表格的一大特性是跨页后仍保留表头内容,即 \endhead 前的内容