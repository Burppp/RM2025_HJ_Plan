%%% 本文档作为功能演示

%%% 表格演示-BEGIN
\begin{longtable}{ p{2cm} | p{3cm} | p{3cm} | p{4.8cm} | p{2cm} |}

    \hline

    %%% 此处标识以上为表格脚,即在每页的表格底部重复显示的内容(一条横线)
    \endfoot
    
    %%% 此处标识该行颜色为指定定义的颜色,tabhdcolor 在 ./introduction.tex 中定义,颜色为hsb(0, 0, 0.82353),即82度灰
    \rowcolor{tabhdcolor}

        \begin{center}
            项目
        \end{center}  &
        \begin{center}
            任务
        \end{center}  &
        \begin{center}
           人力评估
        \end{center} &
        \begin{center}
            人员技能要求
        \end{center}  &
        \begin{center}
            耗时评估
        \end{center}  \\ 
        
    \hline

    %%% 此处标识以上为表格头,即在每页的表格头部重复显示的内容(一堆header)
    \endhead

        %%% 第2行的表格信息
        %\multirow{3}{*}{机械组} &
        \begin{center}
            双极发射
        \end{center} &
        \begin{center}
            将英雄机器人的发射机构由三摩擦轮优化为双级摩擦发射,提高吊射以及中距离击打的命中率
        \end{center} &
        \begin{center}
            1名机械组成员\linebreak1名电控成员
        \end{center} &
        \begin{center}
            熟悉运动简图分析,并且对于弹链运动有清晰的了解。并且对于弹丸定心和稳定需求要有一定的理解
        \end{center} &
        \begin{center}
            3周
        \end{center}\\
        
    \hline
        \begin{center}
            丝杆偏置
        \end{center} &
        %%% 第3行的表格信息
        \begin{center}
            将旧云台的中心丝杆放置位置更改为丝杆外置,提高机械方面的俯仰角度
        \end{center} &
        %%% 第3行的第2~3列单元格被合并,左对齐,且左右均有分割线
        %\multicolumn{2}{| l |}{content 3 2-3} &
        \begin{center}
            1名机械组成员
        \end{center} &
        \begin{center}
            能够使用建模软件对于运动进行简单的建模仿真分析,并且要对于材料力学有一定的学习,基本需要了解如何应力分析并且避免应力集中
        \end{center} &
        \begin{center}
            1周
        \end{center}\\

    \hline
    
        \begin{center}
            图传结构
        \end{center} &
        \begin{center}
            重新设计图传结构,利用简单连杆实现图传的简单运动并增强稳定性。并且重新配置望远镜的放置位置以及机械限位
        \end{center} &
        \begin{center}
            1名机械组成员
        \end{center} &
        \begin{center}
            对于机械原理有所学习,可以绘制简单的摇杆机构
        \end{center} &
        \begin{center}
            1周
        \end{center} \\
        
    \hline
    
        %%% 第4行的表格信息
        %%% 第4行开始,连续2行的第1列单元格被合并,自动列宽
        %\multirow{3}{*}{电控组} &
        \begin{center}
            舵轮驱动
        \end{center} &
        \begin{center}
            将麦轮驱动结构更改为舵轮驱动结构
        \end{center} &
        \begin{center}
            1名机械组成员,2名电控组成员
        \end{center} &
        \begin{center}
            
        \end{center} &
        \begin{center}
            2周
        \end{center}\\

    \hline
    %%% 此处分割一行的第2至3个单元格
    %\cline{2-3}
    
        %%% 第5行的表格信息
        %%% 第5行由于第1列在上一行已被合并,所以建议不要写入信息,虽然写入信息也会显示出
        \begin{center}
            中心供弹
        \end{center} &
        \begin{center}
            供弹方式由下供弹改为中心供弹
        \end{center} &
        \begin{center}
            1名机械组成员,2名电控组成员
        \end{center} &
        \begin{center}
            
        \end{center} &
        \begin{center}
            2周
        \end{center}\\

    \hline
    
        \begin{center}
            
        \end{center} &
        \begin{center}
            
        \end{center} &
        \begin{center}
            
        \end{center} &
        \begin{center}
            
        \end{center} &
        \begin{center}
            
        \end{center} \\
        
    \hline
    
        %%% 第6行的表格信息
        %\multirow{3}{*}{视觉组} &
        &
        &
        %%% 第6行开始,连续2行的第2~4列单元格被合并,居中对齐,且左侧有分割线,自动列宽
        %\multicolumn{3}{| c }{\multirow{2}{*}{content 6-7 2-4}} \\
        &
        &
        \\

    \hline
    %%% 此处分割一行的第1至1个单元格,即只有第一个单元格下面有横线
    %\cline{1-1}
    
        %%% 第7行的表格信息
        %content 7 1 &
        %%% 第7行由于第2~4列在上一行已被合并,所以建议不要写入信息,虽然写入信息也会显示出,而且建议像下面一样合并表格防止竖线分割
        %\multicolumn{3}{| c }{} \\

        &
        &
        &
        &
        \\

    \hline

        &
        &
        &
        &
        \\
        
    \hline

        %\multirow{3}{*}{硬件组} &
        &
        &
        &
        &
        \\

    \hline

        &
        &
        &
        &
        \\

    \hline

        &
        &
        &
        &
        \\
    
\end{longtable}
%%% 表格演示-END
%%% 其中,开头的 { X | X | X | X |} 中,竖线标识分割线,字母X为自动对齐,其余可替换的lrc分别为左对齐、右对齐、居中对齐
%%% 该表格的一大特性是跨页后仍保留表头内容,即 \endhead 前的内容