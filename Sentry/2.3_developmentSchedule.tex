%%% 本文档作为功能演示

%%% 表格演示-BEGIN
\begin{longtable}{ p{2cm} | p{3cm} | p{3cm} | p{4.8cm} | p{2cm} |}

    \hline

    %%% 此处标识以上为表格脚,即在每页的表格底部重复显示的内容(一条横线)
    \endfoot
    
    %%% 此处标识该行颜色为指定定义的颜色,tabhdcolor 在 ./introduction.tex 中定义,颜色为hsb(0, 0, 0.82353),即82度灰
    \rowcolor{tabhdcolor}

        \begin{center}
            项目
        \end{center}  &
        \begin{center}
            任务
        \end{center}  &
        \begin{center}
           人力评估
        \end{center} &
        \begin{center}
            人员技能要求
        \end{center}  &
        \begin{center}
            耗时评估
        \end{center}  \\ 
        
    \hline

    %%% 此处标识以上为表格头,即在每页的表格头部重复显示的内容(一堆header)
    \endhead

        %%% 第2行的表格信息
        %\multirow{3}{*}{机械组} &
        \begin{center}
            重定位
        \end{center} &
        \begin{center}
            完成重定位算法
        \end{center} &
        \begin{center}
            1名视觉组成员
        \end{center} &
        \begin{center}
            了解icp点云配准算法,熟悉ros开发
        \end{center} &
        \begin{center}
            4周
        \end{center}\\
        
    \hline
        \begin{center}
            避障
        \end{center} &
        %%% 第3行的表格信息
        \begin{center}
            完成避障算法
        \end{center} &
        %%% 第3行的第2~3列单元格被合并,左对齐,且左右均有分割线
        %\multicolumn{2}{| l |}{content 3 2-3} &
        \begin{center}
            1名视觉组成员
        \end{center} &
        \begin{center}
           熟悉ros开发
        \end{center} &
        \begin{center}
            2周
        \end{center}\\

    \hline
    
        \begin{center}
            全向感知
        \end{center} &
        \begin{center}
            完善全向感知算法
        \end{center} &
        \begin{center}
            1名视觉组成员
        \end{center} &
        \begin{center}
            熟悉opencv,熟悉ros
        \end{center} &
        \begin{center}
            4周
        \end{center} \\
        
    \hline
    
        %%% 第4行的表格信息
        %%% 第4行开始,连续2行的第1列单元格被合并,自动列宽
        %\multirow{3}{*}{电控组} &
        \begin{center}
            底盘控制
        \end{center} &
        \begin{center}
            底盘控制优化
        \end{center} &
        \begin{center}
            1名电控组成员
        \end{center} &
        \begin{center}
            舵轮的运动解算和代码的实现,熟悉PID算法
        \end{center} &
        \begin{center}
            2周
        \end{center}\\

    \hline
    %%% 此处分割一行的第2至3个单元格
    %\cline{2-3}
    
        %%% 第5行的表格信息
        %%% 第5行由于第1列在上一行已被合并,所以建议不要写入信息,虽然写入信息也会显示出
        \begin{center}
            云台控制
        \end{center} &
        \begin{center}
            云台重构、优化
        \end{center} &
        \begin{center}
            1名电控组成员
        \end{center} &
        \begin{center}
            熟悉云台位姿解算和IMU数据处理,熟悉PID算法
        \end{center} &
        \begin{center}
            4周
        \end{center}\\

    \hline
    
        \begin{center}
            功率控制
        \end{center} &
        \begin{center}
            功率控制
        \end{center} &
        \begin{center}
            1名电控组成员
        \end{center} &
        \begin{center}
            熟悉卡尔曼滤波,了解超电通讯和华科功率控制的代码实现
        \end{center} &
        \begin{center}
            1周
        \end{center} \\
        
    \hline
    
        \begin{center}
            决策
        \end{center} &
        \begin{center}
            行为树控制
        \end{center} &
        \begin{center}
            1名电控组成员,1名视觉组成员
        \end{center} &
        \begin{center}
            ROS和python/C++
        \end{center} &
        \begin{center}
            8周
        \end{center} \\

    \hline
    %%% 此处分割一行的第1至1个单元格,即只有第一个单元格下面有横线
    %\cline{1-1}
    
        %%% 第7行的表格信息
        %content 7 1 &
        %%% 第7行由于第2~4列在上一行已被合并,所以建议不要写入信息,虽然写入信息也会显示出,而且建议像下面一样合并表格防止竖线分割
        %\multicolumn{3}{| c }{} \\

        \begin{center}
            数据通信
        \end{center} &
        \begin{center}
            数据链路优化,新增雷达通讯接口
        \end{center} &
        \begin{center}
            1名电控组成员
        \end{center} &
        \begin{center}
            熟悉各类通讯协议及其代码实现
        \end{center} &
        \begin{center}
            2周
        \end{center} \\

    \hline

        \begin{center}
            布线
        \end{center} &
        \begin{center}
            布线优化
        \end{center} &
        \begin{center}
            1名电控组成员
        \end{center} &
        \begin{center}
            了解布线细节、了解维护需求、熟悉各接口线序等
        \end{center} &
        \begin{center}
            1周
        \end{center} \\
        
    \hline
    
\end{longtable}
%%% 表格演示-END
%%% 其中,开头的 { X | X | X | X |} 中,竖线标识分割线,字母X为自动对齐,其余可替换的lrc分别为左对齐、右对齐、居中对齐
%%% 该表格的一大特性是跨页后仍保留表头内容,即 \endhead 前的内容