\subsection{各方向上的目标}

    \subsubsection{目标成绩}

        在比赛成绩上,由于我们战队已经有 5 次参加 RoboMaster 比赛的经验,并有两次成功参加超级对抗赛的经验,战队内经过全体讨论决定,本赛季希望能够进入全国赛。\par

    \subsubsection{任务进度管理制度}

        在赛季的各个阶段制定各兵种的整体任务进度安排,并将阶段性任务记录在飞书云文档,规定每个任务结束时间,每周定期由项目负责人核实验收上周任务完成情况并对遇到的问题进行总结并制定本周的工作安排,队长及项管参与每周各组工作安排会议,保证每个项目所需资源落实到位,进度情况健康。任务进度管理数字化记录,在项目管理系统上登记每周任务并实时更新任务状态,对进度情况不健康的项目做出及时调整。\par

    \subsubsection{队员管理制度}

        结合队员个人课程表情况对队员在实验室到岗研发时间有所要求,完善实验室考勤请假制度,考虑到学校人数越来越多,我们考虑对人员的要求要精益求精,整顿队风,对比赛参与度不高,研发积极性差,工作效率低效的队员作警告清退处理。同时提高实验室成员的责任意识,爱护实验室资产与环境,并制定了相应的奖惩制度,加强对梯队队员的培养。\par

    \subsubsection{物资管理制度}

        构建数字化流程化物资购买管理制度,增强对物资购买的流程审核及记录,提高现有资源利用率,减少不必要的浪费,对高价值物资如电机进行出入库登记处理,物资领用责任到人,对于由于低级错误或者恶意损坏物资的成员实行罚款制度。\par

    \subsubsection{新人培养体系}

        构建数字化流程化物资购买管理制度,增强对物资购买的流程审核及记录,提高现有资源利用率,减少不必要的浪费,对高价值物资如电机进行出入库登记处理,物资领用责任到人,对于由于低级错误或者恶意损坏物资的成员实行罚款制度。\par

    \subsubsection{管理流程改变}

        本赛季的管理方式继续沿用上赛季队长,项管主导项目方向,各技术组组长评估技术可行性,各车组负责人主导研发方向的形式,主要流程为:队长项管制定各车组的需求,技术组长分析技术可行性,的出最终需求结论,由车组负责人制定研发方向,再由技术成员讨论技术细节。\par

    \subsubsection{技术传承建设}

        在研发调试过程中遇到缺陷或陷阱及时记录,撰写“警钟长鸣”文档,对每个新研发的技术点留下相应的技术文档。在赛季结束都需要撰写自己负责部分的技术文档(包括已实现的技术细节和未实现的预研资料),安排相关负责人跟进文档进度。\par