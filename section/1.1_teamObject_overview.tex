\subsection{团队目标概括}

        深圳技术大学悍匠战队始建于2019年,至今已有5年的备赛经验,自22赛季首次踏入超级对抗赛赛场后,均未能在区域赛的小组中出现,未能实现赛季成绩的进一步突破。\par
        对此,我们剖此往届赛季备赛中出现的管理和技术层面的不足之处,复盘反思与强队交手的经历,25赛季我们战队的比赛目标是晋级全国赛。\par
        为实现这个多年来的愿望,在25赛季中加强进度监管、在原有基础之上细微调整和优化、备赛过程中侧重于测试、进一步体系化传承资料和财务等各方面流程、引导队员探索技术背后的原理;加强队伍文化建设,提高队员比赛积极性和参与感,营造追求极致、患难与共的团队氛围,将生而无畏,精益求精的团队理念贯彻到每个战队成员。\par

        \subsubsection{团队实际情况}

            \paragraph{队伍可用资源}

                得益于学校和学院的大力支持,本赛季战队在资金、研发场地等方面比较充足。本赛季战队的资金来源主要有以下几个方面:上赛季剩余的耗材费、加工费等,2025年学院竞赛专项经费、实验室运行费用等多项经费约30万元。\par
                战队共有研发装配实验室3个,用于加工及物料存放的实验室1个,并配有一个约100平方测试场地可以放置1个联盟赛3V3对抗赛场地或1/4个超级对抗赛场地。\par
                战队拥有多台 3D 打印机,台钻,切割机,磨砂机等设备用于加工研发,拥有电子白板、打印机等设备满足队内授课、开会、成果展示等需求,配备了云台相机、单反相机、无人机等设备,能够满足战队宣传需求。此外通过老队员的资助实验室还配备了冰箱、制冰机等设施,提高队员生活质量。整体来看,我们战队的资源相对充足,能够满足战队的研发需求。\par
                目前我队拥有团队成员共 80 余人,根据我校课程设置、实习安排等实际情况,我们队伍的人员构成主要如下:大二队员是队伍的主干力量,需要完成本赛季各兵种的基础研发装配调试迭代工作,是比赛的主要参与者;大三与大四留队队员参赛经验较为充足,但因为自身学业与未来就业原因,偶尔会在队内对队伍进行指导,同时根据队员自身情况,进行相应技术点的研究与突破。大一队员作为梯队成员,经行比赛基础基础技能学习,完成各组队内考核的任务,进行各类基础模块建设与调试,同时为鼓励大一成员积极参与,研发创新,表现优异者可作为正式队员参赛。\par
                关于技术积累部分,经过上赛季的转变,传承下来的资料相较完整和体系化,在25赛季将继续推进各组文本、视频等形式的技术传承。我们队内主要的技术资源获取途径有邀请老队员回来对技术研究可行性分析、老队员留下的技术文档、公开的论文资料、工业界成熟的技术下放、组织与其他队伍的线下技术交流、RoboMaster论坛各队开源技术分享,以及各RM交流群的技术交流。\par

        \subsubsection{本赛季要完成的基础内容及进阶优化内容}

                经过过去几年各赛季的复盘和反思,历史的备赛策略过于激进、新赛季多次推翻旧赛季千辛万苦打下的基础,赛季初踌躇满志、备赛时忙不过来、缺乏明确的测试计划、赛场上稳定性欠佳。队员们普遍缺乏科学合理的设计和控制策略,这反映了主流的以项目驱动发展的学习路线无法兼顾犄角旮旯的知识空缺,队员们往往只能够解决具体的工程实践问题,而对技术背后的原理漠不关心,这是队员和队伍谋求长远发展无法忽视的阻碍。\par
                稳定性:在设计时,需要充分考虑模块的使用场景,在结构和配件材料选型上尽量做到合理;在装配时选择正确的零件,在交付下一环节前完成可用性和强度的检查,并及时记录缺陷以便于下一版本迭代;注意线路的布置合理与线束的物理保护,对可能磨损线材的地方加强保护,留下记录,并在下一版本设计上做出调整。此外,只有长期高强度的测试能够检验稳定性是否良好,在缺乏科学的分析能力的情况下也只有测试能够暴露出不稳定缺陷。\par
                设计指标与测试计划:在机器人研发的初始节点定下机器人性能指标并根据制定相应的调试测试计划,在队里进度管理系统中留下记录。测试也不止步于记录,需要将大量的测试数据汇总后分析成因。\par


            
