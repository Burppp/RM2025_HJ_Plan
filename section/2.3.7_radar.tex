

\subsubsection{雷达系统}

    \paragraph{规则分析}

    \setlist[itemize]{label=\raisebox{-1.2ex}{\scalebox{3}{$\textbullet$}}}

    \begin{itemize}
        \item 雷达可以为队伍提供敌方机器人的位置信息,提供视野给到队伍。
        \item 雷达可以通过裁判系统将位置信息发给己方单位,并且只能接收哨兵的信息
        \item 己方的雷达可识别对方地面机器人的位置,并将该机器人的坐标发送至裁判系统,若精度符合条件则能持续积攒标记进度,当标记进度大于等于100时被标记的对象回获得一个-15\%的防御增益。
        \item 当雷达每累计使对方机器人易伤 1 分钟(同时有多台机器人易伤时,时间不累加),将会获得 1 次触发“双倍易伤”的机会,雷达可以通过裁判系统主动发送命令消耗机会,并使当前所有正处于易伤状态的负防御增益数值由-15\%变为-30\%,持续 30 秒。每局比赛中,雷达至多可以触发 2 次“双倍易伤”。
    \end{itemize}

    \paragraph{策略战术}

    \setlist[itemize]{label=\raisebox{-1.2ex}{\scalebox{3}{$\textbullet$}}}

    \begin{itemize}
        \item 精准识别出敌方机器人的位置信息,累计标记进度并发送信息给予队伍,为队伍增加优势。
        \item 当有双倍易伤的机会时,配合队伍需求,触发该效果。
        \item 识别出敌方机器人在某些特殊位置(如我方基地前、飞坡等)时,可以通过裁判系统向己方单位发送预警信息。
        \item 通过裁判系统接收哨兵传来的重定位的信息,更加精确的锁定部分目标.
        \item 协助英雄对地方基地进行吊射.
    \end{itemize}
    
    \paragraph{功能需求分析}

    \LTXtable{\textwidth}{Radar/2.3_functionalRequirement_analysis.tex}
    
    \paragraph{改进方向}

    \LTXtable{\textwidth}{Radar/2.3_improvementDirection.tex}

    \paragraph{研发进度安排}

    \LTXtable{\textwidth}{Radar/2.3_developmentSchedule.tex}

    \paragraph{项目组人员分配}

    \LTXtable{\textwidth}{Radar/2.3_personalAssignment.tex}
    