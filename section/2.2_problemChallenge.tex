\subsection{上赛季各项目组问题和挑战}

    \paragraph{机械组}

        \setlist[itemize]{label=\raisebox{-1.2ex}{\scalebox{3}{$\textbullet$}}}

        \begin{itemize}
            \item 上赛季机械组的主要问题为技术传承与研发进度。由于传承管理问题,老的上场车辆未能得到留存,使得上赛季各车组的机械结构几乎都是从零做起,在根本上影响了研发进度,造成技术断代。
            \item 且上赛季各车组机械结构有较大变化,使得研发任务较重,一定程度上也影响了进度。进度的滞后也使得后续测试时间严重不足,导致车辆稳定性完全不足以支撑三场比赛。
            \item 上赛季由于进度安排和技术传承的问题导致机械组成员在技术上无力深度钻研,延续至25赛季暴露出了机械组成员缺乏理论支撑和系统的分析方法的新问题,导致技术难以短期实现突破。
        \end{itemize}

    \paragraph{电控组}

        \setlist[itemize]{label=\raisebox{-1.2ex}{\scalebox{3}{$\textbullet$}}}

        \begin{itemize}
            \item 上赛季电控组的最大问题在于进度把控上,各兵种机器人的测试时间最短的只有一两周,测试不足导致部分问题没能在赛前测试阶段检查出来,从而导致赛时各兵种机器人的总体表现不稳定。
            \item 其次是新人流失问题,在新人培训结束后出于各种原因未能及时给新人布置任务导致新人长期处于无事可做的状态,而且新人缺乏引导,不知道有什么事情可以做,导致新人的贡献度很低,第一批入队的新人流失了一半有余。
            \item 电控方面同样出现了与机械组类似的“技术空心”问题,队员更倾向于项目驱动的发展路线,在过去的一个赛季中培养起了项目实践和解决实际问题的能力,但普遍缺乏对理论知识的钻研精神,后续将有引导性地安排任务。
            \item 由于上述缺陷,导致平衡步兵的技术传承进展比较困难,进而导致新老队员在技术认知上产生分歧、爆发矛盾。
        \end{itemize}

    \paragraph{视觉组}

        \setlist[itemize]{label=\raisebox{-1.2ex}{\scalebox{3}{$\textbullet$}}}

        \begin{itemize}
            \item 由于没有自适应弹道校正,弹道需要手动调整。然而很多时候发现场外测好的抬枪补偿数据,在场内不适用,出现打偏的情况。每次比赛都要测试抬枪补偿的流程过于繁琐和着急,而且场外空间很有限,十分不便。
            \item 哨兵缺少初始位姿校准流程,导航路线会向一个方向偏移;自瞄时云台抖动严重.
            \item 视觉组在上个赛季中暴露出人手严重不足的问题,甚至需要一名视觉组成员负责维护多个兵种,初步认为是算法学习周期长、学习成本大导致的,而且学习期间很难参与机械和电控的进度,导致后期视觉新人无法融入集体、被边缘化,最后人员流失。25赛季中以兵种为单位分配工位、鼓励视觉组新人参与实验室相关活动、进一步加强算法传承、老人带领新人参加其他比赛练手等措施。
        \end{itemize}

    \paragraph{硬件组}

        \setlist[itemize]{label=\raisebox{-1.2ex}{\scalebox{3}{$\textbullet$}}}

        \begin{itemize}
            \item 在上一赛季中,硬件组遭遇了新成员参与度不足的问题。根源在于开展了周期过长的培训工作,新成员在赛季中期尚未完全融入团队的日常调试和维护工作,导致人手短缺,影响了团队的整体运作效率。
            \item 此外,新老技术传承与交替的过程中出现了问题,在备赛中期出现较长时间的技术断层。
            \item 上个赛季中队伍整体的硬件需求局限于超级电容的开发与维护,碍于超级电容的学习成本,新人难以参与其中。25赛季中整体队伍对硬件的需求陡增,反而导致硬件组人力短缺的情况,在招新中将硬件组独立于电控组招揽新人。
        \end{itemize}

    \paragraph{运营组}

        \setlist[itemize]{label=\raisebox{-1.2ex}{\scalebox{3}{$\textbullet$}}}

        \begin{itemize}
            \item 在财务管理方面,由于比赛周期长、投入大,战队面临着资金筹集与合理分配的巨大压力。一方面,运营组需要积极寻求学校支持,确保战队能够持续运转;另一方面,又要精打细算,对每一笔支出进行严格审核,确保资金用在刀刃上,如耗材采购、日常运营等,这对成员们的财务规划能力提出了极高要求。
            \item 在周边设计方面,运营组旨在通过创意周边产品增强战队影响力和团队凝聚力。然而,如何在体现RoboMaster赛事文化的同时,又能贴近大学生群体喜好,设计出既实用又具有吸引力的周边,成本控制与生产效率的考验让战队在设计、生产过程中必须再三权衡。
            \item 在宣传方面,如何在众多参赛队伍中脱颖而出,有效传达悍匠战队的技术实力、战队理念及比赛亮点,是一项不可小觑的任务。运营组不仅要策划并执行多渠道的宣传工作,还要应对内容同质化等挑战。
            \item 在25赛季中主要由管理层优化运营组工作流程来减轻运营组成员的压力,例如将财务压力部分分摊到项管、技术组组长上,优化采购流程;另一方面给运营组提供足够的资源支持其带领技术组成员参与其他比赛,并以此为亮点吸引组织外的运营人才参与比赛和转化为运营组成员。
            \item 在设计方面,广泛汲取其他队伍的优秀设计经验并结合队员的审美偏好设计有创意、实用或有吸引力的周边。
        \end{itemize}