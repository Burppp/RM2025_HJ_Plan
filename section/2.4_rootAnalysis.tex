\subsection{根因分析}

    \subsubsection{问题分析}

        \setlist[itemize]{label=\raisebox{-1.2ex}{\scalebox{3}{$\textbullet$}}}

        \begin{itemize}
            \item 为什么赛场上四号步兵底盘异常?
            \item 四号步兵右前轮电机三相线磨破短路。
            \item 为什么三相线会磨破?
            \item 检录时干涉、磨线。
            \item 为什么测试时会干涉、磨线,为什么缺乏测试?
            \item 结构设计时没有预留足够的走线空间。
            \item 为什么在赛场上才暴露出这个问题?
            \item 备赛过程中没有预留足够的测试时间。
            \item 为什么没有足够的测试时间?
            \item 早期进度规划过于理想化、队员不重视测试。\\
            
            \item 为什么赛场上无人机摇摇晃晃、飘忽不定?
            \item 光流不稳定且妙算电脑不能自启动。
            \item 为什么光流不稳定且没有更换电脑?
            \item 前期进度安排不重视光流,且电脑的选型考虑片面,仅仅考虑重量问题。
            \item 为什么不重视光流定高?
            \item 因为没有进行长期规划,仅仅追求阶段性成果,初始目标较低(不炸就行)。
            \item 为什么没有做好长期规划?
            \item 因为无人机作为此前没有尝试过的兵种,从零开始会遇到很多困难,更多的预期和注意力放在了传统的地面作战单元上。\\
            
            \item 为什么无人机在比赛中出现pitch轴卡住的情况?
            \item 因为pitch的限位逻辑是不适用于无人机的工况的,整机会出现较大的前倾或后仰的。
            \item 为什么限位逻辑不适应具体工况也没有及时纠偏?
            \item 事实上队内已经有合适的解决方案,因为没有充分及时的沟通,这个问题到赛前也没有解决。
            \item 为什么队内会存在这样的信息差?
            \item 队员总是专注于眼前的难题、并不关注通用技术泛化到其他兵种。\\
            
            \item 为什么赛场上无人机会突然出现掉高的情况?
            \item 因为分电方案是两并两串,其他需要24v电压的设备从串联中间取电,导致在开启摩擦轮或者拨盘堵转的时候耗电量突增,整体电压下降而掉高。
            \item 为什么没有设计出合适的无人机分电方案?
            \item 硬件组人手短缺、技术传承艰难。\\
            
            \item 为什么进度管理如此艰难?
            \item 进度管理流程繁杂,负责人无法自觉上传车组进度,需要队长、项管挨个问。
            \item 为什么没有优化进度管理流程?
            \item 队员总是专注于攻克技术难题,无法估计整体的进度规划。\\

            \item 为什么经费管理没有预计的规范化?
            \item 因为造新车大量的物资需求和迟滞的进度,有大量的物资采购、物流信息来往。
            \item 为什么大量的物资无法妥善分配管理?
            \item 采购物资太散太细,填写采购时并不考虑节约物资,财务人手短缺。\\

            \item 为什么大一新人参与度普遍低?
            \item 因为培训后,备赛过程中疏于培养新人,没有给他们更多参与的机会。
            \item 为什么疏于培养新人?
            \item 进度滞后时主力开发在推动进度,空闲时懒得教基础尚浅的新人,新人难以一同进步,两方隔阂越来越大,最终导致梯队人员流失。\\

            \item 为什么备赛时有人空闲有人忙碌?
            \item 赛季规划时没有充分考虑每个主力开发的能力和时间精力,只考虑了每一个功能需求有人去完善。
            \item 为什么规划时没有充分考虑人员需求?
            \item 没有意识到队员的成长总是非线性的,很大程度上取决于对比赛和机器人的热爱,而不是已有的技术基础。\\

            \item 为什么工具、螺丝、标准件线材整理混乱?
            \item 因为各组进度不一,队员取用工具、标准件等习惯不好,整体不注重实验室工具整齐。
            \item 为什么各组进度差异较大、取用工具不规范?
            \item 缺少一套详细的备赛流程和进度规划,和落实进度规划的决心和方案。\\

            \item 为什么场地利用率低?
            \item 整体缺乏测试,而且没有意识到测试的重要性。\\

            \item 为什么设计缺陷无法及时更改?
            \item 因为进度缓慢,大多数时候在被赛程推着走,技术上难以更进一步;由于缺乏测试,许多设计缺陷无法及时发现。\\

            \item 为什么到赛前还总是忙前忙后?
            \item 缺少一套详细的赛前维护、没有实现弹道自动校正、需要手动调整偏置。\\
        \end{itemize}