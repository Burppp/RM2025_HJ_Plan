\subsubsection{飞镖系统}

    \paragraph{规则分析}

        \setlist[itemize]{label=\raisebox{-1.2ex}{\scalebox{3}{$\textbullet$}}}

        \begin{itemize}
            \item 飞镖命中收益改变:25赛季飞镖命中的基地固定目标和随机固定目标的收益下降。具体表现为命中基地固定目标、随机固定目标为步兵和英雄机器人下降到200和600、随机固定目标命中扣除地方机器人血量由25\%改为10\%、命中固定目标由1000下调至625,命中随机固定目标由1200下调至2500。25赛季新增的随机移动目标,命中难度更大,但收益比24赛季的随机固定目标更高。随机移动目标命中提供2500点巨额经验,同时对方操作收界面遮挡15秒,且对方全部存活的地面机器人立即受到相当于各自当前上限血量25\%的攻击伤害、对方基地护甲立即展开。随机固定目标和随机移动目标的收益下降、随机移动目标收益巨大,使我们这赛季击打的目标确定在随机固定目标甚至随机移动目标上。
            \item 场地变化:对比24赛季,飞镖发射站的结构有所变化,三分钟准备阶段内,飞镖发射站闸门将保持闭合状态,飞镖系统需从飞镖发射站的后方放入。这意味着在准备阶段无法进行人工标定及预瞄,飞镖系统需要更有效的利用开启闸门后的时间,对飞镖系统纯视觉瞄准的速度以及精度都有一定的要求。25赛季飞镖发射站闸门的开启次数和时间节点从原本的开始比赛30秒后两次开启机会,变为了开始比赛30秒后一次和4分钟后一次且未使用的机会可以累加。这一改动使得前期无法集中使用飞镖,对于需要在前期使用飞镖获得更大战果的情况,飞镖的准度和命中的稳定性显得更为重要。对比24赛季,从场地尺寸上来看前哨站与飞镖发射站的相对位置变化较为明显,夹角从24赛季的6.6°变为了2.1°。除此之外,飞镖发射站和基地沿与战场长边平行方向的距离减少了640mm。
        \end{itemize}
    
    \paragraph{策略战术}

        \setlist[itemize]{label=\raisebox{-1.2ex}{\scalebox{3}{$\textbullet$}}}

        \begin{itemize}
            \item 飞镖定位:根据25赛季规则的改动,飞镖在25赛季的定位偏向于战略进攻单位,优势时与地面配合进攻,在局势焦灼时通过飞镖致盲效果实现局势逆转。
            \item 准备阶段:发射台是从后方开放放入飞镖系统,在赛前准备时间3分钟内不能人工标定前哨站与基地,因此决定在裁判系统自检的五秒内通过飞镖架的摄像头识别引导灯,标定前哨站并记录敌方基地方位。提前锁定目标方向。缩短飞镖发射时间。
            \item 目标选择:在基地随机移动靶和随机固定靶带来的巨大收益,本赛季飞镖击打的目标需往随机移动靶和随机固定靶靠拢,由此需要研发制导镖架和制导镖体的预研发。
            \item 发射结构:由于摩擦轮发射的飞镖发射架具有较多的不稳定因素,本赛季决定采用拉簧弹射的方式,解决了摩擦轮旋转转速不一致可能导致飞镖不在预估轨道上的问题,同时,拉簧可以为镖体提供稳定的发射能量。
            \item 战术选择:飞镖系统“致盲”的时间改动,增加了首次命中的的“致盲”效果的时间,这	提高飞镖的命中的重要性,充分利用飞镖的致盲效果,配合地面单位对敌方发起进攻,对敌方进行火力压制,消灭敌方地面单位,为己方建立优势。
        \end{itemize}
    
    \paragraph{功能需求分析}

    \LTXtable{\textwidth}{Dart/2.3_functionalRequirement_analysis.tex}
    
    \paragraph{改进方向}

    \LTXtable{\textwidth}{Dart/2.3_improvementDirection.tex}

    \paragraph{研发进度安排}

    \LTXtable{\textwidth}{Dart/2.3_developmentSchedule.tex}

    \paragraph{项目组人员分配}

    \LTXtable{\textwidth}{Dart/2.3_personalAssignment.tex}
    