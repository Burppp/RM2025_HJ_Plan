\subsubsection{工程机器人}

    \paragraph{规则分析}

        \setlist[itemize]{label=\raisebox{-1.2ex}{\scalebox{3}{$\textbullet$}}}

        \begin{itemize}
            \item 小资源岛变更:小资源岛依旧紧贴中央高地护栏外侧,但银矿由原来的半嵌入式改变为完全嵌入,从原来可以通过夹爪直接取出改变成要求工程机器人从上面将其拔出,对工程机器人的机构要求更加灵活。
            \item 前往资源岛的路径变更:原来前往大资源岛的路径可以通过前哨站然后直接登上大资源岛或者直接钻隧道,而25赛季登上大资源岛的路径更加灵活,在基地正方向新增一个高300mm的二级台阶,第一级是高为200mm的小资源岛,所以工程机器人可以设计一个灵活的机构来完成登岛直接抵达大资源岛,而且现在是公路区连接着中央高地,所以工程机器人也可以选择绕远路先通过公路区再上岛,而上公路区又有选择,可以通过上200mm高的台阶进入公路区,也可以通过爬10°坡上公路区进而前往大资源岛,所以工程机器人的构型可以有更多选择特别是在底盘上。
            \item 经济体制改变:删除了五级难度的兑矿等级,等比例增加了剩余难度等q级的金币数量,且比例提高了随着兑矿次数增多时最低可选择兑矿难度的金币倍率,且新增选择四级难度时,工程机器人需要在15秒的时间限制内,连续兑换2块矿石,若在时限内兑换成功,2块矿石的所得金币将依次连续结算,否则将不会获得任何金币的机制,这需要工程机器人需要有非常灵活的兑矿机构,且操作手需要对操作更加熟练。 
        \end{itemize}
    
    \paragraph{策略战术}

        \setlist[itemize]{label=\raisebox{-1.2ex}{\scalebox{3}{$\textbullet$}}}

        \begin{itemize}
            \item 工程机器人定位:根据 25赛季规则的改动,我们认为工程机器人在25赛季的定位更加 偏向于通过取矿-兑矿换取金币达到为团队提供经济的一个角色,为其他车组提供更好的 配置条件,为多样化的战术选择提供经济支持。
            \item 取矿机构:六轴机械臂+吸盘。考虑到维修成本和迭代的选择上,使用板材替代上届的机加工方案。同时考虑到6r型机械臂Link1和Link2电机的承重,决定在这两个关节中引入弹簧的重力补偿,去避免电机的发热,延长使用寿命。
            \item 底盘选择:月球车底盘。由于场地的变更,工程机器人可以选择跨台阶的方式直接登上大资源岛,为了更快的争夺金矿来提高团队经济,我们选择可以适应复杂地形的月球车底盘来实现跨台阶的目标,在轮组的选择上我们将采用麦克纳姆轮来实现底盘的转向。 抬升兼存矿机构:剪式抬升架+一键2/3银矿.
            \item 抬升架的选择:剪式抬升。由于六轴机械臂的操控,我们采用剪式抬升架来抬升,并且在上面安装取银矿机构,使得可以取2/3个矿,由规则手册中场地部分可知,每两个银矿的中心距离为270mm,一键3矿则需要540mm,是可以尝试并且完成的(就怕后面又改规则),同时银矿一开始距离地面200mm,完全抽出来则需要400mm,所以会做一个180度的翻转使其更加稳定并且为后期存矿做足准备。
            \item 自定义控制器:带有六个角传感器的大臂缩小机械臂,采用大臂构型缩小的自定义控制器,有利用操作手更加直观的操作机械臂的运动,并同时减少解算需求,并考虑在自定义控制器中融入移动、吸取等快捷功能,以提升工程在存取矿的能力。
            \item 上位机实现:上位机采用ros2和movelt,将实时获取大臂的六个角度并跟踪机械臂的构型,在需要存取矿时,上位机将实现一次性取矿的路径规划。
            \item 下位机:优化各部分通讯,尝试在下位机实现机械臂正逆解算。
        \end{itemize}
    
    \paragraph{功能需求分析}

    \LTXtable{\textwidth}{Engineer/2.3_functionalRequirement_analysis.tex}
    
    \paragraph{改进方向}

    \LTXtable{\textwidth}{Engineer/2.3_improvementDirection.tex}

    \paragraph{研发进度安排}

    \LTXtable{\textwidth}{Engineer/2.3_developmentSchedule.tex}

    \paragraph{项目组人员分配}

    \LTXtable{\textwidth}{Engineer/2.3_personalAssignment.tex}
    