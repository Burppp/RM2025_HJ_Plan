\subsubsection{机械组}

    \paragraph{现有技术优化}

        \setlist[itemize]{label=\raisebox{-1.2ex}{\scalebox{3}{$\textbullet$}}}
    
        \begin{itemize}
            \item 技术点:步兵枪管上下v型轴承定心与单发限位。
            \item 优化目标:现步兵枪管基本为上下v型轴承定心且充当单发限位,可以做到不卡弹,但弹道散布一般且有双发问题,调整轴承位置中心及选型以达到不双发且不漏弹。
            \item 技术点:轮系电机内嵌式设计。
            \item 优化目标:步兵与英雄机器人如使用常见的电机直接与麦克纳姆轮连接,整个机器人自身重量的所有冲击都会作用在电机轴上,出现“外八”的情况,而内嵌式轮组将冲击分散给电机套筒,从而解决了这个问题。继续沿用电机内嵌式结构,搭配纵臂悬挂来获得比较好的避震效果。
            \item 技术点:半下供弹云台和下供云台。
            \item 优化目标:为解决弹丸数量云台重心影响大的问题,设计出半下供弹云台和下供云台,将弹舱与pitch分离优化拨盘,降低拨盘卡弹率;优化拨盘以及弹链位置以及尺寸,减小云台转动半径。
            \item 技术点:渐开线拨盘。
            \item 优化目标:为解决卡弹问题,设计出渐开线式的拨盘和拨爪,契合弹丸运动轨迹,有效减少卡弹的几率。
            \item 技术点:中置弹舱设计。
            \item 优化目标:空中机器人的重心最佳位置在机翼平面,而内嵌式弹仓可以打通上下中心板,使其重心在中间位置。继续对中置弹舱做极限设计,极限利用空中机器人中心板之间的位置,而又能保证其强度。
            \item 技术点:管结构机架。
            \item 优化目标:为减少空中机器人整体重量,采用管结构机架替代层结构,在保证整体刚度和正常飞行的同时有效减少较多重量。
            \item 技术点:同步带涨紧技术。
            \item 优化目标:英雄机器人yaw轴采用自行设计松紧可调节的利用法兰轴承进行张紧的结构对同步带进行涨紧;工程机器人机械臂的同步带采用预留多个孔位用于安装轴承进行涨紧的方式,对同步带进行涨紧,消除同步带传动的虚位。
            \item 技术点:双级摩擦轮发射机构。
            \item 优化目标:利用两对摩擦轮实现双级加速,延长加速道路,使弹速尽量稳定。并且在上下部分增加了两个摩擦轮进行单发限位,并且进行上下精确定心,进一步减小弹道扩散,提高吊射命中率。
            \item 技术点:转臂平衡重力补偿。
            \item 优化目标:应用在步兵和英雄机器人上,可以对云台进行配平(使用齿轮与滑块与拉簧制作的紧凑型重力补偿,将小齿轮固连在云台pitch轴上,在pitch轴转动时,带动另一个大齿轮旋转,而大齿轮上连接着一个连杆,连杆会推动滑块,而滑块上有一根拉簧做连接,从而达到在云台旋转时,拉簧做重力补偿。
            \item 技术点:鹅颈供弹。
            \item 优化目标:顺滑不卡弹的鹅颈供弹,配合上超高弹频的拨盘,可以大大程度增加机器人攻击性。可应用在哨兵上,一定程度减少卡弹情况,并且可以优化云台空间利用率,轻量化设计云台。
            \item 技术点:舵轮轮组。
            \item 优化目标:目前以3508电机和6020电机为驱动制作了两版舵轮底盘供英雄机器人和步兵机器人使用,采用垂直悬挂作为减震方式,以90°夹角的两组滑块滑轨限制轮组的自由度。舵轮轮组的装配方式有待进行测试改善,减小维修难度,对轮组部分做有效保护。
            \item 技术点:剪式升降台。
            \item 优化目标:供工程机器人用来抬升取矿机械臂,通过丝杆来控制高度,稳定性高、承载能力强,本赛季将继续优化,提高其升高高度。
            \item 技术点:自定义控制器。
            \item 优化目标:机械制作大臂缩小模型的小机械臂,带有六个角度传感器,将角度直接传入大臂。争取将大部分功能集合在自定义控制器中且做出舒适且精确的自定义控制器。
            \item 技术点:拉簧式弹射型发射架。
            \item 优化目标:将飞镖机器人的摩擦轮式发射架更改为拉簧式弹射型发射架,利用蓄能能量稳定的优势,减小飞镖体落点散布,增加飞镖发射的稳定性。
        \end{itemize}

    \paragraph{技术发展计划}
        