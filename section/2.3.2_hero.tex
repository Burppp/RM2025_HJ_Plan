\subsubsection{英雄机器人}

    \paragraph{规则分析}

        \setlist[itemize]{label=\raisebox{-1.2ex}{\scalebox{3}{$\textbullet$}}}

        \begin{itemize}
            \item 新赛季对英雄性能体系作出较大改变,发射机构为默认模式,发射速度上限默认为16m/s, 相较于上赛季的弹速优先发射机构,本赛季的发射机构的枪口热量上限和枪口热量每秒 冷却值都提升了,跟上赛季爆发优先的发射机构的初始数值一致,相当于本赛季默认的 发射机构结合了上赛季爆发优先和弹速优先的优点,有利于英雄机器人在短时间内持续 进行有效输出,英雄机器人的输出能力会得到提高。
            \item 新增雷达机制被对方雷达机器人标记进度大于100时,对方攻击效益加增百分之十五, 要求英雄机器人机动性更强,躲避对方集火攻击。
            \item 新赛季增益点机制中,相较于上赛季的增益点增益都是单数值,本赛季增益点增益值随 时间,在比赛开始2-3分钟、3-5分钟、5-7分钟时,占领己方基地增益点或高地增益点 深圳技术大学 悍匠  或飞坡增益点或前哨站增益点的机器人,分别可获得2、3、5倍枪口热量冷却增益,比赛 开始前两分钟无冷却增益,前期在R2高地上不能短时间内攻破对方前哨战,因此本赛季 英雄机器人前期攻打前哨战保险方案是在公路区或是R3高地对前哨战进行击打,这对英 雄机器人命中率有了更高的要求。
            \item 新赛季中R3梯形高地的变化使得英雄机器人在此处增益点的机动性更强,利于英雄机器 人在面对对方步兵机器人的追捕;并且,新赛季中R2环形高地新增隧道地形,机器人可 通过该隧道从基地区突击至荒地区;公路区的台阶高度降低有利于降低英雄机器人下台 阶卡住的概率;台阶高度的降低也降低了英雄机器人上台阶的难度,研发可上台阶的机 构也成为本赛季的一大技术难点。
            \item 新赛季中新增半自动控制模式,选手端将会显示与云台手近似的大地图界面,操作手可 以通过在大地图界面内点击的方式向所控制的机器人发送信息。并且在该种模式下机器 人获得较大增益,相较于手动操作,有 50\%的经验增益。但需要英雄机器人具有建图和 导航能力,还需要有精准的自瞄能力,对英雄机器人各个方面提出了更高的要求。
            \item 经验体系:新增发射弹丸即可获得经验机制;对机器人、基地和前哨站造成伤害即可获得 经验机制;删除随时间自然增长经验值机制;击毁机器人时击毁者和助攻者经验值获取 计算方式发生改变;新增狙击伤害获得经验值机制和首次获得飞坡增益获取 300 经验值 机制。总之,在战术上英雄机器人应主动出击,增强飞坡能力,快速精准狙击敌方机器人 和基地、前哨站,以较快获取经验升级。
        \end{itemize}
    
    \paragraph{策略战术}

        \setlist[itemize]{label=\raisebox{-1.2ex}{\scalebox{3}{$\textbullet$}}}

        \begin{itemize}
            \item 英雄机器人作为伤害值最高的地面机器人,参与团战可在关键混战时刻实现破局,并且 击杀敌方机器人也可使英雄机器人快速获得经验升级。
            \item 普通的英雄机器人一般都采用推进到建筑物附近进行“贴脸攻击”,但是对于重兵防守的 基地单位,如果不能将对方战队地面单位消耗殆尽并且推掉哨兵机器人,很难有机会打 到基地,且伤害值较小。而英雄机器人发射的大弹丸可对建筑物造成可观的伤害值,是重 要的攻城单位,发挥其远程吊射基地和前哨站的优势,便可轻松突破重围对敌方建筑物 造成重击,相当于“第二飞镖”。
            \item 在战局中英雄机器人可采用近战方式在通过敌方哨兵机器人的火力区域后对敌方基地进 行打击,并且对英雄机器人来说,除吊射外第二个突破重围实现攻城的方式是飞坡,通过 飞坡避开哨兵的火力范围,直接进入对面基地区域进行打击。
        \end{itemize}
    
    \paragraph{功能需求分析}
    
    \paragraph{改进方向}

    \paragraph{研发进度安排}

    \paragraph{项目组人员分配}
    