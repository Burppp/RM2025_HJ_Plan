\subsubsection{电控组}

    \paragraph{现有技术优化}
    
        \setlist[itemize]{label=\raisebox{-1.2ex}{\scalebox{3}{$\textbullet$}}}
    
        \begin{itemize}
            \item 技术点:PID控制算法
            \item 优化目标:依据自动控制原理相关知识对PID算法做优化和改进,如加入前馈控制器或者积分分离,使PID控制器的控制效果更加稳定,控制系统的响应更加快速。
            \item 技术点:IMU数据处理
            \item 优化目标:消除绝大部分的IMU数据噪声,使IMU数据更加可靠,通过IMU数据解算得到的云台空间位姿可以收敛在较为准确的位置。
            \item 技术点:路径规划
            \item 优化目标:目前已经掌握了三维导航与重定位技术,后续目标为完善导航时的实时避障功能,以更好的适应真实情境下的导航需求。
            \item 技术点:掉线检测
            \item 优化目标:完善已有的模块掉线看门狗的功能,可以在模块掉线的时候以外界更可感的形式(如蜂鸣器、LED灯等)让维护人员得知是哪一个模块掉线以便利维护工作。
            \item 技术点:底层通讯链路
            \item 优化目标:在尽可能保证正常通信和通信端口负载不过高的情况下减少通信链路中节点(主控板和从控板)的数量,缩短通讯延迟,提高资源利用率。
            \item 技术点:行为树
            \item 优化目标:目前已实现通过behavior trees实现机器人的行为规划,而且可以图形化查看机器人目前的行为状态,为适应真实情境需要设计更加完备的行为树以让哨兵的决策更加智能,在变化更频繁的情况下更加游刃有余。
            \item 技术点:无线调参技术
            \item 优化目标:使用外置模块实现无线的实时参数调整,让调参效果更加直观,便于调试,缩短调试环节的时间。
            \item 技术点:工程机械臂自定义控制器
            \item 优化目标:完善自定义控制器,便于操作手控制机械臂进行取兑矿操作。
            \item 技术点:LQR控制
            \item 优化目标:调整Q、R矩阵,使得轮腿步兵兼顾收敛快和功耗低的优点。
        \end{itemize}

    \paragraph{技术发展计划}

        \setlist[itemize]{label=\raisebox{-1.2ex}{\scalebox{3}{$\textbullet$}}}

        \begin{itemize}
            \item 兵种:哨兵机器人
            \item 技术点:决策树
            \item 发展目标:使用机器学习和强化学习算法帮助哨兵决策,研发哨兵机器人的自生成行为树以实现实时评估所处情境做出最优决策
            \item 兵种:哨兵机器人
            \item 技术点:与雷达之间的通讯接口
            \item 发展目标:实现与雷达之间的通讯接口,哨兵可以接收雷达反馈场上信息辅助行为决策
            \item 兵种:哨兵机器人
            \item 技术点:ROS2移植
            \item 发展目标:将哨兵现有的ROS导航框架移植到ROS2
            \item 兵种:全体兵种
            \item 技术点:现有控制代码框架的底层模块的进一步封装
            \item 发展目标:维护代码可读性,简化调用底层底层模块
            \item 兵种:工程机器人
            \item 技术点:下位机实现机械臂正运动学解算
            \item 发展目标:在下位机上实现机械臂的正运动学解算和控制
            \item 兵种:飞镖系统
            \item 技术点:镖体制导
            \item 发展目标:通过视觉信息和传感器数据调整飞镖飞行时姿态与轨迹
            \item 兵种:工程机器人
            \item 技术点:无线充电
            \item 发展目标:可以对步兵、英雄、哨兵机器人的电容实现无线充电
            \item 兵种:步兵、英雄、哨兵机器人
            \item 技术点:超级电容及控制器
            \item 发展目标:实现超级电容的动能回收
        \end{itemize}