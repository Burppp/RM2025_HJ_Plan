\subsubsection{哨兵机器人}

    \paragraph{规则分析}

        \setlist[itemize]{label=\raisebox{-1.2ex}{\scalebox{3}{$\textbullet$}}}

        \begin{itemize}
            \item 哨兵在这个赛季的机制改动相当大,主要的改动在于哨兵取消了与前哨站挂钩的无敌机制、哨兵与比赛胜负的判定机制以及哨兵巡逻区的取消。此外,哨兵的初始发弹量也从400发削减到了300发,而且虽然底盘功率上限未做改动,但是哨兵和步兵、英雄一样适用于全局20000J底盘总能量的约束。这些改动无疑削弱了哨兵的单兵作战能力,也进一步提高了哨兵对完善的导航避障、自身定位、全局规划、灵活决策、资源调度和与其他机器人打配合的需求。
            \item 哨兵在这个赛季的战场上也依旧存在着不小的优势:首先,这个赛季并未对哨兵的枪口热量上限、每秒冷却做改动,这保证了哨兵在可发弹量充足的情况下输出能力依旧恐怖,而且,每一分钟哨兵都可以通过导航到补给区补血补弹或者远程兑弹的方式补充发弹量,这些机制平衡了哨兵初始发弹量削减的影响,在导航完善和工程获取经济能力强的情况下哨兵的发弹量甚至可能只取决于预装的弹丸有多少;其次,哨兵虽然取消了与前哨站挂钩的无敌状态,但是初始数据400血量100功率上限(步兵满级数据)保证了哨兵前期的性能和生存能力在线,配合恐怖的火力,在前期哨兵在面对对面的地面输出单位是具有绝对的压制力的,而且哨兵可以远程兑换血量以及它的复活机制并未改变,这意味着原本弹尽粮绝、性命垂危的哨兵可能突然以一个较好的状态重返赛场,对敌方形成巨大的威胁;况且,这个赛季新增了哨兵可以占领的堡垒增益点,当堡垒增益点开启,哨兵占领之后依旧可以获得无敌、额外发弹量和额外的枪口冷却增益,这一改动大大增强了哨兵的防守能力。
            \item 这个赛季的机制变动使哨兵更加全能,不论是在进攻端还是在防守端都有着独特的优势。显然,本赛季的改动无疑是令哨兵机器人的技术栈更加接近于全自动机器人的本质。没有一个完善的决策体系的话,在全局资源的规划和调动上会出现不可忽视的问题;巡逻区的取消无疑是对哨兵如何及时的规划进攻和回防路线将会是导航和定位上的挑战。
        \end{itemize}
    
    \paragraph{策略战术}

        \setlist[itemize]{label=\raisebox{-1.2ex}{\scalebox{3}{$\textbullet$}}}
    
            \begin{itemize}
                \item 哨兵的前期作战能力是所有地面机器人中最强的,在前期利用高性能迅速占领中央高地上的增益点的战术价值很高,不论是前压前哨站还是阻击对面攻势都会令对手很头疼。
                \item 哨兵可以在能量机关激活的时候击打能量机关获得额外增益。
                \item 哨兵能远程补弹,可以在自身增益足够高或敌方状态差时乘胜追击,或是在被动防御时 尝试突围、扭转局势。
                \item 哨兵可以通过雷达通讯接口得到鸟瞰全局的数据,通过自主决策可以选择进攻、防守、配合其他机器人或者规避围剿。
                \item 哨兵可以占据中央高地,对靠近高地的机器人进行阻击和对取矿的工程机器人进行拦截。
                \item 哨兵能接收云台手发送指令,在必要的时候可以解决一些意料之外、情况紧急的情况。
            \end{itemize}
    
    \paragraph{功能需求分析}

    \LTXtable{\textwidth}{Sentry/2.3_functionalRequirement_analysis.tex}
    
    \paragraph{改进方向}

    \LTXtable{\textwidth}{Sentry/2.3_improvementDirection.tex}

    \paragraph{研发进度安排}

    \LTXtable{\textwidth}{Sentry/2.3_developmentSchedule.tex}

    \paragraph{项目组人员分配}

    \LTXtable{\textwidth}{Sentry/2.3_personalAssignment.tex}
    