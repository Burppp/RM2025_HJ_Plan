\subsubsection{视觉组}

    \paragraph{技术储备规划}

        \setlist[itemize]{label=\raisebox{-1.2ex}{\scalebox{3}{$\textbullet$}}}

        \begin{itemize}
            \item 优化识别速度:在原来的代码基础框架下对代码进行优化,实现更高的识别速度,提高自瞄帧率
            \item 能量机关击打:本赛季可以在更多的位置上击打能量机关,自瞄算法需要针对多角度的识别以及瞄准做出调整
            \item 弹道运动轨迹计算:弹道解算模型可以更精确,提升自瞄准度
            \item 针对各兵种的适配优化:自瞄代码在步兵上的准度较低,而在固定云台基座的无人机上效果良好;而将来无人机在空中的工况也会和地面的固定位置射击有不同。也就是说代码需要针对个兵种的云台稳定性以及响应速度做出相应的优化
        \end{itemize}
        
    \paragraph{技术发展计划}

        \setlist[itemize]{label=\raisebox{-1.2ex}{\scalebox{3}{$\textbullet$}}}

        \begin{itemize}
            \item 弹道偏移校正:弹道会因为各种原因偏移原来的预定轨迹,而人工调准又十分繁琐耗时,现在需要一套算法来识别弹道的轨迹和落点来自动修正枪口的射击方向
            \item 场上手动弹道校正:由于场下调整的弹道在场上不一定适用,而弹道自动修正难度较大,需要写一个场上手动校正弹道的程序作为保底方案
            \item 全向感知:继续完成哨兵的全向感知的研发,在条件允许的情况下进行上车调试
            \item 避障与重定位:由于没有很好的初始位姿校准机制,没有精确摆放的哨兵导航路线会向一个方向偏移,现在需要重定位机制来修正哨兵的初始位姿;由于上赛季没有避障机制,本赛季需要添加
            \item 点云聚类:通过雷达点云聚类来识别该点云是否是敌方机器人的图像,可以作为哨兵全向感知的补充
            \item 镖架制导:通过识别前哨站和基地绿灯实现飞镖镖架的自动瞄准
            \item 视觉辅助兑矿:通过视觉算法结算兑矿站的位姿,辅助工程机器人进行矿石兑换
        \end{itemize}