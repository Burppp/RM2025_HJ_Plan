\subsubsection{空中机器人}

    \paragraph{规则分析}

        \setlist[itemize]{label=\raisebox{-1.2ex}{\scalebox{3}{$\textbullet$}}}

        \begin{itemize}
            \item 空中机器人相较于上个赛季改动最大的是再比赛开始即可起飞,起飞后可以通过花钱的方式续费,这无疑是对空中机器人的大增强,首先就是对于哨兵和英雄存在极其强大的压制力,并且对于大部分队伍也能开局清除前哨站。
            \item 空中机器人的限重减少,这一改动直接影响了空中机器人的设计,并且取消了中途补弹,如果想发挥空中机器人的最大优势,首先便是需要将空中机器人的整体做轻,并且是弹舱的设计要能容纳的下最少1500发弹,对于很多队伍来说相当于只能重新设计一台无人机。
        \end{itemize}
    
    \paragraph{策略战术}

        \setlist[itemize]{label=\raisebox{-1.2ex}{\scalebox{3}{$\textbullet$}}}

        \begin{itemize}
            \item 前期刚开局,压制英雄前期的吊射,英雄的改动使得其可以随处吊射,对于前哨站来说也是非常有威胁,如果不针对英雄便会与上赛季一样,英雄轻易吊射,并且新增的43°高坡使得步兵更难第一时间针对对方英雄,因此空中机器人前期可以采取压制哨兵与英雄的打法,如果对方并没有很强势的英雄与哨兵则可以优先打击前哨站。
            \item 在中期时,空中机器人主要起到帮助己推进或是防守的角色,在对方英雄,步兵处于优势位置时,可以优先针对,并且对于信息的获取也是一大关键,空中机器人也能开能量机关为团队进一步提供帮助。
            \item 后期可能会遇见两方的功率消耗殆尽,或者难以攻入对方,亦或者是基地/前哨/哨兵,血量接近此时无人机如果还能有电量留存则可以起飞去消磨对方的血量,拿下关键点。
        \end{itemize}
    
    \paragraph{功能需求分析}

    \LTXtable{\textwidth}{Drone/2.3_functionalRequirement_analysis.tex}
    
    \paragraph{改进方向}

    \LTXtable{\textwidth}{Drone/2.3_improvementDirection.tex}

    \paragraph{研发进度安排}

    \LTXtable{\textwidth}{Drone/2.3_developmentSchedule.tex}

    \paragraph{项目组人员分配}

    \LTXtable{\textwidth}{Drone/2.3_personalAssignment.tex}
    