%%% 本文档作为功能演示

\subsubsection{步兵机器人}

    %%% 长段演示-BEGIN
    第一段第一段第一段第一段第一段第一段第一段第一段第一段第一段第一段第一段第一段第一段第一段第一段第一段第一段第一段第一段。\par
    
    第二段第二段第二段第二段第二段第二段第二段第二段第二段第二段第二段第二段第二段第二段第二段第二段第二段第二段第二段第二段。\par
    
    第三段第三段第三段第三段第三段第三段第三段第三段第三段第三段第三段第三段第三段第三段第三段第三段第三段第三段第三段第三段。\par
    %%% 长段演示-END

    \paragraph{需求分析}
    
        %%% 有序列表演示-BEGIN
        \begin{enumerate}
            \item 需求1。
        
            \item 需求2。
        \end{enumerate}
        %%% 有序列表演示-END
    
        %%% 插入公式演示-BEGIN
        \begin{align*}
            \Vec{OP} &= \frac{1}{2} ( \Vec{OB} + \Vec{OD} ) = \frac{1}{2} ( \Vec{OA} + \Vec{AB} + \Vec{OE} + \Vec{ED} ) \\
            |PC| &= \sqrt{|BC|^2 - |BP|^2} \\
            \Vec{OC} &= \Vec{OP} + |PC|\hat{PC} , ~{where}~ \hat{PC} \perp \hat{BP} \\
            \Rightarrow l &= |OC| \\
            \Rightarrow \varphi_0 &= \pi + \arctan(z_C, x_C) \\
        \end{align*}
        %%% 插入公式演示-END
        %%% 具体公式具体分析,公式排版很复杂,是一门学问
    
    \paragraph{设计思路}
        
        %%% 插入图片演示-BEGIN
        \begin{figure}[H]
            \centering
            \includegraphics[height=0.35\textwidth]{figure/2_test1.png}
            \hspace{0.5em}
            \includegraphics[height=0.35\textwidth]{figure/2_test2.png}
            \label{fig:test}
        \end{figure}
        %%% 插入图片演示-END
        %%% 图片居中放置
        %%% 该图片存储路径为 ./figure/2_test1.png 以及 ./figure/2_test2.png两张图
        %%% \label表示的是该图坐标位置,可以由其他地方引用跳转

    \paragraph{任务安排}
    
        %%% 插入图片引用演示-BEGIN
        点击此处跳转到四舵轮底盘图片处:
        \ref{fig:test}
        %%% 插入图片引用演示-END
        %%% fit:后面的内容为图片对应的label
        
        点击此处跳转到大疆官方比赛配件网站:
        %%% 插入网址引用演示-BEGIN
        \href{https://www.robomaster.com/}{RM官网}
        %%% 插入网址引用演示-END

        %%% 无需缩进的时候可以加一句整个
        \noindent
        
        %%% 插入表格演示-BEGIN
        \LTXtable{\textwidth}{table/2_插入表格示例.tex}
        %%% 插入表格演示-END
        %%% 该表格存储路径为 table/2_插入表格示例.tex ,表格相关代码的具体内容可以打开该文件查看