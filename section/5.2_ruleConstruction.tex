\subsection{规章制度建设}

    \subsubsection{规章制度建设重点分析}

        robomaster这个比赛相比其他比赛,投入时间长,压力大,而且虽然对成员的锻炼很大但是投入产出比相比于其他比赛还是稍显逊色,所以成员个人对于比赛的投入程度主要还是依靠热爱和对技术的不懈追求,并且优秀的人才都是自驱的,所以队伍的规章制度较为开放轻松,但是这主要面向自觉的队员,对于不自觉且态度不端正的队员,实行相应的惩罚,如果屡教不改,则在不影响队伍进度的前提下,实行冷处理,不派发核心任务,所以规章制度建设的重点就在于对于自觉者来说可以很相对自如,而对于不自觉者而言可能倍感压力。\par
        且团队毕竟是比赛形式和科技社团性质,按理来说不应当过度干涉队员的个人生活,所以规章制度上最严重的处罚也仅仅是离队,并不能够解决实际进度遇到的问题。所以规章制度对于不自觉的成员更倾向于更换主力开发、回收工位、下放梯队等措施,保留该成员在战队中的知情权和参与开发的权限,但不给予核心任务、降低团队对该成员的期望,制度相较灵活,进退留有余量。\par

    \subsubsection{规章制度落地方案分析}

        \setlist[itemize]{label=\raisebox{-1.2ex}{\scalebox{3}{$\textbullet$}}}

        \begin{itemize}
            \item 实行开放的打卡考勤制度,对于进度可查,且经常出现在实验室者,不做硬性要求,对于进度模糊,而时长不出现在实验室者,考勤便可以作为考察依据;
            \item 对于物资管理,由于低级错误或者恶意毁坏高价值物资者,将根据该物品市场价的百分之二十赔偿,第二次毁坏则增加百分之二十,以此类推,实行至今,此事件出现几率减少近百分之八十,效果显著;
            \item 制定电机出入库记录表,有效避免电机随意放在地上、甚至电机消失不见的现象;
            \item 对于进度,制定阶段性目标,负责人参与主力开发任务,日常监督进度,阶段性考核,动态调整项目进度,效果良好;
            \item 队长、项管长期在实验室,和参赛队员坐在一起,能够方便、有效地视察进度,效果显著;
            \item 制定“警钟长鸣”文档,记录各种物件的损坏,警示所有队员不要犯同样的错误,并记录某些重要部件的使用寿命;
            \item 继续推动进度公示制度,给每一个车组建立进度规划表格,负责人长期维护表格,组员自觉新增项目和更新进度,但组员总是忘记新增项目和更新进度,制度推进效果差;
            \item 优化财务流程,由技术组组长分摊一部分财务压力,负责财务的队员只需要汇总发票、对接指导老师和学院方;
            \item 完善实验室卫生值日表,将实验室卫生责任落实到个人,实践效果可以应付上级的检查,日常卫生仍然需要提醒,值日人通常比较乐意整理实验室;
            \item 完善对外宣传值日表,将成员分组轮流接待各种宣传需要,扩大战队和赛事文化的知名度,实践效果非常有效;
            \item 完善会议记录,战队大会、技术组或运营组组会、车组组会,留下会议记录,写各类文档时能派上用场;
            \item 大体使用飞书平台互相交流、进度管理,将成员飞书活跃度作为衡量其贡献度的维度之一;
            \item 整顿实验室游戏现象,在未完成个人任务时在实验室玩游戏,接到举报后当即惩罚该成员请在场所有人喝奶茶,有效减少游戏现象;
            \item 后期紧张备赛阶段日更新各车组每日进度。
            \item 管理层内部共同维护一个待办清单表格,防止个别成员遗忘进度;
            \item 管理层、负责人和指导老师每两周开一次例会,可以及时向指导老师反馈当前进度和面临的问题。
        \end{itemize}

    \subsubsection{规章制度落地闭环分析}

        \setlist[itemize]{label=\raisebox{-1.2ex}{\scalebox{3}{$\textbullet$}}}

        \begin{itemize}
            \item 某成员长时间缺勤不打卡,队长、项管进行约谈,约谈后出勤情况有明显改善;
            \item 打卡制度后队员到勤率有明显上升;
            \item 物资管理办法施行后,乱丢工具、耗材的现象有明显改善,取用电机、开发板等规范化;
            \item 负责人、队长、项管和其他队员物理距离的拉近能够显著提升进度监察的效率;
            \item “警钟长鸣”文档后可以明晰哪些问题是备赛过程中常常遇见的,数据积累中;
            \item 进度公示制度推行困难,队员难以养成自觉新增项目和更新进度的协管;
            \item 优化财务流程后,财务负责人有更多的时间和精力确保报账准确无误;
            \item 实验室卫生值日表可以督促值日人整理实验室,但常常需要因为发现某些地方脏乱后经人提醒才来收拾;
            \item 对外宣传值日表将宣传责任明确地分配到每一个组,不会出现没有人愿意去或一群人抢着去的情况,效果良好;
            \item 例会通常不会写会议记录,组会、大会会写会议记录,提供给以后作为进度参考依据;
            \item 使用飞书平台管理,功能齐全、后台能够收集很多队员的日常数据,个别队员仍然无法及时关注到飞书消息;
            \item 整顿实验室游戏现象后明显减少了游戏现象;
            \item 备赛日更新进度还未开始进行;
            \item 管理层待办清单通常由队长一人新增任务,提醒其他管理层成员任务截止日期;
            \item 指导老师例会确保了财务安全、进度清晰。
        \end{itemize}