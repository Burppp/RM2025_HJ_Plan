%%% 本文档作为功能演示

%%% 表格演示-BEGIN
\begin{longtable}{ p{2cm} | p{7.8cm} | p{6cm} |}

    \hline

    %%% 此处标识以上为表格脚,即在每页的表格底部重复显示的内容(一条横线)
    \endfoot
    
    %%% 此处标识该行颜色为指定定义的颜色,tabhdcolor 在 ./introduction.tex 中定义,颜色为hsb(0, 0, 0.82353),即82度灰
    \rowcolor{tabhdcolor}

        %%% 第1行的表格信息
        \begin{center}
            功能
        \end{center} &
        \begin{center}
            需求分析
        \end{center} &
        \begin{center}
            设计思路
        \end{center} \\

    %%% 此处完整分割一行
    \hline

    %%% 此处标识以上为表格头,即在每页的表格头部重复显示的内容(一堆header)
    \endhead

        %%% 第2行的表格信息
        \begin{center}
            高机动性,可飞坡的底盘结构
        \end{center} &
        \begin{center}
            高机动性可以使得英雄机器人较难被敌方锁定,自身较快的移动速度也可以提高进攻的频率,给予敌方更大的威胁。可飞坡可以使英雄机器人可以绕后进攻,在偷袭基地与快速回归战场上有重要的战略意义
        \end{center} &
        \begin{center}
            轻量化主要是使用 SolidWorks 进行受力的分析,在拓扑算例和仿真的基础上进 行合理镂空,以达到减重的目的轻量化设计,降低英雄机器人的重量通过优化设计降低英雄机器人的云台高度使重心更接近底盘,提高飞坡后的稳定性缩减底盘的宽度,使飞坡时更容易调整姿态,加快比赛时的飞坡速度,提高飞坡的成功率
        \end{center} \\
        
    \hline

        \begin{center}
            适应各种地形,下台阶,过隧,过盲道的悬挂系统
        \end{center} &
        \begin{center}
            可以适应多种地形是保证英雄机器人稳定的重要前提。能适应多种地形也可以很大程度的提高英雄机器人的核心竞争力
        \end{center} &
        \begin{center}
            为了防止下台阶时底盘受到磕碰,可以在底盘增加两排导轮为了能通过隧道,机器人应该在能实现该有的功能的前提下尽可能压缩尺寸,设计高度集成化,比如避震器和自适应连杆可以做成一体,一些地方可以做成一块板材多种用途
        \end{center} \\
        
    \hline
    
        \begin{center}
            外壳快拆设计 
        \end{center} \cellcolor{gndcolor} &
        \begin{center}
            因为原先的保护壳并没有使用快拆的设计,导致在平常测试与在赛场维修时出现了较多的不便,带来了不必要的时间浪费
        \end{center} \cellcolor{gndcolor} &
        \begin{center}
            设计上队内先进行沟通,先是电控和视觉来提出需要较常拆装和维修的位置,再由机械来设计出快拆功能;快拆主要使用了合页与弹簧搭扣的配合设计
        \end{center} \cellcolor{gndcolor} \\

    %%% 此处分割一行的第2至3个单元格
    \hline
    %\cline{2-3}
    
        \begin{center}
            
        \end{center} &
        \begin{center}
            
        \end{center} &
        \begin{center}
            
        \end{center} \\
        
    \hline
    
        \begin{center}
            
        \end{center} &
        \begin{center}
            
        \end{center} &
        \begin{center}
            
        \end{center} \\

    %%% 此处分割一行的第1至1个单元格,即只有第一个单元格下面有横线
    %\cline{1-1}
    \hline
    
        \begin{center}
            
        \end{center} &
        \begin{center}
            
        \end{center} &
        \begin{center}
            
        \end{center} \\
        
    \hline
    
        \begin{center}
            
        \end{center} &
        \begin{center}
            
        \end{center} &
        \begin{center}
            
        \end{center} \\
        
    \hline
    
        \begin{center}
            
        \end{center} &
        \begin{center}
            
        \end{center} &
        \begin{center}
            
        \end{center} \\
        
    \hline
    
        \begin{center}
            
        \end{center} &
        \begin{center}
            
        \end{center} &
        \begin{center}
            
        \end{center} \\
        
    \hline
    
        \begin{center}
            
        \end{center} &
        \begin{center}
            
        \end{center} &
        \begin{center}
            
        \end{center} \\
        
    \hline
    
\end{longtable}
%%% 表格演示-END
%%% 其中,开头的 { X | X | X | X |} 中,竖线标识分割线,字母X为自动对齐,其余可替换的lrc分别为左对齐、右对齐、居中对齐
%%% 该表格的一大特性是跨页后仍保留表头内容,即 \endhead 前的内容