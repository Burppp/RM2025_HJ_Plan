%%% 本文档作为功能演示

%%% 表格演示-BEGIN
\begin{longtable}{ p{2cm} | p{7.8cm} | p{6cm} |}

    \hline

    %%% 此处标识以上为表格脚,即在每页的表格底部重复显示的内容(一条横线)
    \endfoot
    
    %%% 此处标识该行颜色为指定定义的颜色,tabhdcolor 在 ./introduction.tex 中定义,颜色为hsb(0, 0, 0.82353),即82度灰
    \rowcolor{tabhdcolor}

        %%% 第1行的表格信息
        \begin{center}
            功能
        \end{center} &
        \begin{center}
            需求分析
        \end{center} &
        \begin{center}
            设计思路
        \end{center} \\

    %%% 此处完整分割一行
    \hline

    %%% 此处标识以上为表格头,即在每页的表格头部重复显示的内容(一堆header)
    \endhead

        %%% 第2行的表格信息
        \begin{center}
            对移动装甲板的识别与攻击
        \end{center} &
        \begin{center}
            稳定的pitch轴和yaw 轴,稳定的初速度和连发功能。发射架可识别基地装甲板的移动并自主调节pitch和 yaw 轴,并且飞镖轨迹稳定。发射架的识别功能比飞镖可以有更大的摄像头和更强的计算性能。重量冗余更大,复用成本更低,损坏几率较小
        \end{center} &
        \begin{center}
            在飞镖架上增加运算端和视觉摄像头,视觉通过识别装甲板绿灯来定位基地装甲板。判断位差后通过串口通信给电机发送需要移动的角度
        \end{center} \\
        
    \hline

        \begin{center}
            稳定的飞镖体与小范围的落点散布
        \end{center} &
        \begin{center}
            通过研究微型飞行器的气动外形设计与重心压心的分配,纯机械镖需拥有较高的静稳定性,制导镖需拥有较强的机动性
        \end{center} &
        \begin{center}
            查阅文献与理论学习,了解气动设计需考虑的因素,如弹翼、弹身、尾翼等部分的设计,在设计好镖体后通过流体仿真观察是否与设计的思路一致
        \end{center} \\
        
    \hline
    
        \begin{center}
            可稳定且迅速标定前哨站与基地的方位 
        \end{center} \cellcolor{gndcolor} &
        \begin{center}
            这个赛季飞镖发射台是从后方开放放入飞镖系统,因此在赛前准备时间3分钟内不能人工标定前哨站与基地,但在裁判自检时有5s时间打开飞镖舱门,在这短时间可利用飞镖架视觉标定前哨站与基地,记录各自方位,因此飞镖架视觉制导在这个赛季依旧拥有很高的重要性
        \end{center} \cellcolor{gndcolor} &
        \begin{center}
            在飞镖架增加运算端和视觉摄像头,视觉通过识别装甲板绿灯来定位前哨站与基地的装甲板,并记录各自位置,通过串口通信给电机发送瞄准前哨站与基地所需要移动的角度
        \end{center} \cellcolor{gndcolor} \\

    %%% 此处分割一行的第2至3个单元格
    \hline
    %\cline{2-3}
    
        \begin{center}
            可稳定发射飞镖体的飞镖发射架
        \end{center} &
        \begin{center}
            摩擦轮发射的飞镖发射架具有较多不稳定因素,因此这个赛季的飞镖发射架采用拉簧弹射的方式,为飞镖体提供固定但可变的发射能量
        \end{center} &
        \begin{center}
            选用型号为4*40*450的闭口钩拉簧,利用动滑轮连接发射座,通过丝杆改变扳机位置从而改变飞镖体的蓄能
        \end{center} \\
        
    \hline
    
        \begin{center}
            制导镖体
        \end{center} &
        \begin{center}
            虽然无控制飞镖能完成对基本固定目标的打击,但对于更高收益的“随机固定目标”或“随机移动目标“的精度需求还是比较缺乏。对于可控的制导飞镖体,就可以通过姿态控制实现误差调整,提高飞镖体的精度
        \end{center} &
        \begin{center}
            给镖体装载imu获取镖体姿态,通过四元数及卡尔曼滤波实现准确姿态数据的读取,解算姿态后控制镖体的机翼,尾翼实现姿态控制,利用该功能可以控制镖体上升期的俯仰稳定性和横向稳定性,并且受瞬时干扰后具有恢复到原来平衡状态的能力。对于下降期,结合长焦摄像头,识别目标位置,可以微调姿态
        \end{center} \\
        
    \hline
    
\end{longtable}
%%% 表格演示-END
%%% 其中,开头的 { X | X | X | X |} 中,竖线标识分割线,字母X为自动对齐,其余可替换的lrc分别为左对齐、右对齐、居中对齐
%%% 该表格的一大特性是跨页后仍保留表头内容,即 \endhead 前的内容