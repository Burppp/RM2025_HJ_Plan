%%% 本文档作为功能演示

%%% 表格演示-BEGIN
\begin{longtable}{ p{1.5cm} | p{3cm} | p{6cm} | p{4.3cm} |}

    \hline

    %%% 此处标识以上为表格脚,即在每页的表格底部重复显示的内容(一条横线)
    \endfoot
    
    %%% 此处标识该行颜色为指定定义的颜色,tabhdcolor 在 ./introduction.tex 中定义,颜色为hsb(0, 0, 0.82353),即82度灰
    \rowcolor{tabhdcolor}

        \begin{center}
            组别
        \end{center} &
        \begin{center}
            改进对象
        \end{center} &
        \begin{center}
            改进内容
        \end{center} &
        \begin{center}
            验收指标
        \end{center}\\

    %%% 此处完整分割一行
    \hline

    %%% 此处标识以上为表格头,即在每页的表格头部重复显示的内容(一堆header)
    \endhead

        %%% 第2行的表格信息
        \begin{center}
            %\multirow{3}{*}{机械组}
            机械组
        \end{center} &
        \begin{center}
            飞镖底盘
        \end{center} &
        \begin{center}
            pitch轴从可变为固定,减小不不稳定因素,利用丝杆驱动yaw轴,增加稳定性
        \end{center} &
        \begin{center}
            底盘可稳定改变yaw轴,固定pitch轴可提高飞镖上层结构稳定性
        \end{center}\\
        
    \hline
        \begin{center}
            机械组
        \end{center}&
        \begin{center}
            飞镖发射架
        \end{center}&
        \begin{center}
            改为拉簧式弹射型发射架
        \end{center}&
        \begin{center}
            利用蓄能能量稳定的优势,减小飞镖体落点散布,增加飞镖发射的稳定性
        \end{center}\\
        
    \hline

        \begin{center}
            机械组
        \end{center}&
        \begin{center}
            飞镖体
        \end{center}&
        \begin{center}
            更换镖体材料为TPU-95a,增加镖体气动性能,制作制导镖体
        \end{center}&
        \begin{center}
            增加飞镖体寿命;确保镖体有高度的一致性;增加非制导镖体在飞行过程的稳定性;增加制导镖体在飞行过程的机动性
        \end{center}\\
        
    \hline
    
        %%% 第4行的表格信息
        %%% 第4行开始,连续2行的第1列单元格被合并,自动列宽
        %\multirow{3}{*}{电控组} &
        \begin{center}
            机械组
        \end{center} &
        \begin{center}
            飞镖装填系统
        \end{center} &
        \begin{center}
            改用了全新的换弹机构,使得装填部分可以适配更多形态的飞镖,采用更为轻便的材料
        \end{center} &
        \begin{center}
            提高装填速度和稳定性,确保能在开启闸门的时间里完成稳定的1~2次装填
        \end{center} \\

    \hline
    %%% 此处分割一行的第2至3个单元格
    %\cline{2-3}
    
        %%% 第5行的表格信息
        %%% 第5行由于第1列在上一行已被合并,所以建议不要写入信息,虽然写入信息也会显示出
        \begin{center}
            电控组
        \end{center} &
        \begin{center}
            飞镖架yaw轴稳定
        \end{center} &
        \begin{center}
            在飞镖发射瞬间,减小弹簧震动对飞镖yaw轴带来的影响
        \end{center} &
        \begin{center}
            将震动带来的角度偏差影响缩小到0.1度以下
        \end{center} \\

    \hline

       \begin{center}
           电控组
       \end{center} &
       \begin{center}
           换弹系统
       \end{center} &
       \begin{center}
           调整换弹系统的GM6020电机pid参数,优化pid算法,减少电机旋转到指定位置时的抖动和偏差
       \end{center} &
       \begin{center}
           卡弹率不超过5%
       \end{center} \\
        
    \hline
    
        %%% 第6行的表格信息
        %\multirow{3}{*}{视觉组} &
        \begin{center}
            电控组
        \end{center}&
        \begin{center}
            制导镖体
        \end{center}&
        \begin{center}
            给镖体装载imu获取镖体姿态,解算姿态后控制镖体的机翼,尾翼实现姿态控制,并且受瞬时干扰后具有恢复到原来平衡状态的能力
        \end{center}&
        \begin{center}
            通过四元数及卡尔曼滤波实现准确姿态数据的读取,结合长焦摄像头,识别目标位置,做到击打基地随机移动靶
        \end{center}\\

    %%% 此处分割一行的第1至1个单元格,即只有第一个单元格下面有横线
    %\cline{1-1}
    \hline
    
        %%% 第7行的表格信息
        %content 7 1 &
        %%% 第7行由于第2~4列在上一行已被合并,所以建议不要写入信息,虽然写入信息也会显示出,而且建议像下面一样合并表格防止竖线分割
        %\multicolumn{3}{| c }{} \\

       \begin{center}
           视觉组
       \end{center} &
       \begin{center}
           引导灯识别与定位
       \end{center} &
       \begin{center}
           优化识别模型,提高识别速率。对引导灯坐标进行滤波
       \end{center} &
       \begin{center}
           实现实时性识别,识别不抖动
       \end{center} \\

    \hline
    
\end{longtable}
%%% 表格演示-END
%%% 其中,开头的 { X | X | X | X |} 中,竖线标识分割线,字母X为自动对齐,其余可替换的lrc分别为左对齐、右对齐、居中对齐
%%% 该表格的一大特性是跨页后仍保留表头内容,即 \endhead 前的内容