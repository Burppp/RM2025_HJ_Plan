%%% 本文档作为功能演示

%%% 表格演示-BEGIN
\begin{longtable}{ p{1.5cm} | p{3cm} | p{6cm} | p{4.3cm} |}

    \hline

    %%% 此处标识以上为表格脚,即在每页的表格底部重复显示的内容(一条横线)
    \endfoot
    
    %%% 此处标识该行颜色为指定定义的颜色,tabhdcolor 在 ./introduction.tex 中定义,颜色为hsb(0, 0, 0.82353),即82度灰
    \rowcolor{tabhdcolor}

        \begin{center}
            组别
        \end{center} &
        \begin{center}
            改进对象
        \end{center} &
        \begin{center}
            改进内容
        \end{center} &
        \begin{center}
            验收指标
        \end{center}\\

    %%% 此处完整分割一行
    \hline

    %%% 此处标识以上为表格头,即在每页的表格头部重复显示的内容(一堆header)
    \endhead

        %%% 第2行的表格信息
        \begin{center}
            %\multirow{3}{*}{机械组}
            机械组
        \end{center} &
        \begin{center}
            机架结构
        \end{center} &
        \begin{center}
            使用管结构,而不是层结构
        \end{center} &
        \begin{center}
            刚度足够,能正常飞行·,重量达标整体10kg以内
        \end{center}\\
        
    \hline
        \begin{center}
            机械组
        \end{center}&
        \begin{center}
            云台结构
        \end{center}&
        \begin{center}
            改进为直驱结构
        \end{center}&
        \begin{center}
            反复测试能够稳定,不发生疲劳等问题
        \end{center}\\
        
    \hline

        \begin{center}
            机械组
        \end{center}&
        \begin{center}
            发射机构
        \end{center}&
        \begin{center}
            枪管限位改进
        \end{center}&
        \begin{center}
            能够稳定单发,连续发射
        \end{center}\\
        
    \hline
    
        %%% 第4行的表格信息
        %%% 第4行开始,连续2行的第1列单元格被合并,自动列宽
        %\multirow{3}{*}{电控组} &
        \begin{center}
            电控组
        \end{center} &
        \begin{center}
            云台控制
        \end{center} &
        \begin{center}
            从丝杆换成达妙4310直驱
        \end{center} &
        \begin{center}
            获得良好的pitch轴控制,不卡限位
        \end{center} \\

    \hline
    %%% 此处分割一行的第2至3个单元格
    %\cline{2-3}
    
        %%% 第5行的表格信息
        %%% 第5行由于第1列在上一行已被合并,所以建议不要写入信息,虽然写入信息也会显示出
        \begin{center}
            电控组
        \end{center} &
        \begin{center}
            guidance部署
        \end{center} &
        \begin{center}
            使用视觉定高方案
        \end{center} &
        \begin{center}
            相比上赛季的定位可以更稳定
        \end{center} \\

    \hline

       \begin{center}
           电控组
       \end{center} &
       \begin{center}
           飞控调参与尝试xy轴定位方案
       \end{center} &
       \begin{center}
           对飞机进行调试,根据重量限制分析是否增加雷达作为xy轴定位
       \end{center} &
       \begin{center}
           飞行稳定
       \end{center} \\
        
    \hline
    
        %%% 第6行的表格信息
        %\multirow{3}{*}{视觉组} &
        \begin{center}
            视觉组
        \end{center}&
        \begin{center}
            自瞄
        \end{center}&
        \begin{center}
            适配空中机器人工况的自瞄
        \end{center}&
        \begin{center}
            稳定击打前哨站
        \end{center}\\

    %%% 此处分割一行的第1至1个单元格,即只有第一个单元格下面有横线
    %\cline{1-1}
    \hline
    
        %%% 第7行的表格信息
        %content 7 1 &
        %%% 第7行由于第2~4列在上一行已被合并,所以建议不要写入信息,虽然写入信息也会显示出,而且建议像下面一样合并表格防止竖线分割
        %\multicolumn{3}{| c }{} \\

       \begin{center}
           硬件组
       \end{center} &
       \begin{center}
           分电方案
       \end{center} &
       \begin{center}
           设计48v转24v分电板
       \end{center} &
       \begin{center}
           电压稳定输出
       \end{center} \\

    \hline
    
\end{longtable}
%%% 表格演示-END
%%% 其中,开头的 { X | X | X | X |} 中,竖线标识分割线,字母X为自动对齐,其余可替换的lrc分别为左对齐、右对齐、居中对齐
%%% 该表格的一大特性是跨页后仍保留表头内容,即 \endhead 前的内容