%%% 本文档作为功能演示

%%% 表格演示-BEGIN
\begin{longtable}{ p{2cm} | p{7.8cm} | p{6cm} |}

    \hline

    %%% 此处标识以上为表格脚,即在每页的表格底部重复显示的内容(一条横线)
    \endfoot
    
    %%% 此处标识该行颜色为指定定义的颜色,tabhdcolor 在 ./introduction.tex 中定义,颜色为hsb(0, 0, 0.82353),即82度灰
    \rowcolor{tabhdcolor}

        %%% 第1行的表格信息
        \begin{center}
            功能
        \end{center} &
        \begin{center}
            需求分析
        \end{center} &
        \begin{center}
            设计思路
        \end{center} \\

    %%% 此处完整分割一行
    \hline

    %%% 此处标识以上为表格头,即在每页的表格头部重复显示的内容(一堆header)
    \endhead

        %%% 第2行的表格信息
        \begin{center}
            稳定的飞行
        \end{center} &
        \begin{center}
            需要有牢固的机架与合理配平与调试
        \end{center} &
        \begin{center}
            通过建模加上计算机仿真,逐步迭代到稳定
        \end{center} \\
        
    \hline

        \begin{center}
            稳定的发弹供弹
        \end{center} &
        \begin{center}
            需要有合理的设计,是必要的需求
        \end{center} &
        \begin{center}
            沿用鹅颈的供弹方式,并且在原有的基础上测试修改迭代
        \end{center} \\
        
    \hline
    
        \begin{center}
            能够控制功率全场飞行十分钟 
        \end{center} \cellcolor{gndcolor} &
        \begin{center}
            需求较后,在有余力可以开发
        \end{center} \cellcolor{gndcolor} &
        \begin{center}
            消耗资源比较大,硬件给出合适的分电方案,需要比较多的测试
        \end{center} \cellcolor{gndcolor} \\

    %%% 此处分割一行的第2至3个单元格
    \hline
    %\cline{2-3}
    
        \begin{center}
            自稳
        \end{center} &
        \begin{center}
            使用guidance或是光流定高
        \end{center} &
        \begin{center}
            guidance、光流数据融合
        \end{center} \\
        
    \hline
    
        \begin{center}
            自瞄
        \end{center} &
        \begin{center}
            击打敌方前哨站和地面单位
        \end{center} &
        \begin{center}
            移植步兵自瞄代码
        \end{center} \\
        
    \hline
    
\end{longtable}
%%% 表格演示-END
%%% 其中,开头的 { X | X | X | X |} 中,竖线标识分割线,字母X为自动对齐,其余可替换的lrc分别为左对齐、右对齐、居中对齐
%%% 该表格的一大特性是跨页后仍保留表头内容,即 \endhead 前的内容